% !TeX root = bachelorarbeit.tex
Nachdem wir in Abschnitt 2 an einige Grundlagen erinnert, in Abschnitt 3 und 4 sowohl die Monte Carlo Methode, als auch die Multilevel Monte Carlo Methode eingeführt und in Abschnitt 5 das Transportproblem, sowie das Potentialströmungsproblem inklusive numerischer Verfahren, welche zur Lösung dergleichen genutzt werden können, erklärt haben, soll nun dieser Abschnitt dazu dienen, die bisherigen Ergebnisse zu bündeln und die Anwendung der Multilevel Monte Carlo Methode auf partielle Differentialgleichungen am Beispiel des Transportproblem nahe zu legen. Dabei nimmt dieser Abschnitt einen zentralen Platz in dieser Thesis ein, weswegen wir noch einmal darlegen wollen, was genau unser Ziel ist und wie wir die uns an dieser Stelle zu Verfügung stehenden Mittel einsetzen, um das Gewünschte zu erreichen.
\begin{align*}
&\text{Für ein stochastisches Flussvektorfeld } q: \Omega \times \overline{\mathbb{D}} \to \R^2 \text{, bestimme }\rho: \Omega \times \overline{\mathbb{D}} \times \mathbb{T} \to \R_{\geq 0} \text{ mit} \\
&\text{(pTP)} 
\begin{cases}
\begin{array}{rlll}
\partial_t \rho (\omega, x, t) + \dive(\rho(\omega,x,t)q(\omega,x)) &= 0 &\text{, für } (x,t) \in \mathbb{D} \times (0,T] \\
\rho(\omega,x,t) &= \rho_{\text{in}}(x,t) &\text{, für } (x,t) \in \Gamma_{\text{in}} \times \mathbb{T} \\
\rho(\omega,x,0)  &= \rho_0(x) &\text{, für } x \in  \mathbb{D}
\end{array}
\end{cases} \\
&\text{ für die Anfangs- und Randwerte: } \\ 
&\begin{array}{llr}
g_N&: \Gamma_{\text{N}} \to \R \\
u_D&: \Gamma_{\text{D}} \to \R \\
\rho_{\text{in}}&: \Gamma_{\text{in}} \times \mathbb{T} \to \R_{\geq0} \\
\rho_0&: \mathbb{D} \to \R_{\geq0} \\
\end{array} \newline \\
&\text{ wobei } \partial \mathbb{D} = \Gamma_{\text{D}} \dot{\cup} \Gamma_{\text{N}}  \text{ und }  \Gamma_{\text{in}} \coloneqq  \{ z \in \partial \mathbb{D}: q(z)\cdot n(z) \leq 0 \} \subset  \partial \mathbb{D}
\end{align*}
Genauer sei $ Q(\omega) = J(\rho(\omega)) $ ein gegebenes Zielfunktional, dann ist es unser Ziel den Erwartungswert $ \mathbb{E}[Q(\omega)] $ möglichst genau zu bestimmen. Beispielsweise kann eine beliebige Norm von $ \rho(\omega) $ als Zielfunktional betrachtet werden.
Unser Modellproblem lautet also:

\begin{align}
\text{(MP)}
\begin{cases}
\label{Modellproblem}
&\text{Für ein stochastisches Flussvektorfeld } q: \Omega \times \overline{\mathbb{D}} \to \R^2 \\
&\text{und ein Zielfunktional } J \text{ , bestimme }  \mathbb{E}[J(\rho)]  \\
&\text{mit } \rho \text{ als Lösung von (pTP) inklusive Anfangs- und Randwerten}
\end{cases}
\end{align} 
Dabei erhalten wir das stochastische Flussvektorfeld $ q $, wie bereits in \ref{Probabilistisches Problem} erklärt selbst als Lösung des Potentialströmungsproblem.
Da wir an der numerischen Lösung partieller Differentialgleichungen interessiert sind und wir die im Allgemeinen unendlich dimensionale Lösung $ \rho $ durch eine endlich dimensionale Lösung $ \rho_{h,\Delta t} $ approximieren, betrachten wir $ Q_{h,\Delta t}(\omega) \coloneqq J(\rho_{h,\Delta t}(\omega )) $.
Wir nutzen dabei das in Abschnitt \ref{DG} behandelte Discontinuous Galerkin Verfahren zur Lösung des linearen Transportproblems. Außerdem werden wir im Folgenden eine uniforme Familie $ \{ \mathcal{T}_h \} $ von Zerlegungen von $ \mathcal{D} $ (vgl. Abschnitt \ref{num_pot} Definition \ref{FEMDISC}) als Diskretisierungsgitter für die Ortsdiskretisierung, sowie $ \mathbb{T}_{\Delta t} $ als Zerlegung von $ \mathbb{T} $ betrachten.
Insbesondere wählen wir später $ \Delta t $ in Abhängigkeit von $ h $, etwa $ \Delta t = c  h $ für ein $ c>0 $ und betrachten dann
$ Q_h(\omega) \coloneqq Q_{h,ch}(\omega) = J(\rho_h(\omega )) \coloneqq J(\rho_{h,ch}(\omega )) $.
An dieser Stelle sei betont, dass es sich in diesem Abschnitt bei $ \rho_h $ um eine volldiskrtisierte Approximation an $ \rho $ handelt und nicht die Semidiskretisierung aus Abschnitt \ref{DG} gemeint ist.
\begin{Annahme}(Konvergenz im Erwartungswert des Zielfunktionals)\\
	In obiger Situation gelte 
	\[ 
	\mathbb{E}[Q_h] \to \mathbb{E}[Q] \text{ für } h \to 0   
	\]
\end{Annahme}


\subsection{Die Monte Carlo Methode}
Wie bereits im Kontext der numerischen Integration wollen wir die Monte Carlo Methode als Ausgangspunkt nutzen und anschließend bei der Betrachtung der Multilevel Monte Carlo Methode auch auf entscheidende Unterschiede zu und Vorteile gegenüber der Monte Carlo Methode eingehen.
Sowohl bei der Monte Carlo Methode, als auch bei der Multilevel Variante approximieren wir den Erwartungswert $ \mathbb{E}[Q_h] $ durch einen Schätzwert $ \widehat{Q}_h $. Um die Genauigkeit und die Kosten zu bemessen betrachten wir zum einen den sogenannten 'root mean square error' (RMSE) 
\begin{align}
	\label{RMSE}
	e( \widehat{Q}_h) \coloneqq \left(  \mathbb{E} \left[ (\widehat{Q}_h - \mathbb{E}[Q] )^2 \right] \right)^{\frac{1}{2}}
\end{align}
zum anderen die Anzahl an floating-point-Rechenoperationen $ C_{\epsilon}(\widehat{Q}_h) $, die benötigt werden um einen RMSE mit $ e( \widehat{Q}_h) \leq \epsilon $ zu erhalten.
Zu beachten ist hierbei, dass in dem RMSE einige Fehlerquellen gemeinsam betrachtet werden. So gehen sowohl der Finite Elemente Fehler des Potentialströmungsproblems (Approximation von $ q $ durch $ q_h $), der Discontinuous Galerkin Fehler des Transportproblems (Approximation von $ \rho $ durch $ \rho_h $), der Approximationsfehler der Approximation von $ Q $ durch $ Q_h $, als auch der statistische Fehler des Schätzers $\widehat{Q}_h$ in den RMSE mit ein. Insbesondere bedeutet also $ e( \widehat{Q}_h) \leq \epsilon $, dass alle oben genannten Fehlerquellen kleiner als $ \epsilon $ ausfallen. Betrachten wir also, wie bei der Monte Carlo Methode, welche wir gleich noch einmal kurz behandeln wollen, nur eine einzige Zerlegung $ \mathcal{T} $ der Familie $ \mathcal{T}_h $, so kann es sein, dass es gar nicht möglich ist den RMSE durch ein bestimmtes $ \epsilon $ zu beschränken, da einer der Approximationsfehler für das gewählte feste Level bereits alleine größer als $ \epsilon $ ist. Wir wollen diesen Sachverhalt im Hinterkopf behalten, denn auch bei der Multilevel Monte Carlo  Methode werden wir später initiale Level wählen, in der konkreten Anwendung geben wir aus praktischen Gründen sogar ein maximales Level an, welches wiederum unter Umständen einen RMSE $ e( \widehat{Q}_h) \leq \epsilon $ verhindern kann.\\
Bei der Standard Monte Carlo Methode schätzen wir $ \mathbb{E}[Q] $ durch den Mittelwert  $ n $ unabhängiger gleichverteilter Zufallssamples und erhalten so 
\begin{align}
	\label{MC-Schätzer}
	\widehat{Q}_{h,n}^{\text{MC}} \coloneqq \frac{1}{n} \sum_{i=1}^{n} Q_h(\omega_i) = \sum_{i=1}^{n} J(\rho_h(\omega_i))
\end{align}
Dabei modellieren wir das zufällige Flussvektorfeld $ q(\omega_i)  : \mathbb{D} \to \R^2 $, indem wir zunächst für $ \kappa(\omega_i) : \mathbb{D} \rightarrow (\R_{\text{sym}})^{d \times d} $ ein lognormal-verteiltes unabhängiges Zufallsfeld erzeugen und anschließend das Potentialströmungsproblem
\begin{align*}
\setlength\arraycolsep{1pt}
&\text{Für } \kappa(\omega_i) : \mathbb{D} \rightarrow (\R_{\text{sym}})^{d \times d} \text{, bestimme } u(\omega_i,\cdot):\overline{\mathbb{D}} \to \R \text{ und } q(\omega_i, \cdot): \overline{\mathbb{D}} \to \R^2 \text{ mit } \\
&\text{(PS)}
\begin{cases}
\begin{array}{rlll}
\dive (q(\omega_i,x)) &= 0  &\text{, für } x \in \mathbb{ D}\\  
q(\omega_i,x) &= - \kappa(\omega_i) \nabla u(\omega_i,x)  &\text{, für } x \in \mathbb{D}\\
-q(\omega_i,x) \cdot n &= g_N(x)  &\text{, für } x \in \Gamma_N \\
u(\omega_i,x) &= u_D(x)  &\text{, für } x \in \Gamma_D \\
\end{array}
\end{cases} \\
\end{align*}
lösen. Zur Erzeugung des lognormal-verteilten Zufallsfeldes können wir auf entsprechende Algorithmen zurückgreifen, in unserem Fall etwa dem sogenannten Circulant Embedding.
Der Algorithmus wurde 1997 erstmals in \cite{dietrich1997fast} vorgestellt und 
erzeugt Gauß'sche Zufallsfelder auf regulären Gittern und basiert auf der Fast Fourier Transformation, welche in der Literatur auch oft unter der Abkürzung FFT zu finden ist.
Dabei werden spezielle Strukturen der Kovarianzmatrix ausgenutzt. Anschließend kann das Gauß'sche Zufallsfeld über eine einfache Transformation in ein lognormal Feld überführt werden. Mehr zu Circulant Embedding findet sich z.B. in \cite{schmidt2014stochastic} Abschnitt 12.
Auch die grundsätzliche Idee Circulant Embedding für die Modellierung der Ausgangsdaten in stochastischen partiellen Differentialgleichungen zu nutzen ist keineswegs neu und findet sich z.B. in \cite{charrier2012strong} oder \cite{cliffe2011multilevel}.
Wir haben nun alle Mittel in der Hand um die Monte Carlo Methode angewandt auf das Transportproblem als Algorithmus zu formulieren. Dabei fassen wir die eben erklärte Erzeugung des zufälligen Vektorfeldes in der Funktion 'RndVecField' zusammen:

\begin{algorithm}[H]
	\DontPrintSemicolon
	\SetAlgoLined
	%\KwResult{}
	\SetKwInOut{Input}{Input}\SetKwInOut{Output}{Output}
	\Input{$h,n$}
	\Output{$\widehat{Q}_{h,n}^{\text{MC}}$}
	\BlankLine
	Initialisiere: $ \Sigma =0, i=0 $\;
	\While{$i<n$}{
		Erzeuge ein Zufallssample: $ q(\omega_i,x)  \leftarrow $ RndVecField\;
		Löse das Transportproblem: $ \rho_{h}(\omega_i,x)  \leftarrow$ Löse mit Discontinuous Galerkin\;
		Berechne das zugehörige Zielfuinktional: $ Q_h(\omega_i) \leftarrow$ Berechne Zielfunktional\;
		Setze: $ \Sigma = \Sigma + Q_h(\omega_i) $, i = i+1\;
	}
	\BlankLine
	\KwResult{$\widehat{Q}_{h,n}^{\text{MC}} = \Sigma /n$}
	\caption{Monte Carlo Methode angewandt auf das Transportproblem}
\end{algorithm}
\bigskip % add 12pt space in-between
 Wir nehmen an dieser Stelle an, dass für die Gesamtkosten an Rechenoperationen, welche für die Berechnung eines $ Q_h(\omega_i) $ sich durch
 \[
 	C(Q_h(\omega_i))  \lesssim h^{- \lambda}
 \]
 Wir können den 'mean square error' $ e(\widehat{Q}_{h,n}^{\text{MC}})^2 $ auch folgendermaßen betrachten:
 \begin{align}
 	e(\widehat{Q}_{h,n}^{\text{MC}})^2 &= \mathbb{E} \left[ \left( \widehat{Q}_{h,n}^{\text{MC}} -  \mathbb{E}[\widehat{Q}_{h,n}^{\text{MC}}] + \mathbb{E}[\widehat{Q}_{h,n}^{\text{MC}}] - \mathbb{E}[Q] \right) \right] \nonumber \\
 	&= \mathbb{E} \left[ ( \widehat{Q}_{h,n}^{\text{MC}} -    \mathbb{E}[\widehat{Q}_{h,n}^{\text{MC}}]])^2 \right] + \left( \mathbb{E}[\widehat{Q}_{h,n}^{\text{MC}}] - \mathbb{E}[Q] \right)^2 \nonumber \\
 	&= \mathbb{V}[\widehat{Q}_{h,n}^{\text{MC}}] + \left( \mathbb{E}[\widehat{Q}_{h,n}^{\text{MC}}] - \mathbb{E}[Q] \right)^2
 \end{align}
 Da weiter $ \mathbb{E}[\widehat{Q}_{h,n}^{\text{MC}}] = \mathbb{E}[Q_h] $ und $ \mathbb{V}[\widehat{Q}_{h,n}^{\text{MC}}] = \frac{1}{n^2} n \mathbb{V}[Q_h] $ gilt, erhalten wir so 
 
 
 
 
\bigskip % add 12pt space in-between
 
 
 
%\begin{itemize}
%	\item Theorie Konvergenzannahmen
%	\item Resultate Giles + parrallelFEM Paper
%	\item Algorithmus
%\end{itemize}
%
%\begin{algorithm}[H]
%	\DontPrintSemicolon
%	\SetAlgoLined
%	%\KwResult{}
%	\SetKwInOut{Input}{Input}\SetKwInOut{Output}{Output}
%	\Input{h,n}
%	\Output{$\widehat{Q}_{h,n}^{\text{MC}}$}
%	\BlankLine
%	
%	\While{While condition}{
%		instructions\;
%		\eIf{condition}{
%			instructions1\;
%			instructions2\;  
%		}{
%			instructions3\;
%		}
%	}
%	\BlankLine
%	\KwResult{$\widehat{Q}_{h,n}^{\text{MC}} = \sum/n$}
%	\caption{While loop with If/Else condition}
%\end{algorithm}
%
%\bigskip % add 12pt space in-between
%

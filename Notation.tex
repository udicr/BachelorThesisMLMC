% !TeX root = bachelorarbeit.tex
Folgende Tabelle soll weder stichhaltige Definitionen festlegen, noch die gesamte Notation der Thesis bis auf den letzten Index ausarbeiten. Sie soll dem Leser eher als Orientierung dienen, welche Symbole für die verschiedenen Teilbereiche genutzt werden.

\begin{longtable}[c]{ p{0.15\textwidth} p{0.84\textwidth}}
	\hline
	Symbol    & Beschreibung   \\
	\hline
	$ \dive $ & Divergenz \\
	$ \nabla $ & Gradient \\
	$ \mathcal{D} \subset \R^d$      & Rechengebiet       \\
	$ \nu $ & äußerer Normalenvektor \\
	$ d \in \N $      & Dimension des Rechengebietes        \\
	$ d_1, d_2 \in \N $ & Dimensionen des Integrations- und Testraumes in Beispiel \ref{mlmcbeispiel}  \\
	$ (\Omega,\mathcal{A},\mathbb{P}) $       & Wahrscheinlichkeitsraum $ \Omega $ über der $ \sigma $-Algebra $ \mathcal{A} $ mit Wahrscheinlichkeitsmaß $ \mathbb{P} $       \\
	$ \mathbb{T} \coloneqq  (0,T]  $  &  Zeitintervall für ein $ T>0 $       \\
	$ f $ & skalarwertige Funktion \\
	$ F $ & vektorwertige Funktion \\
	$ \phi , \psi $ & Testfunktionen \\
	$ \omega ,\omega_i , \dots \in \Omega $ & Ereignis \\
	$ X , X_i , \dots $ & Zufallsvariable bzw. Zufallsvektor \\
	$ x , x_i , \dots $ & Realisierung der zugehörigen Zufallsvariable bzw. des Zufallsvektors \\
	$ \mathfrak{N} $ & Nullmenge bzgl. $ \mathbb{P} $ \\
	$ \mathcal{N}(\mu,\sigma^2) $ & Normalverteilung mit Parametern $ \mu $ und $ \sigma^2 $ \\
	$ \mathcal{N}_n(\mu,\sigma^2) $ & multivariate Normalverteilung eines $ n $-dim Zufallsvektors\\
	$\widetilde{\mathcal{N}}$ & standardnormalverteilte Zufallsvariable \\
	$ \mathcal{U}([a,b]) $ & Gleichverteilung auf dem Intervall $ [a,b] $ mit $ a<b $ \\
	$ U,U_i,\dots $ &  auf $ [0,1] $  gleichverteilte Zufallsvariablen \\
	$ \mathbb{E}[X] $ & Erwartungswert der Zufallsvariable X \\
	$ \mathbb{V}[X] $ & Varianz der Zufallsvariablen X \\
	$ n $ & Anzahl der betrachteten Zufallsvariablen, später auch Anzahl der betrachteten Samples \\
	$ l,\dots,L \in \N_0$ & Level, beginnend mit $ l=0 $ bis $ L $ \\
	$ l_0, \dots , L_{\text{max}} \in \N_0 $ & Level, beginnend mit $ l_0 \stackrel{i.Allg.}{\not =} 0 $ bis $ L_{\text{max}} $  \\
	$ P $ & Interpolationsoperator aus Beispiel \ref{mlmcbeispiel} \\
	$ Q,J $ & Zielfunktionale \\
	$ \kappa $ & Permeabilitätstensor \\
	$ q $ & Flussvektorfeld \\
	$ \Gamma_{\text{D}} $ & Dirichletrand \\
	$ \Gamma_{\text{N}} $ & Neumannrand \\
	$ \Gamma_{\text{in}} $ & Einflussrand \\
	$ h $ & Schrittweite Ortsdiskretisierung \\
	$ \Delta t $ & Schrittweite Zeitdiskretisierung \\
	$ \rho $ & Konzentration eines Stoffes im Rechengebiet \\
	$ \alpha,\beta,\gamma $ & Konvergenzparameter aus Satz \ref{MLMCTheorem} \\
	$ e(\cdot) $ & Fehlerfunktion, z.B. RMSE \\
	$ \Upsilon $ & Flussfunktion \\
	$ \Upsilon^{\star} $ & numerische Flussfunktion \\
	$ \mathcal{T} $ & Zerlegung von $ \mathcal{D} $ \\
	$ N= \abs{\mathcal{T}} $ & Anzahl der Zellen der Zerlegung $ \mathcal{T} $\\
	$ \mathcal{F} $ & Menge aller Seiten der Zerlegung $ \mathcal{T} $ \\
	$ M = \abs{\mathcal{F}} $ & Anzahl der Seiten der Zerlegung $ \mathcal{T} $ \\
	$ \mathcal{V}_{\mathcal{T}} $ & Menge aller Knoten der Zerlegung $ \mathcal{T} $\\
	$ {\psi_i} $ & Seitenbasis \\
	$ {\mu_i} $ & Zellenbasis \\
	$ C(\cdot) , C_l , \dots $ & Kosten oder Kostenfunktion \\
	$ W^{k,p},W_0^{k,p} $ & Sobolevräume (vgl. 2.1) \\
	$ H^1,H_0^1$ & Sobolevräume (vgl. 2.1) \\
	$ L^1_{loc}(\mathcal{ D}) $ & Raum der lokal integrierbaren Funktionen auf $ \mathcal{ D} $ \\
	$ C_c^{1} $ & Raum stetig differenzierbarer Funktionen mit kompaktem  Träger \\
	$ C_c^{\infty} $ & Raum unendlich oft stetig differenzierbarer Funktionen mit kompaktem  Träger \\
	$ W,W_h,\mathcal{Q},\mathcal{Q}_h $ & Test- und Ansatzräume der Finite Elemente Verfahren, jeweils im entsprechenden Abschnitt definiert\\
	$ (K,\mathcal{P},\mathcal{N}) $ & finites Element, oft auch nur $ K \in \mathcal{T}$ als Zelle \\
	$ G $ & Anzahl der Zellenfreiheitsgrade \\
	\hline
\end{longtable}


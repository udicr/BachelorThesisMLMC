% !TeX root = bachelorarbeit.tex

In diesem Abchnitt soll nun, nachdem wir $q(\omega,\cdot)$ als Finite-Elemente-Lösung des Potentialströmungsproblems erhalten haben, die numerische Lösung des linearen Transportproblems behandelt werden:
\begin{align*}
	&\text{Für } \omega \in \Omega \text{ und }q(\omega,\cdot): \overline{\mathbb{D}} \to \R^2 \text{, bestimme }\rho(\omega,\cdot): \overline{\mathbb{D}} \times \mathbb{T} \to \R_{\geq 0} \text{ mit} \\
	&\text{(TP)} 
	\begin{cases}
	\begin{array}{rlll}
	\partial_t \rho (\omega, x, t) + \dive(\rho(\omega,x,t)q(\omega,x)) &= 0 &\text{, für } (x,t) \in \mathbb{D} \times (0,T] \\
	\rho(\omega,x,t) &= \rho_{\text{in}}(x,t) &\text{, für } (x,t) \in \Gamma_{\text{in}} \times \mathbb{T} \\
	\rho(\omega,x,0)  &= \rho_0(x) &\text{, für } x \in  \mathbb{D}
	\end{array}
	\end{cases} \\
\end{align*}
Insbesondere wollen wir an dieser Stelle wieder $\omega \in \Omega$ fest halten und betrachten deshalb zunächst nur das deterministische Problem wie in \ref{det_prob}.
Wie im letzten Abschnitt führen wir zudem einen schwachen Lösungsbegriff ein:


===TODO SPACING + Herleitung====
\begin{align*}
\text{Gegeben } q(\cdot) : \overline{\mathbb{D}} \to \R^2 \text{, bestimme } \rho:\Omega\times [0,T] \to \R_{\ge 0} \text{ so, dass}\\
\text{(swTP)}\begin{cases}
\displaystyle
\int_{\Omega} \rho_0 \phi(0) \dx = \mkern-16mu &- \displaystyle \int_{0}^{T} \int_{\Omega} \rho (\partial_t \phi + q \nabla \phi ) \dx \dt \\
&+\displaystyle\int_{0}^{T}  \int_{\Gamma_{\text{in}}} \rho_{\text{in}} q \cdot \nu \phi \da  \dt \\
&+\displaystyle\int_{0}^{T}  \int_{\Gamma_{\text{out}}} \rho q \cdot \nu \phi \da  \dt
\end{cases}	\\
\text{ für alle } \phi: \Omega \times (0,T) \to \R \text{ mit } \phi(\cdot,T) = 0 \text{ auf } \Omega.
\end{align*}

Dabei hängen der klassische und der schwache Lösungsbegriff folgendermaßen miteinander zusammen:
\begin{Lemma}(Zusammenhang der Lösungsbegriffe)
	
	\begin{enumerate}[label=(\alph*)]
		\item Ist $ \rho $ eine klassische Lösung, so ist $ \rho $ auch eine schwache Lösung.
		\item Ist $ \rho $ glatt genug (TODO) und eine schwache Lösung, so ist $ \rho $ eine klassische Lösung. 
	\end{enumerate}
\end{Lemma}

%\begin{Beweis}
 %  todo
%\end{Beweis}
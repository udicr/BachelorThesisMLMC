% Vorlage für eine Bachelorarbeit
% Siehe auch LaTeX-Kurs von Mathematik-Online
% www.mathematik-online.org/kurse
% Anpassungen für die Fakultät für Mathematik
% am KIT durch Klaus Spitzmüller und Roland Schnaubelt Dezember 2011

\documentclass[12pt,a4paper]{scrartcl}
% scrartcl ist eine abgeleitete Artikel-Klasse im Koma-Skript
% zur Kontrolle des Umbruchs Klassenoption draft verwenden


% die folgenden Packete erlauben den Gebrauch von Umlauten und ß
% in der Latex Datei
\usepackage[utf8]{inputenc}
% \usepackage[latin1]{inputenc} %  Alternativ unter Windows
\usepackage[T1]{fontenc}
\usepackage[ngerman]{babel}


\usepackage[pdftex]{graphicx}
\usepackage{latexsym}
\usepackage{amsmath,amssymb,amsthm}
\usepackage{setspace}
\usepackage{url}



% Abstand obere Blattkante zur Kopfzeile ist 2.54cm - 15mm
\setlength{\topmargin}{-15mm}


% Umgebungen für Definitionen, Sätze, usw.
% Es werden Sätze, Definitionen etc innerhalb einer Section mit
% 1.1, 1.2 etc durchnummeriert, ebenso die Gleichungen mit (1.1), (1.2) ..
\newtheorem{Satz}{Satz}[section]
\newtheorem{Definition}[Satz]{Definition} 
\newtheorem{Lemma}[Satz]{Lemma}		   
                  
\numberwithin{equation}{section} 

% einige Abkuerzungen
\newcommand{\C}{\mathbb{C}} % komplexe
\newcommand{\K}{\mathbb{K}} % komplexe
\newcommand{\R}{\mathbb{R}} % reelle
\newcommand{\Q}{\mathbb{Q}} % rationale
\newcommand{\Z}{\mathbb{Z}} % ganze
\newcommand{\N}{\mathbb{N}} % natuerliche

% eigene Definitionen
\DeclareMathOperator{\lTG}{(TG)}
\DeclareMathOperator{\dive}{div}


\begin{document}
  % Keine Seitenzahlen im Vorspann
  \pagestyle{empty}
  
  
  % Titelblatt der Arbeit
  \begin{titlepage}

    \includegraphics[scale=0.45]{kit-logo.jpg} 
    \vspace*{2cm} 

 \begin{center} \large 
    
    Bachelorarbeit
    \vspace*{2cm}

    {\huge Die Multilevel Monte Carlo Methode und deren Anwendung am Beispiel der linearen Transportgleichung}
    \vspace*{2.5cm}

    Tim Buchholz
    \vspace*{1.5cm}

    ??.??.??
    \vspace*{4.5cm}


    Betreuung: Prof.Dr. Christian Wieners und M.Sc. Niklas Baumgarten \\[1cm]
    Fakultät für Mathematik \\[1cm]
		Karlsruher Institut für Technologie
  \end{center}
\end{titlepage}



  % Inhaltsverzeichnis
  \tableofcontents

\newpage
 


  % Ab sofort Seitenzahlen in der Kopfzeile anzeigen
  \pagestyle{headings}

\section{Einleitung}


Monte Carlo Methoden sind weit verbreitet und finden in verschiedenen Bereichen der Mathematik ihre Anwendung.
Neben der numerischen Integration können sie unter anderem auch bei der Lösung von partiellen Differentialgleichungen auf Grundlage unsicherer Daten genutzt werden.
So entstehende Problemstellungen fallen in das Gebiet der Uncertainty Quantification, einem 'Zusammentreffen der Wahrscheinlichkeitstheorie, Numerik, Statistik und der echten Welt' \cite{sullivan2015introduction}.
Allerdings besitzt die Monte Carlo Methode einen entscheidenden Nachteil, will man sie im Zusammenhang unsicherer Ausgangsdaten für die Lösung von partiellen Differentialgleichungen nutzen, sie konvergiert im Normalfall relativ langsam und das (numerische) Lösen von PDE's ist oft sehr aufwändig.
Es werden also unter Umständen sehr viele, sehr teure (Zufalls-)Samples benötigt, um ein vernünftiges Ergebnis zu erhalten. \newline
Diese Thesis soll sich daher mit der Multilevel Monte Carlo Methode (im Folgenden MLMC Methode genannt) beschäftigen, welche an die Monte Carlo Methode angelehnt ist, aber durch die geschickte Auswertung der (Zufalls-Samples) deutliche Effizienzvorteile gegenüber der Standard Monte Carlo Methode besitzt.
Die MLMC Methode soll, nach einer ausführlichen theoretischen Analyse auch praktisch auf das (lineare) Transportproblem angewandt werden. \newline
Die Thesis ist dazu folgendermaßen unterteilt: 
Abschnitt 2 sammelt verschiedene Grundlagen aus den Bereichen der Stochastik, der Analysis und Numerik partieller Differentialgleichungen, sowie einige Aspekte der (standard) Monte Carlo Methoden, welche auch der MLMC Methode als theoretischer Unterbau dienen soll. \newline
In Abschnitt 3 wird anschließend die MLMC Methode an sich erklärt, ohne die Theorie hierbei allzu sehr auf die Anwendung auf das Transportproblem zu beschränken. \newline
In Abschnitt 4 werden dann das (lineare) Transportproblem und das Potentialströmungsproblem beschrieben, welches wir lösen müssen, um an die entsprechenden Ausgangsdaten zu kommen. Anschließend wird die numerische Lösung der beiden Probleme mit Finite Elementen Methoden behandelt, bevor schließlich auf die Anwendung der MLMC Methode auf das Transportproblem mit unsicheren Ausgangsdaten am Beispiel der Permeabilität $\kappa$ eingegangen wird. \newline
Der fünfte und letzte Abschnitt befasst sich mit konkreten Durchführung und Implementierung des zuvor theoretisch beleuchteten Problem innerhalb der parallelen Finte Elemente Softwarebibliothek "M++" \cite{siteM++},
welche am Institut für Angewandte und Numerische Mathematik 3 (KIT) von Herrn Prof. Dr. C. Wieners entwickelt wurde. \newline
Am Schluss der Thesis steht eine kleine Zusammenfassung der bis dahin erarbeiteten Resultate und der Ausblick auf Möglichkeiten vielfältiger Art an diese in weiteren Arbeiten anzuknüpfen.

%%%%%%%%%%%%%%%%%%%%%%%%%%%%%%%%%
 \newpage  % neuer Abschnitt auf neue Seite, kann auch entfallen
%%%%%%%%%%%%%%%%%%%%%%%%%%%%%%%%%
 
\section{Grundlagen}
\subsection{analytische/numerische Grundlagen}
\subsection{stochastische Grundlagen}
\subsection{Monte Carlo Methoden}
\section{Multilevel Monte Carlo Methode (MLMC)}
\section{MLMC angewandt auf das Transportproblem}
\subsection{Problemstellung}
\subsection{Numerische Lösung des Potentialströmungsproblem}
\subsection{Numerische Lösung des Transportproblem}
\subsection{Anwendung der MLMC Methode auf das Transportproblem für unsichere Permeabilität $\kappa$ }
\section{Beispiel/Experiment}
\subsection{Konkretes Problem}
\subsection{Ergebnisse}
\section{Ausblick und Fazit}

  % Literaturverzeichnis (beginnt auf einer ungeraden Seite)
  \newpage

%\begin{thebibliography}{Lam00}
 %Bibiographie
  \bibliographystyle{abbrv}
  \bibliography{References}
%\end{thebibliography}
 
      
  % ggf. hier Tabelle mit Symbolen 
  % (kann auch auf das Inhaltsverzeichnis folgen)

\newpage
  
 \thispagestyle{empty}


\vspace*{8cm}


\section*{Erkl\"arung}

Ich  versichere  wahrheitsgem\"a\ss,  die  Arbeit selbstst\"andig verfasst,  alle  benutzten  Hilfsmittel  vollst\"andig  und  genau  angegeben  und  alles kenntlich  gemacht  zu  haben,  was  aus  Arbeiten  anderer  unver\"andert  oder  mit  Ab\"anderungen entnommen  wurde,  sowie die Satzung  des  KIT  zur  Sicherung guter wissenschaftlicher Praxis in der jeweils g\"ultigen Fassung beachtet zu haben.
\\[2ex] 

\noindent
Ort, den Datum\\[5ex]

% Unterschrift (handgeschrieben)



\end{document}


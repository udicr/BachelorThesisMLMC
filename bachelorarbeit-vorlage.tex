% Vorlage für eine Bachelorarbeit
% Siehe auch LaTeX-Kurs von Mathematik-Online
% www.mathematik-online.org/kurse
% Anpassungen für die Fakultät für Mathematik
% am KIT durch Klaus Spitzmüller und Roland Schnaubelt Dezember 2011

\documentclass[12pt,a4paper]{scrartcl}
% scrartcl ist eine abgeleitete Artikel-Klasse im Koma-Skript
% zur Kontrolle des Umbruchs Klassenoption draft verwenden


% die folgenden Packete erlauben den Gebrauch von Umlauten und ß
% in der Latex Datei
\usepackage[utf8]{inputenc}
% \usepackage[latin1]{inputenc} %  Alternativ unter Windows
\usepackage[T1]{fontenc}
\usepackage[ngerman]{babel}


\usepackage[pdftex]{graphicx}
\usepackage{latexsym}
\usepackage{amsmath,amssymb,amsthm}
\usepackage{setspace}
\usepackage{url}
\usepackage{mathtools}


% Abstand obere Blattkante zur Kopfzeile ist 2.54cm - 15mm
\setlength{\topmargin}{-15mm}


% Umgebungen für Definitionen, Sätze, usw.
% Es werden Sätze, Definitionen etc innerhalb einer Section mit
% 1.1, 1.2 etc durchnummeriert, ebenso die Gleichungen mit (1.1), (1.2) ..
\newtheorem{Satz}{Satz}[section]
\newtheorem{Definition}[Satz]{Definition} 
\newtheorem{Lemma}[Satz]{Lemma}		   
                  
\numberwithin{equation}{section} 

\newtheorem*{Bemerkung}{Bemerkung}

% einige Abkuerzungen
\newcommand{\C}{\mathbb{C}} % komplexe
\newcommand{\K}{\mathbb{K}} % komplexe
\newcommand{\R}{\mathbb{R}} % reelle
\newcommand{\Q}{\mathbb{Q}} % rationale
\newcommand{\Z}{\mathbb{Z}} % ganze
\newcommand{\N}{\mathbb{N}} % natuerliche

% eigene Definitionen
\DeclareMathOperator{\lTG}{(TG)}
\DeclareMathOperator{\dive}{div}


\begin{document}
  % Keine Seitenzahlen im Vorspann
  \pagestyle{empty}
  
  
  % Titelblatt der Arbeit
  \begin{titlepage}

    \includegraphics[scale=0.45]{kit-logo.jpg} 
    \vspace*{2cm} 

 \begin{center} \large 
    
    Bachelorarbeit
    \vspace*{2cm}

    {\huge Die Multilevel Monte Carlo Methode und deren Anwendung am Beispiel der linearen Transportgleichung}
    \vspace*{2.5cm}

    Tim Buchholz
    \vspace*{1.5cm}

    ??.??.??
    \vspace*{4.5cm}


    Betreuung: Prof.Dr. Christian Wieners und M.Sc. Niklas Baumgarten \\[1cm]
    Fakultät für Mathematik \\[1cm]
		Karlsruher Institut für Technologie
  \end{center}
\end{titlepage}



  % Inhaltsverzeichnis
  \tableofcontents

\newpage
 


  % Ab sofort Seitenzahlen in der Kopfzeile anzeigen
  \pagestyle{headings}

\section{Einleitung}


Monte Carlo Methoden sind weit verbreitet und finden in verschiedenen Bereichen der Mathematik ihre Anwendung.
Neben der numerischen Integration können sie unter anderem auch bei der Lösung von partiellen Differentialgleichungen auf Grundlage unsicherer Daten genutzt werden.
So entstehende Problemstellungen fallen in das Gebiet der Uncertainty Quantification, einem 'Zusammentreffen der Wahrscheinlichkeitstheorie, Numerik, Statistik und der echten Welt' \cite{sullivan2015introduction}.
Allerdings besitzt die Monte Carlo Methode einen entscheidenden Nachteil, will man sie im Zusammenhang unsicherer Ausgangsdaten für die Lösung von partiellen Differentialgleichungen nutzen, sie konvergiert im Normalfall relativ langsam und das (numerische) Lösen von PDE's ist oft sehr aufwändig.
Es werden also unter Umständen sehr viele, sehr teure (Zufalls-)Samples benötigt, um ein vernünftiges Ergebnis zu erhalten. \newline
Diese Thesis soll sich daher mit der Multilevel Monte Carlo Methode (im Folgenden MLMC Methode genannt) beschäftigen, welche an die Monte Carlo Methode angelehnt ist, aber durch die geschickte Auswertung der (Zufalls-Samples) deutliche Effizienzvorteile gegenüber der Standard Monte Carlo Methode besitzt.
Die MLMC Methode soll, nach einer ausführlichen theoretischen Analyse auch praktisch auf das (lineare) Transportproblem angewandt werden.
Genauer soll für
\begin{itemize}
	\item ein beschränktes Gebiet $\mathbb{D} \subseteq \R^d$
	\item  ein Zeitintervall $\mathbb{T} = [0,T]$
	\item  ein (unsicheres) Flussvektorfeld $q: \overline{\mathbb{D}} \rightarrow \R^d$
	\item  eine Anfangskonzentration eines (zu transportierenden) Stoffes $\rho_0: \overline{\mathbb{D}} \rightarrow \R^d$
	\item einen Einfluss $\rho_{\text{in}} : \Gamma_{\text{in}} \times \mathbb{T} \rightarrow \R$ über den Einflussrand $\Gamma_{\text{in}} \coloneqq  \{ z \in \partial \mathbb{D}: q(z)\cdot n(z) \leq 0 \} \subset  \partial \mathbb{D}$ mit $n(z)$ als äußeren Normalenvektor im (Rand-)Punkt $z$
\end{itemize}
    die Konzentration des Stoffes $\rho: \overline{\mathbb{D}} \times \mathbb{T}  \rightarrow \R_{\geq0}$ als Lösung der folgenden (partiellen) Differentialgleichung bestimmt werden.
\begin{gather*}
\text{Bestimme } \rho: \overline{\mathbb{D}} \times \mathbb{T} \to \R_{\geq 0} \text{, sodass}\\
(\text{TP})
\begin{cases}
\partial_t \rho + \dive(\rho q) = 0 &\text{ in } \mathbb{D} \times (0,T)\\
\rho(x,t) = \rho_{\text{in}}(x,t) &\text{ auf } \Gamma_{\text{in}} \times (0,T)\\
\rho(x,0) = \rho_0(x) &\text{ auf } \mathbb{D}.
\end{cases}
\end{gather*}
Außerdem muss zunächst ein zwar zufälliges aber dennoch sinnvolles Vektorfeld $q$ erzeugt werden. Wir nutzen hierbei das Darcy-Gesetz $q = - \kappa (\nabla p + G)$, welches als Modellierung von Fluiden in porösen Bodenschichten bereits oft genutzt wurde (vgl. z.B. \cite{de1986quantitative}).
Dabei ist 
\begin{itemize}
	\item $p:\mathbb{D} \rightarrow \R$ der hydrostatische Druck
	\item $\kappa: \mathbb{D} \rightarrow (\R_{\text{sym}})^{d \times d}$ der Permeabilitätstensor 
	\item $G = (0,0,p_0 g_0)^{\top}$ der Gravitationsvektor	
\end{itemize}
und es soll später, bevor wir das eigentliche Transportproblem lösen, stets zunächst für ein zufälliges $\kappa$ ein entsprechendes Flussvektorfeld $q$ über das sogenannte Potentialströmungsproblem, welches sich aus dem Darcy-Gesetz ableitet, berechnet werden. 
Die genauere Modellierung des so entstehenden Gesamtproblems soll aber an späterer Stelle erfolgen. \newline
Die Thesis ist dazu folgendermaßen unterteilt:\newline 
Abschnitt 2 sammelt verschiedene Grundlagen aus den Bereichen der Stochastik, der Analysis und Numerik partieller Differentialgleichungen, sowie einige Aspekte der (standard) Monte Carlo Methoden, welche auch der MLMC Methode als theoretischer Unterbau dienen soll. \newline
In Abschnitt 3 wird anschließend die MLMC Methode an sich erklärt, ohne die Theorie hierbei allzu sehr auf die Anwendung auf das Transportproblem zu beschränken. \newline
In Abschnitt 4 werden dann das (lineare) Transportproblem und das Potentialströmungsproblem beschrieben, welches wir lösen müssen, um an die entsprechenden Ausgangsdaten zu kommen. Anschließend wird die numerische Lösung der beiden Probleme mit Finite Elementen Methoden behandelt, bevor schließlich auf die Anwendung der MLMC Methode auf das Transportproblem mit unsicheren Ausgangsdaten am Beispiel der Permeabilität $\kappa$ eingegangen wird. \newline
Der fünfte und letzte Abschnitt befasst sich mit konkreten Durchführung und Implementierung des zuvor theoretisch beleuchteten Problem innerhalb der parallelen Finte Elemente Softwarebibliothek "M++" \cite{siteM++},
welche am Institut für Angewandte und Numerische Mathematik 3 (KIT) von Herrn Prof. Dr. C. Wieners entwickelt wurde. \newline
Am Schluss der Thesis steht eine kleine Zusammenfassung der bis dahin erarbeiteten Resultate und der Ausblick auf Möglichkeiten vielfältiger Art an diese in weiteren Arbeiten anzuknüpfen.

%%%%%%%%%%%%%%%%%%%%%%%%%%%%%%%%%
 \newpage  % neuer Abschnitt auf neue Seite, kann auch entfallen
%%%%%%%%%%%%%%%%%%%%%%%%%%%%%%%%%
 
\section{Grundlagen}
\subsection{analytische/numerische Grundlagen}
\subsection{stochastische Grundlagen}
\subsection{Monte Carlo Methoden}
\section{Multilevel Monte Carlo Methode (MLMC)}
\section{MLMC angewandt auf das Transportproblem}
\subsection{Problemstellung}
Sei $\mathbb{T} = [0,T]$ ein Zeitintervall für $T>0$ und $\mathbb{D} \subset \R^d , d \in \N$ ein beschränktes, offenes und konvexes L-Gebiet mit Rand $ \partial \mathbb{D} = \Gamma_{\text{D}}  \dot{\cup} \Gamma_{\text{N}} $. 
Wie bereits in der Einleitung beschrieben, wollen wir den Transport eines Stoffes in einer porösen Bodenschicht auf Grundlage eines vorhandenen Flusses beschreiben. 
Als modellhaftes Problem soll uns hierfür die Regenwasserversickerung dienen: In einer porösen Bodenschicht befindet sich zum Zeitpunkt $t=0$ ein Stoff (beispielsweise Öl) in einer gegebenen Anfangskonzentration und -verteilung. Nun sickert Regenwasser in diese poröse Bodenschicht ein. Zusätzlich wollen wir weitere Zuflüsse des Fremdstoffes über den Einflussrand $\Gamma_{\text{in}} \subset \partial \mathbb{D}$ zulassen.
Wir sind letztendlich an der Konzentration dieser Substanz an einer Stelle $x \in \overline{\mathbb{D}}$ zu einem Zeitpunkt $t \in \mathbb{T}$ interessiert. \newline
Bevor wir uns allerdings dem eigentlichen Transportproblem widmen können, müssen wir zunächst das Flussvektorfeld $q: \overline{\mathbb{D}} \rightarrow \R^d$ berechnen. \newline
Sei hierfür $p: D \rightarrow \R$ der hydrostatische Druck, $\kappa: D \rightarrow (\R_{\text{sym}})^{d \times d}$ der Permeabilitätstensor und $G=(0,0,p_0 g_0)^{\top}$. 
Wie bereits in der Einleitung angedeutet, kann der Fluss des Regenwassers durch das Darcy-Gesetz $q=-\kappa(\nabla p + G)$ modelliert werden.
Durch $u(x) := p(x) + p_0 g_0 x_3$ vereinfacht sich das Darcy-Gesetz zu $q=-\kappa \nabla u$.\newline
Nehmen wir die physikalische Annahme hinzu, dass der Fluss $q$ "quellfrei" sein soll, also an keiner Stelle Masse verschwinden oder erscheinen kann, erhalten wir das Potentialströmungsproblem:
\begin{gather*}
\text{Bestimme } u:\overline{\mathbb{D}} \to \R \text{ und } q: \overline{\mathbb{D}} \to \R^2 \text{ mit } \\
\text{(PS)}\begin{cases}
\dive q &= 0 \text{ in } \mathbb{D}\\
q &= - \kappa \nabla u \text{ in } \mathbb{D}\\
u &= u_D \text{ auf } \Gamma_D \\
-q \cdot n &= g_N \text{ auf } \Gamma_N \\
\end{cases}
\end{gather*} 
\begin{Bemerkung}
	Wir wollen aus verschiedenen Gründen direkt die sogenannte gemischte Fromulierung des Potentialströmungsproblem nutzen. Näheres dazu findet sich im nächsten Abschnitt.
\end{Bemerkung}
Anschließend suchen wir die Dichteverteilung $\rho: \mathbb{D} \times \mathbb{T} \rightarrow \R_{\geq0} $ einer transportierten Substanz (in unserem Modell das Öl).  \newline
Gegeben sei dazu die Anfangsverteilung $\rho_0: \mathbb{D} \rightarrow \R_{\geq0}$ und der Einfluss der Substanz über die Zeit
$
	\rho_{\text{in}} : \Gamma_{\text{in}} \times \mathbb{T} \rightarrow \R_{\geq0} \text{ mit }  \Gamma_{\text{in}} \coloneqq  \{ z \in \partial \mathbb{D}: q(z)\cdot n(z) \leq 0 \} \subset  \partial \mathbb{D}
$.
Dabei ist $n(z)$ der äußere Normalenvektor im (Rand-)Punkt $z$.
Wir bedienen uns wieder der Physik und fordern die Erfüllung der Bilanzgleichung
\begin{align*}
	\forall K \subseteq \mathbb{D} , t\in \mathbb{T} : \frac{d}{dt} \int_K \rho(x,t) \, \mathrm{d}x + \int_{\partial K} \rho(x,t)q(x)\cdot n(x) \, \mathrm{d}a = 0
\end{align*}.
Wenden wir für ein zulässiges $K \subseteq \mathbb{D}$ und $\rho,q \in C^1(\mathbb{D})$ den Satz von Gauß an erhalten wir
\begin{align*}
	\int_K \partial_t \rho(x,t) + \dive (\rho q)(x,t) \, \mathrm{d}x = 0
\end{align*}
und können so die (lineare) Transportgleichung ableiten:
\begin{align*}
	\partial_t \rho + \dive (\rho q) = 0 \text{ in } \mathbb{D} \times (0,T]
\end{align*}
Mit den entsprechenden Rand- und Anfangswerten erhalten wir so:
\begin{gather*}
\text{Bestimme } \rho: \overline{\mathbb{D}} \times \mathbb{T} \to \R_{\geq 0} \text{, sodass}\\
(\text{TP})
\begin{cases}
\partial_t \rho + \dive(\rho q) = 0 &\text{ in } \mathbb{D} \times (0,T)\\
\rho(x,t) = \rho_{\text{in}}(x,t) &\text{ auf } \Gamma_{\text{in}} \times (0,T)\\
\rho(x,0) = \rho_0(x) &\text{ auf } \mathbb{D}.
\end{cases}
\end{gather*}
Nun ist es allerdings so, dass wir bereits bei der Lösung des Potentialströmungsproblems davon ausgehen sämtliche benötigten Randwerte sowie (vor allem) den Permeabilitätstensor $\kappa$ exakt für das gesamte Gebiet $\mathbb{D}$ kennen.
Klar sollte jedoch sein, dass dies in der Realität eigentlich nie der Fall ist.
Wir wollen deshalb zusätzlich die Permeabilität $\kappa$ mit Mitteln der Stochastik modellieren.
Sei dazu $(\Omega, \mathcal{A},\mathbb{P})$ ein Wahrscheinlichkeitsraum und ab nun $d=2$, also $\mathbb{D} \subseteq \R^2$.
\begin{Bemerkung}
	Grundsätzlich funktionieren alle vorgestellten Verfahren auch für $d=3$, wir wollen uns aber der Einfachheit halber auf zwei Dimensionen beschränken. Das so betrachtete Gebiet $\mathbb{D}$ lässt sich so z.B. als Querschnitt einer Bodenschicht interpretieren.
\end{Bemerkung} 
Weiter sei nun $\kappa (\cdot,x): \Omega \rightarrow \R_{\geq0}$ die (vom Zufall abhängige) Permeabilität.
Wie schon an anderer Stelle (z.B. in \cite{kumar2018multigrid} oder cite) wollen wir die Permeabilität als lognormal-Feld (TODO) modellieren.
Unser so entstehendes Problem fällt somit in den Bereich der Uncertainty Quantification und lautet (TODO SPACING): 
\begin{gather*}
\text{Für } \omega \in \Omega \text{, bestimme } u(\omega,\cdot):\overline{\mathbb{D}} \to \R \text{ und } q(\omega, \cdot): \overline{\mathbb{D}} \to \R^2 \text{ mit } \\
	\text{(PS)}
	\begin{cases}
		\dive (q(\omega,x)) = 0  &\text{ für } x \in \mathbb{ D}\\  
		q(\omega,x) = - \kappa(\omega) \nabla u(\omega,x)  &\text{ für } x \in \mathbb{D}\\
		-q(\omega,x) \cdot n = g_N(x)  &\text{ für } x \in \Gamma_N \\
		u(\omega,x) = u_D(x)  &\text{ für } x \in \Gamma_D \\
	\end{cases} \\
	\text{Für } \omega \in \Omega \text{ und }q(\omega,\cdot): \overline{\mathbb{D}} \to \R \text{, bestimme }\rho(\omega,\cdot): \overline{\mathbb{D}} \times \mathbb{T} \to \R_{\geq 0} \text{ mit} \\
	\text{(TP)} 
	\begin{cases}
		\partial_t \rho (\omega, x, t) + \dive(\rho(\omega,x,t)q(\omega,x)) = 0 &\text{ für } (x,t) \in \mathbb{D} \times (0,T] \\
		\rho(\omega,x,t) = \rho_{\text{in}}(x,t) &\text{ für } (x,t) \in \Gamma_{\text{in}} \times \mathbb{T} \\
		\rho(\omega,x,0) = \rho_0(x) &\text{ für } x \in  \mathbb{D}
	\end{cases} \\
\text{ für die Anfangs- und Randwerte: } \\ 
	g_N: \Gamma_{\text{N}} \to \R \\
	u_D: \Gamma_{\text{D}} \to \R \\
	\rho_{\text{in}}: \Gamma_{\text{in}} \times \mathbb{T} \to \R_{\geq0} \\
	\rho_0: \mathbb{D} \to \R_{\geq0} \\
	\text{ wobei } \partial \mathbb{D} = \Gamma_{\text{D}} \dot{\cup} \Gamma_{\text{N}}  \text{ und }  \Gamma_{\text{in}} \coloneqq  \{ z \in \partial \mathbb{D}: q(z)\cdot n(z) \leq 0 \} \subset  \partial \mathbb{D}
\end{gather*}
Dabei stellen wir uns die Aufgabe den Erwartungswert eines gegebenen Zielfunktionals $Q(\rho)$ zu berechnen, etwa dem Ausfluss der transportierten Substanz über den Rand. An dieser Stelle können wir dann, nachdem wir uns in den nächsten zwei Unterabschnitten damit beschäftigt haben, wie wir obige Probleme numerisch lösen, die MLMC Methode nutzen, um diesen Erwartungswert zu berechnen. 
 
\subsection{Numerische Lösung des Potentialströmungsproblem}
\subsection{Numerische Lösung des Transportproblem}
\subsection{Anwendung der MLMC Methode auf das Transportproblem für unsichere Permeabilität $\kappa$ }
\section{Beispiel/Experiment}
\subsection{Konkretes Problem}
\subsection{Ergebnisse}
\section{Ausblick und Fazit}

  % Literaturverzeichnis (beginnt auf einer ungeraden Seite)
  \newpage

%\begin{thebibliography}{Lam00}
 %Bibiographie
  \bibliographystyle{abbrv}
  \bibliography{References}
%\end{thebibliography}
 
      
  % ggf. hier Tabelle mit Symbolen 
  % (kann auch auf das Inhaltsverzeichnis folgen)

\newpage
  
 \thispagestyle{empty}


\vspace*{8cm}


\section*{Erkl\"arung}

Ich  versichere  wahrheitsgem\"a\ss,  die  Arbeit selbstst\"andig verfasst,  alle  benutzten  Hilfsmittel  vollst\"andig  und  genau  angegeben  und  alles kenntlich  gemacht  zu  haben,  was  aus  Arbeiten  anderer  unver\"andert  oder  mit  Ab\"anderungen entnommen  wurde,  sowie die Satzung  des  KIT  zur  Sicherung guter wissenschaftlicher Praxis in der jeweils g\"ultigen Fassung beachtet zu haben.
\\[2ex] 

\noindent
Ort, den Datum\\[5ex]

% Unterschrift (handgeschrieben)



\end{document}


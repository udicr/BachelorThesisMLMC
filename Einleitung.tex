% !TeX root = bachelorarbeit.tex

\section{Einleitung}

Monte Carlo Methoden sind weit verbreitet, um in unterschiedlichsten Situationen Erwartungswerte stochastischer Modelle zu schätzen.
So werden bei der Monte Carlo Quadratur, der wohl bekanntesten Anwendungen der Monte Carlo Methode, zufällig gleichverteilte Stützstellen dafür genutzt, den Erwartungswert der Funktionsauswertung an einer zufälligen Stützstelle zu bestimmen. Daraus lässt sich anschließend der approximierte Integralwert auf einer gegebenen Grundmenge bestimmen. 
Wir wollen Monte Carlo Methoden dazu nutzen, eine stochastische partielle Differentialgleichung zu lösen. 
Genauer betrachten wir eine stochastisch beeinflusste Zielgröße $ Q $ der Lösung der Differentialgleichung und fragen uns, welchen Wert $ \mathbb{E}[Q] $ nimmt $ Q $ im Mittel an? Wir suchen also nach einem Erwartungswert, was die Verwendung von Monte Carlo Methoden zumindest einmal nahelegt. 
Außerdem wollen wir uns dabei die Vorteile der sogenannten Multilevel Monte Carlo Methode zunutze machen. Diese wurde zunächst von Heinrich für Approximation von parameterabhängigen Integralen in hohen Dimensionen entwickelt (vgl. \cite{heinrich2001multilevel}) und anschließend unter anderem von  Giles auf stochastisch modellierte partielle Differentialgleichungen übertragen. Folgendes Zitat bringt die Vorteile auf den Punkt, welche die Multilevel Monte Carlo Methode in diesem Zusammenhang bietet:
\begin{quote}
	\textit{Monte Carlo methods are a very general and useful approach for the estimation of expectations arising from stochastic simulation. However, they can be computationally expensive, particularly when the cost of generating individual stochastic samples is very high, as in the case of stochastic PDEs. \\ Multilevel Monte Carlo is a recently developed approach which greatly reduces the computational cost by performing most simulations with low accuracy at a correspondingly low cost, with relatively few simulations being performed at high accuracy and a high cost.}  \\
	\rightline{Michael B. Giles in \cite{giles_2015}. \, \qquad \qquad }
\end{quote}
Das Lösen, bzw. in unserem Fall das Bestimmen eines Erwartungswert, von stochastisch modellierten partiellen Differentialgleichungen fallen in das Gebiet der Uncertainty Quantification. Dieses noch recht "junge" Feld der Mathematik wird von Sullivan in \cite{sullivan2015introduction} als ein 'Zusammentreffen der Wahrscheinlichkeitstheorie, Numerik, Statistik und der echten Welt' beschrieben. \\
Die Thesis lässt sich aktuellen Arbeiten am Institut für Angewandte und Numerische Mathematik, wie etwa \cite{bibid}, zuordnen und soll daher zwar zum einen den theoretischen Hintergrund darlegen, aber auch einige Ergebnisse präsentieren, die im parallelen Finite Elemente System M++ \cite{siteM++} erzielt werden konnten, welches am Institut für Angewandte und Numerische Mathematik unter anderem von Herrn Prof. Dr. C. Wieners entwickelt wurde. \\
Konkret wollen wir vor allem die zeitabhängige lineare Transportgleichung betrachten. Dabei handelt es sich um ein weit verbreitetes Modellproblem, welches unter Anderem auch in \cite{di2011mathematical}, einem Standardwerk für Discontinuous Galerkin Methoden, zu finden ist. 
Dabei nutzen wir wie in \cite{di2011mathematical} die sogenannte 'method of lines': Zur numerischen Lösung der Differentialgleichung nutzen wir ein Discontinuous Galerkin Verfahren (DGV) im Ort und kombinieren diese zunächst semidiskrete Lösung mit einem geeigneten Zeitschrittverfahren. Aufgrund der Wahl eines Runge-Kutta Verfahrens erhalten wir somit ein Runge-Kutta Discontinuos Galerkin Verfahren. Einen Überblick über diese Verfahrensart findet man zum Beispiel auch in \cite{cockburn2001runge}.
Ebenfalls mit der Anwendung von Multilevel Monte Carlo Methoden auf parabolische Differentialgleichungen haben sich zum Beispiel auch Barth und Stein in \cite{barth2013multilevel} bzw. \cite{barth2019multilevel} oder Kumar et al. in \cite{kumar2018multigrid} beschäftigt. Mit Multilevel Monte Carlo Methoden im Allgemeinen haben sich neben den bereits erwähnten Giles und Barth auch Cliffe in \cite{cliffe2011multilevel}, Charrier in \cite{charrier2012strong} oder Teckentrup in \cite{teckentrup2013further}, um nur Einige zu nennen, befasst.
Um die eigentliche Idee der Multilevel Monte Carlo Methode in den Vordergrund zu rücken, wollen wir diese aber auch wie in \cite{heinrich2001multilevel} am Beispiel der numerischen Integration betrachten.
Dabei wollen wir insbesondere die Unterschiede zu und Vorteile gegenüber der Standard Monte Carlo Methode hervorheben, welche im Kern bereits durch obiges Zitat zusammengefasst sind.

%Monte Carlo Methoden sind weit verbreitet und finden in verschiedenen Bereichen der Mathematik ihre Anwendung.
%Sie dienen dabei als statistische Schätzer für Erwartungswerte. 
%Eine der bekanntesten Anwendungen ist wohl die Monte Carlo Quadratur, welche zur numerischen Integration genutzt werden kann.
% 
%Nachdem Giles (cite ...) ... gewöhnliche DGL ... kam ... für SPDE's zu nutzen ...cite .
%
%
%Allerdings besitzt die Monte Carlo Methode einen entscheidenden Nachteil, will man sie im Zusammenhang unsicherer Ausgangsdaten für die Lösung von partiellen Differentialgleichungen nutzen, sie konvergiert im Normalfall relativ langsam und das numerische Lösen von PDE's ist oft sehr aufwendig.
%Es werden also unter Umständen sehr viele, sehr teure Zufallssamples benötigt, um ein vernünftiges Ergebnis zu erhalten. \newline
%Diese Thesis soll sich daher mit der Multilevel Monte Carlo Methode (im Folgenden MLMC Methode genannt) beschäftigen, welche an die Monte Carlo Methode angelehnt ist, aber durch die geschickte Auswertung der (Zufalls-Samples) deutliche Effizienzvorteile gegenüber der Standard Monte Carlo Methode besitzt.
%Die MLMC Methode soll nach einer ausführlichen theoretischen Analyse auch praktisch auf das Transportproblem angewandt werden.
%Genauer soll für
%\begin{itemize}
%	\item ein beschränktes Gebiet $\mathbb{D} \subseteq \R^d$
%	\item  ein Zeitintervall $\mathbb{T} = [0,T]$
%	\item  ein Wahrscheinlichkeitsraum $(\Omega,\mathcal{A},\mathbb{P})$
%	\item  ein zufälliges Flussvektorfeld $q: \Omega \times \overline{\mathbb{D}} \rightarrow \R^d$
%	\item  eine Anfangskonzentration eines (zu transportierenden) Stoffes $\rho_0: \overline{\mathbb{D}} \rightarrow \R^d$
%	\item einen Einfluss $\rho_{\text{in}} : \Gamma_{\text{in}} \times \mathbb{T} \rightarrow \R$ über den Einflussrand $\Gamma_{\text{in}} \coloneqq  \{ z \in \partial \mathbb{D}: q(z)\cdot n(z) \leq 0 \} \subset  \partial \mathbb{D}$ mit $n(z)$ als äußeren Normalenvektor im (Rand-)Punkt $z$
%\end{itemize}
%der Erwartungswert eines Funktionals der  Konzentration des Stoffes $\rho: \overline{\mathbb{D}} \times \mathbb{T}  \rightarrow \R_{\geq0}$ bestimmt werden. Dabei erhält man $\rho$ als Lösung der folgenden partiellen Differentialgleichung:
%\begin{gather*}
%\text{Bestimme } \rho: \overline{\mathbb{D}} \times \mathbb{T} \to \R_{\geq 0} \text{, sodass}\\
%(\text{TP})
%\begin{cases}
%\partial_t \rho + \dive(\rho q) = 0 &\text{ in } \mathbb{D} \times (0,T)\\
%\rho(x,t) = \rho_{\text{in}}(x,t) &\text{ auf } \Gamma_{\text{in}} \times (0,T)\\
%\rho(x,0) = \rho_0(x) &\text{ auf } \mathbb{D}.
%\end{cases}
%\end{gather*}
%Außerdem muss zunächst ein zwar zufälliges, aber dennoch sinnvolles Vektorfeld $q$ erzeugt werden. Wir nutzen hierbei das Darcy-Gesetz, welches als Modellierung von Fluiden in porösen Bodenschichten bereits oft genutzt wurde (vgl. z.B. \cite{de1986quantitative}).
%Dabei soll später, bevor wir das eigentliche Transportproblem lösen, stets zunächst für einen zufälligen Permeabilitätstensor, welcher die unbekannte Bodenbeschaffenheit modellieren soll, ein entsprechendes Flussvektorfeld $q$ über das sogenannte Potentialströmungsproblem, welches sich aus dem Darcy-Gesetz ableitet, berechnet werden. 
%Die genauere Modellierung des so entstehenden Gesamtproblems soll aber an späterer Stelle erfolgen. \newline
Die Thesis ist dazu folgendermaßen unterteilt:\newline 
Abschnitt 2 sammelt verschiedene Grundlagen aus den Bereichen der Stochastik, der Analysis und der Numerik partieller Differentialgleichungen. Besonders werden wir hierbei auf einige zentrale Aussagen der Wahrscheinlichkeitstheorie eingehen, welche für die Konvergenzanalyse von Monte Carlo Methoden im Allgemeinen eine wichtige Rolle spielen. \newline
In Abschnitt 3 betrachten wir einige Aspekte der (standard) Monte Carlo Methode, welche auch der MLMC Methode als theoretischer Unterbau dienen sollen. Dabei erklären wir die Monte Carlo Methoden zunächst anhand des Beispiels der numerischen Integration, gehen dann aber auch abstrakter auf Konvergenz und Genauigkeit der Methode ein.\newline
Anschließend werden wir in Abschnitt 4 die Multilevel Monte Carlo Methode an sich erklären.
Dazu greifen wir das Beispiel der numerischen Integration aus Abschnitt 3 in einer etwas abgewandelten Form wieder auf. Auch hier wollen wir dann aber etwas abstrakter Eigenschaften der Methode betrachten, welche uns auch später bei der Anwendung auf das Transportproblem wieder beschäftigen werden. \newline
In Abschnitt 5 werden wir dann das Transportproblem einführen und dabei auch das Potentialströmungsproblem beschreiben, welches wir lösen müssen, um an die entsprechenden Ausgangsdaten zu kommen. Anschließend wird die numerische Lösung der beiden Probleme mit Finite Elemente Methoden behandelt, bevor schließlich in Abschnitt 6 auf die Anwendung der Multilevel Monte Carlo Methode auf das Transportproblem mit unsicheren Ausgangsdaten am Beispiel der Permeabilität $\kappa$ eingegangen wird. 
Der siebte und letzte Abschnitt befasst sich mit den konkreten Ergebnissen der Durchführung und Implementierung des zuvor theoretisch beleuchteten Problems innerhalb der parallelen Finite Elemente Softwarebibliothek "M++" \cite{siteM++}. Außerdem wird auf einige zusätzliche neue Features eingegangen, welche während des praktischen Teils der Thesis erarbeitet wurden. So wurde unter anderem Tools zum Lesen, Verarbeiten und Darstellung von '.vtk'-Dateien, welche in von M++ als Speicherformat genutzt werden, in der Programmiersprache Python implementiert. Neben dem automatisierten Erstellen von Schaubildern auf Grundlage dieser '.vtk'-Dateien, ist es uns außerdem damit möglich, auf Grundlage des durchgeführten Experiments auch eine konkrete MLMC-Lösung zu berechnen. Diese kombiniert die Lösungen, in unserem Fall die Konzentrations\-verläufe, der einzelnen Differentialgleichungen auf die gleiche Weise, wie die Erwartungswerte innerhalb der MLMC Methode verrechnet wurden. Wir erhalten so letztendlich nicht nur den gesuchten Erwartungswert, sondern auch eine auf die Berechnung dieses Erwartungswertes zugeschnittene erwartete Lösung der stochastischen Differentialgleichung. 
Der Thesis ist außerdem ein entsprechendes Notebook beigefügt, mit welchem die Ergebnisse reproduziert werden können.
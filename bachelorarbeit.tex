% Vorlage für eine Bachelorarbeit
% Siehe auch LaTeX-Kurs von Mathematik-Online
% www.mathematik-online.org/kurse
% Anpassungen für die Fakultät für Mathematik
% am KIT durch Klaus Spitzmüller und Roland Schnaubelt Dezember 2011

\documentclass[12pt,a4paper]{scrartcl}
% scrartcl ist eine abgeleitete Artikel-Klasse im Koma-Skript
% zur Kontrolle des Umbruchs Klassenoption draft verwenden


% die folgenden Packete erlauben den Gebrauch von Umlauten und ß
% in der Latex Datei
\usepackage[utf8]{inputenc}
% \usepackage[latin1]{inputenc} %  Alternativ unter Windows
\usepackage[T1]{fontenc}
\usepackage[ngerman]{babel}


\usepackage[pdftex]{graphicx}
\usepackage{latexsym}
\usepackage{amsmath,amssymb,amsthm}
\usepackage{setspace}
\usepackage{url}
\usepackage{mathtools}
\usepackage{enumitem}
\usepackage{float}
\usepackage{dsfont}


% Abstand obere Blattkante zur Kopfzeile ist 2.54cm - 15mm
\setlength{\topmargin}{-15mm}


% Umgebungen für Definitionen, Sätze, usw.
% Es werden Sätze, Definitionen etc innerhalb einer Section mit
% 1.1, 1.2 etc durchnummeriert, ebenso die Gleichungen mit (1.1), (1.2) ..
\theoremstyle{definition}
\newtheorem{Satz}{Satz}[section]
\newtheorem{Folgerung}[Satz]{Folgerung} 
\newtheorem{Definition}[Satz]{Definition} 
\newtheorem{Lemma}[Satz]{Lemma}	
\newtheorem{Beispiel}[Satz]{Beispiel}	   
                  
\numberwithin{equation}{section} 

\newtheorem*{Bemerkung}{Bemerkung}

% einige Abkuerzungen
\newcommand{\C}{\mathbb{C}} % komplexe
\newcommand{\K}{\mathbb{K}} % komplexe
\newcommand{\R}{\mathbb{R}} % reelle
\newcommand{\Q}{\mathbb{Q}} % rationale
\newcommand{\Z}{\mathbb{Z}} % ganze
\newcommand{\N}{\mathbb{N}} % natuerliche

% eigene Definitionen
\newcommand{\dx}{\, \mathrm{d} x}
\newcommand{\da}{\, \mathrm{d} a}
\newcommand{\dt}{\, \mathrm{d} t}
\newcommand{\ds}{\, \mathrm{d} s}
\newcommand{\du}{\, \mathrm{d} u}
\newcommand{\dlam}{\, \mathrm{d} \lambda}
\newcommand{\dP}{\, \mathrm{d} \mathbb{P}}

\newcommand{\icol}[1]{%inline row vector
		\left(	\begin{smallmatrix}#1\end{smallmatrix} \right) %
	}
\newcommand{\abs}[1]{| #1 |}
\newcommand{\ubar}[1]{\underline{#1}{}}
\DeclareMathOperator{\dive}{div}
\DeclareMathOperator{\spann}{span}
\DeclareMathOperator{\conv}{conv}
\DeclareMathOperator{\supp}{supp}

\begin{document}
  % Keine Seitenzahlen im Vorspann
  \pagestyle{empty}
  
  
  % Titelblatt der Arbeit
  \begin{titlepage}

    \includegraphics[scale=0.45]{kit-logo.jpg} 
    \vspace*{2cm} 

 \begin{center} \large 
    
    Bachelorarbeit
    \vspace*{2cm}

    {\huge Die Multilevel Monte Carlo Methode und deren Anwendung am Beispiel der linearen Transportgleichung}
    \vspace*{2.5cm}

    Tim Buchholz
    \vspace*{1.5cm}

    30.02.20
    \vspace*{4.5cm}


    Betreuung: Prof.Dr. Christian Wieners und M.Sc. Niklas Baumgarten \\[1cm]
    Fakultät für Mathematik \\[1cm]
		Karlsruher Institut für Technologie
  \end{center}
\end{titlepage}



  % Inhaltsverzeichnis
  \tableofcontents

\newpage
 


  % Ab sofort Seitenzahlen in der Kopfzeile anzeigen
  \pagestyle{headings}

\section{Einleitung}
% !TeX root = bachelorarbeit.tex

TODO(Einleitung wird zu einem späterem Zeitpunkt noch ausgebaut und nachgebessert mehr cites mehr forschung mehr inhalt)
Monte Carlo Methoden sind weit verbreitet und finden in verschiedenen Bereichen der Mathematik ihre Anwendung.
Sie dienen dabei als statistische Schätzer für Erwartungswerte. 
Eine der bekanntesten Anwendungen ist wohl die Monte Carlo Quadratur, welche zur numerischen Integration genutzt werden kann.
 
Nachdem Giles (cite ...) ... gewöhnliche DGL ... kam ... für SPDE's zu nutzen ...cite .

So entstehende Problemstellungen fallen in das Gebiet der Uncertainty Quantification, einem 'Zusammentreffen der Wahrscheinlichkeitstheorie, Numerik, Statistik und der echten Welt' \cite{sullivan2015introduction}.
Allerdings besitzt die Monte Carlo Methode einen entscheidenden Nachteil, will man sie im Zusammenhang unsicherer Ausgangsdaten für die Lösung von partiellen Differentialgleichungen nutzen, sie konvergiert im Normalfall relativ langsam und das numerische Lösen von PDE's ist oft sehr aufwendig.
Es werden also unter Umständen sehr viele, sehr teure Zufallssamples benötigt, um ein vernünftiges Ergebnis zu erhalten. \newline
Diese Thesis soll sich daher mit der Multilevel Monte Carlo Methode (im Folgenden MLMC Methode genannt) beschäftigen, welche an die Monte Carlo Methode angelehnt ist, aber durch die geschickte Auswertung der (Zufalls-Samples) deutliche Effizienzvorteile gegenüber der Standard Monte Carlo Methode besitzt.
Die MLMC Methode soll nach einer ausführlichen theoretischen Analyse auch praktisch auf das Transportproblem angewandt werden.
Genauer soll für
\begin{itemize}
	\item ein beschränktes Gebiet $\mathbb{D} \subseteq \R^d$
	\item  ein Zeitintervall $\mathbb{T} = [0,T]$
	\item  ein Wahrscheinlichkeitsraum $(\Omega,\mathcal{A},\mathbb{P})$
	\item  ein zufälliges Flussvektorfeld $q: \Omega \times \overline{\mathbb{D}} \rightarrow \R^d$
	\item  eine Anfangskonzentration eines (zu transportierenden) Stoffes $\rho_0: \overline{\mathbb{D}} \rightarrow \R^d$
	\item einen Einfluss $\rho_{\text{in}} : \Gamma_{\text{in}} \times \mathbb{T} \rightarrow \R$ über den Einflussrand $\Gamma_{\text{in}} \coloneqq  \{ z \in \partial \mathbb{D}: q(z)\cdot n(z) \leq 0 \} \subset  \partial \mathbb{D}$ mit $n(z)$ als äußeren Normalenvektor im (Rand-)Punkt $z$
\end{itemize}
der Erwartungswert eines Funktionals der  Konzentration des Stoffes $\rho: \overline{\mathbb{D}} \times \mathbb{T}  \rightarrow \R_{\geq0}$ bestimmt werden. Dabei erhält man $\rho$ als Lösung der folgenden partiellen Differentialgleichung:
\begin{gather*}
\text{Bestimme } \rho: \overline{\mathbb{D}} \times \mathbb{T} \to \R_{\geq 0} \text{, sodass}\\
(\text{TP})
\begin{cases}
\partial_t \rho + \dive(\rho q) = 0 &\text{ in } \mathbb{D} \times (0,T)\\
\rho(x,t) = \rho_{\text{in}}(x,t) &\text{ auf } \Gamma_{\text{in}} \times (0,T)\\
\rho(x,0) = \rho_0(x) &\text{ auf } \mathbb{D}.
\end{cases}
\end{gather*}
Außerdem muss zunächst ein zwar zufälliges, aber dennoch sinnvolles Vektorfeld $q$ erzeugt werden. Wir nutzen hierbei das Darcy-Gesetz, welches als Modellierung von Fluiden in porösen Bodenschichten bereits oft genutzt wurde (vgl. z.B. \cite{de1986quantitative}).
Dabei soll später, bevor wir das eigentliche Transportproblem lösen, stets zunächst für einen zufälligen Permeabilitätstensor, welcher die unbekannte Bodenbeschaffenheit modellieren soll, ein entsprechendes Flussvektorfeld $q$ über das sogenannte Potentialströmungsproblem, welches sich aus dem Darcy-Gesetz ableitet, berechnet werden. 
Die genauere Modellierung des so entstehenden Gesamtproblems soll aber an späterer Stelle erfolgen. \newline
Die Thesis ist dazu folgendermaßen unterteilt:\newline 
Abschnitt 2 sammelt verschiedene Grundlagen aus den Bereichen der Stochastik, der Analysis und Numerik partieller Differentialgleichungen. Besonders werden wir hierbei auf einige zentrale Aussagen der Wahrscheinlichkeitstheorie eingehen, welche für die Konvergenzanalyse von Monte Carlo Methoden im Allgemeinen eine wichtige Rolle spielen. 
In Abschnitt 3 betrachten wir einige Aspekte der (standard) Monte Carlo Methode, welche auch der MLMC Methode als theoretischer Unterbau dienen sollen. Dabei erklären wir die Monte Carlo Methoden zunächst anhand des Beispiels der numerischen Integration, gehen dann aber auch abstrakter auf Konvergenz und Genauigkeit der Methode ein.  \newline
Anschließend werden wir in Abschnitt 4 die Multilevel Monte Carlo Methode an sich erklären.
Dazu greifen wir das Beispiel der numerischen Integration aus Abschnitt 3 in einer etwas abgewandelten Form wieder auf. Auch hier wollen wir dann aber auch etwas abstrakter Eigenschaften der Methode betrachten, welche uns auch später bei der Anwendung auf das Transportproblem wieder beschäftigen werden. \newline
In Abschnitt 5 werden dann das Transportproblem und das Potentialströmungsproblem beschrieben, welches wir lösen müssen, um an die entsprechenden Ausgangsdaten zu kommen. Anschließend wird die numerische Lösung der beiden Probleme mit Finite Elemente Methoden behandelt, bevor schließlich in Abschnitt 6 auf die Anwendung der Multilevel Monte Carlo Methode auf das Transportproblem mit unsicheren Ausgangsdaten am Beispiel der Permeabilität $\kappa$ eingegangen wird. \newline
Der siebte und letzte Abschnitt befasst sich mitder  konkreten Durchführung und Implementierung des zuvor theoretisch beleuchteten Problem innerhalb der parallelen Finite Elemente Softwarebibliothek "M++" \cite{siteM++},
welche am Institut für Angewandte und Numerische Mathematik 3 (KIT) von Herrn Prof. Dr. C. Wieners entwickelt wurde. \newline
Am Schluss der Thesis steht eine kleine Zusammenfassung der bis dahin erarbeiteten Resultate und der Ausblick auf Möglichkeiten verschiedener Art an, diese weiter zu entwickeln.

%%%%%%%%%%%%%%%%%%%%%%%%%%%%%%%%%
 \newpage  % neuer Abschnitt auf neue Seite, kann auch entfallen
%%%%%%%%%%%%%%%%%%%%%%%%%%%%%%%%%
 
\section{Grundlagen}
\subsection{analytische/numerische Grundlagen}
% !TeX root = bachelorarbeit.tex
Sei $\mathbb{D} \subseteq \R^d$ für $d\in \N$
\begin{Definition}(Einige Operatoren)
	\begin{enumerate}
		\item Für $F: \R^d \to \R^d$ ist die \underline{Divergenz} von F definiert durch
			\begin{align*}
				\dive F = \nabla \cdot F \coloneqq \sum_{i=1}^{d} \frac{\partial F_i}{ \partial x_i}
			\end{align*}
		\item Für $f: \R^d \to \R$ und $\alpha = (\alpha_1,\dots,\alpha_d) \in \N^d$ ist die partielle Ableitung von $f$ nach dem Multiindex $\alpha$ definiert durch
			\begin{align*}
				D^{\alpha}f \coloneqq 
				\frac{\partial^{|\alpha|} f}{\partial x_1 ^{\alpha_1} \cdots  \partial x_d^{\alpha_d} } 
				=\frac{\partial^{\alpha_1+\dots +\alpha_d} f}{\partial x_1 ^{\alpha_1} \cdots  \partial x_d^{\alpha_d} } 
			\end{align*}
	\end{enumerate}
\end{Definition}
\begin{Definition}(Einige Funktionenräume)
	todo C,C1,Ccinf,Lp, L1loc
\end{Definition}
\begin{Definition}(schwache Ableitung)\\
	Sei $u \in L_{\text{loc}}^1$. Wir sagen v besitzt eine schwache Ableitung zum Multiindex $\alpha$, falls eine Funktion $v \in L_{\text{loc}}^1$ existiert, mit 
	\begin{align*}
		\int_{\mathbb{D}} u D^{\alpha} \Phi \dx = (-1)^{|\alpha|} \int_{\mathbb{D}} v \Phi \dx \qquad \forall \Phi \in C_0^{\infty}(\mathbb{D})
	\end{align*}
	In diesem Zusammenhang nennen wir $\Phi$ auch Testfunktion und wir definieren $D^{\alpha} u \coloneqq v$ als die schwache Ableitung von $u$ zum Multiindex $\alpha$.
\end{Definition}
\begin{Definition}(Sobolevräume)
	todo RWP 
\end{Definition}
\begin{Satz} (Satz von Gauß)
	todo RWP/FEM
\end{Satz}
\begin{Satz} (Rechnen mit Differentialoperatoren)
	todo FEM
\end{Satz}
%%%%%%%%%%%%%%%%%%%%%%%%%%%%%%%%%
\newpage  % neuer Abschnitt auf neue Seite, kann auch entfallen
%%%%%%%%%%%%%%%%%%%%%%%%%%%%%%%%%
\subsection{stochastische Grundlagen}
% !TeX root = bachelorarbeit.tex
An dieser Stelle wollen wir an einige grundlegende Resultate der Wahrscheinlichkeitstheorie erinnern. Außerdem führen wir dabei auch Teile der Notation ein, die wir an späterer Stelle noch brauchen werden. Als Referenzen sind vor allem \cite{brokate2016grundwissen} und die Vorlesung Wahrscheinlichkeitstheorie von Herrn Prof. Dr. Henze (SS18) 
%sowie die gleichnamige Vorlesung von Herrn Prof. Dr. Hug (SS19) 
zu nennen.\\ Sei $ \Omega \not = \emptyset $ eine beliebige nichtleere Teilmenge.
Einige grundlegende Begriffe der Maßtheorie wollen wir an dieser Stelle voraussetzen, sie sind aber ebenfalls in \cite{brokate2016grundwissen} und zum Teil in \cite{lapeyre2003introduction} zu finden. Dazu zählen:
\begin{itemize}
	\item eine $ \sigma $-Algebra $ \mathcal{A} \subset \mathcal{P}(\Omega) $
	\item die von einem Mengensystem $ \mathcal{M} \subset \mathcal{P}(\Omega)  $ erzeugte $ \sigma $-Algebra $ \sigma(\mathcal{M}) $
	\item ein Maß $ \mu $ auf einer $ \sigma $-Algebra $ \mathcal{A} $
	\item das Maß-Integral einer messbaren Funktion $ f:\Omega \to \overline{\R} $ über einem Maßraum $ (\Omega, \mathcal{A},\mu) $
\end{itemize}

\begin{Definition}(Wahrscheinlichkeitsraum)\\
	\label{Wraum}
	Ein Wahrscheinlichkeitsraum ist ein Tripel $ (\Omega,\mathcal{A},\mathbb{P}) $. Dabei seien:
	\begin{enumerate}[label=(\alph*)]
		\item $ \Omega $ eine beliebige nichtleere Teilmenge
		\item $ \mathcal{A} $ eine $ \sigma $-Algebra über $ \Omega $
		\item $ \mathbb{P}:\mathcal{A} \to \R $ eine Funktion mit:
		\begin{enumerate}[label=(\roman*)]
			\item $ \mathbb{P}(A) \geq 0 $ für jedes $ A \in \mathcal{A} $
			\item $ \mathbb{P}(\Omega) = 1 $
			\item Sind $ A_1,A_2,\dots $ $\in  \mathcal{A} $ paarweise disjunkt , dann gilt $ \mathbb{P}(\sum\limits_{j=1}^{\infty}A_j) = \sum\limits_{j=1}^{\infty} \mathbb{P}(A_j)$
		\end{enumerate}
	\end{enumerate}
Insbesondere erfüllt $ \mathbb{P} $ auch die Bedingungen eines Maßes. Somit sind Wahrscheinlichkeitsräume Spezialfälle eines Maßraumes.
Jede Menge $ A \in \mathcal{A} $ heißt dann auch Ereignis, zu $ \mathbb{P} $ sagen wir Wahrscheinlichkeitsmaß und wir nennen $ \mathbb{P}(A) $ die Wahrscheinlichkeit des Ereignisses $ A $.  Ein Tupel $ (\Omega,\mathcal{A}) $ heißt Messraum oder auch messbarer Raum. 
\end{Definition}
\begin{Definition}(Zufallsvariable und deren Verteilung.)
	\begin{enumerate}[label=(\alph*)]
		\item Seien $ (\Omega,\mathcal{A}) $ und $ (\Omega',\mathcal{A}')  $ Messräume.
		Eine ($ \Omega' $-wertige) Zufallsvariable ist ein eine $ (\mathcal{A},\mathcal{A}') $-messbare Funktion $ X:\Omega \to \Omega' $, d.h. es gilt: $ X^{-1}(A') \in \mathcal{A} \quad \forall A' \in \mathcal{A}' $.\\
		Der Wert $ X(\omega) $ heißt auch Realisierung der Zufallsvariablen $ X $ zum Ausgang $ \omega\in\Omega $
		\item Sei in obiger Situation zusätzlich $ (\Omega,\mathcal{A}) $ ausgerüstet mit Wahrscheinlichkeitsmaß $ \mathbb{P} $, also ein Wahrscheinlichkeitsraum. Dann ist durch 
		\begin{align*}
			\mathbb{P}^X : \mathcal{A}' &\to [0,1] \\
			A' &\mapsto \mathbb{P}(X^{-1}(A')), \quad A'\in \mathcal{A}'
		\end{align*}
		ein Maß auf $ \mathcal{A}' $ definiert. $ \mathbb{P}^X $ heißt dann die Verteilung von $ X $.
	\end{enumerate}
\end{Definition}
$ \newline $
Sei ab nun $ (\Omega,\mathcal{A},\mathbb{P}) $ stets ein Wahrscheinlichkeitsraum.

\begin{Definition}(Unabhängigkeit)\\
	Sei $ \mathcal{J} $ eine Menge mit mindestens zwei Elementen.
	\begin{enumerate}[label=(\alph*)]
		\item Es seien $ A_j \in \mathcal{A}$ für $ j \in \mathcal{J} $ Ereignisse. Die Familie $ (A_j)_{j \in \mathcal{J}} $ heißt unabhängig, falls gilt:
		\begin{align}
			\label{unabh1}
			\mathbb{P}\left( \bigcap_{j \in J} A_j \right) = \prod_{j \in J} \mathbb{P}(A_j) \; \forall J \subset\mathcal{J} \text{ mit } 2\leq|J|\leq\infty
		\end{align}
		\item Seien $ \mathcal{M}_j \subset \mathcal{A} $ für $ j \in \mathcal{J} $ Mengensysteme. Die Familie $ (\mathcal{M}_j)_{j \in \mathcal{J}} $ von Mengensystemen heißt unabhängig, falls Bedingung (\ref{unabh1}) für jede endliche mindestens zweielementige Teilmenge $ J \subset \mathcal{J} $ und jede Wahl $ A_j \in \mathcal{M}_j, \; j \in J $ erfüllt ist.
		\item Seien $ (\Omega_j,\mathcal{A}_j)_{j \in \mathcal{J}} $ messbare Räume und $ X_j : \Omega \to \Omega_j $ für $ j \in \mathcal{J} $ Zufallsvariablen.
		Die Familie $ (X_j)_{j \in \mathcal{J}} $ heißt unabhängig, falls die Familie der erzeugten $ \sigma $-Algebren \[ (\sigma(X_j))_{j \in \mathcal{J}} \coloneqq \sigma \left(\bigcup_{j \in \mathcal{J}} X_j^{-1}(\mathcal{A}_j) \right) \] unabhängig ist.
	\end{enumerate}	
\end{Definition} 
\begin{Definition}(Erwartungswert)\\
	$ X:\Omega \to \overline{\R} $ sei eine Zufallsvariable.
	Der Erwartungswert von $ X $ existiert genau dann, wenn $ \int_{\Omega} |X| \dP < \infty $. In diesem Fall heißt
	\begin{align*}
	\mathbb{E}[X] \coloneqq \int_{\Omega} X \dP 
	\end{align*}
	der Erwartungswert von X
\end{Definition}
\begin{Satz}(schwaches Gesetz großer Zahlen) \\
	Sei $ (X_n)_{n \in \N } $ eine Folge unabhängiger reellwertiger Zufallsvariablen auf $ (\Omega,\mathcal{A},\mathbb{P}) $ mit identischer Verteilung $ \mathbb{P}^{X_1} = \mathbb{P}^{X_n} \; \forall n \in \N$. Wir nennen $ (X_n) $ dann auch eine u.i.v-Folge, dabei steht u.i.v. für 'unabhängig identisch verteilt'.
	Ist zudem $ \mathbb{E}[X_1^2] < \infty $ so gilt:
	\[
		\frac{1}{n} \sum_{j=1}^n X_j \stackrel{\mathbb{P}}{\to} \mathbb{E}[X_1]
	\]
	Mit $ \stackrel{\mathbb{P}}{\to} $ bezeichnen wir dabei die Konvergenz bezüglich des Wahrscheinlichkeitsmaßes $ \mathbb{P} $. Es gilt also 
	$ \lim\limits_{n \to \infty} \mathbb{P}(|\frac{1}{n} \sum\limits_{j=1}^n X_j - \mathbb{E}[X_1]|>\epsilon) = 0 $. Wir sagen auch $ \frac{1}{n} \sum\limits_{j=1}^n X_j $ konvergiert stochastisch gegen $ \mathbb{E}[X_1] $. Im Falle eines Zufallsvektors (einer $ \R^d $-wertigen Zufallsvariable) kann der Betrag gegen eine beliebige Norm auf $ \R^d $ ersetzt werden.
\end{Satz}

\begin{Satz}(starkes Gesetz großer Zahlen)\\
	Es sei $ (X_n)_{n \in \N} $ eine u.i.v.-Folge und es gelte $ \mathbb{E}[|X_1|] < \infty$. Dann gilt für fast alle $ \omega  \in \Omega$ 
	\[
		\mathbb{E}[X] = \lim_{n \to \infty} \sum_{i=1}^n X_i(\omega)
	\]
	das heißt es existiert eine Menge $ N \subset \Omega $ mit $ \mathbb{P}(N)=0 $ und obige Aussage gilt für alle $ \omega \not \in N $. Eine solche Menge $ N $ heißt auch Nullmenge. In der Literatur findet man diese Art der Konvergenz oft auch unter dem Namen der ($\mathbb{P}$)-fast sicheren Konvergenz.
\end{Satz} 
\begin{Bemerkung}
	Aus Konvergenz für fast alle $ \omega \in \Omega $ folgt insbesondere Konvergenz bezüglich des Wahrscheinlichkeitsmaßes $ \mathbb{P} $.\\
	 Denn falls $ \mathbb{E}[X] = \lim\limits_{n \to \infty} \sum_{i=1}^n X_i(\omega) $ für fast alle $ \omega \in \Omega $, dann ist das gleichbedeutend mit $ \mathbb{P}(\{ \omega \in \Omega:  \lim\limits_{n \to \infty} \sum_{i=1}^n X_i(\omega) \not = \mathbb{E}[X] \}) = 0 $ und somit insbesondere $ \frac{1}{n} \sum_{j=1}^n X_j \stackrel{\mathbb{P}}{\to} \mathbb{E}[X_1] $.
\end{Bemerkung}
\begin{Satz}(Zentraler Grenzwertsatz)\\
	Sei $ (X_n)_{n \in \N} $ eine u.i.v.-Folge und es gelte $ \mathbb{E}[X_1^2] < \infty $.
	Mit $ \mathbb{V}[X_1] $ bezeichnen wir die Varianz der Zufallsvariable $ X_1 $:
	\[
		\mathbb{V}[X_1] = \mathbb{E}[X_1^2]-\mathbb{E}[X_1]^2 = \mathbb{E}[(X_1-\mathbb{E}[X_1])^2]
	\]
	Dann gilt für eine standardnormalverteilte Zufallsvariable N:
	\[
		\hat{S}_n \coloneqq\frac{\sum_{j=1}^{n}X_j-n\mathbb{E}[X_1]}{\sqrt{nV[X_1]}} \stackrel{\mathcal{D}}{\to} N
	\]
	Dabei bezeichnet $ \stackrel{\mathcal{D}}{\to} $ die Konvergenz in Verteilung und ist genau dann erfüllt, wenn 
	\[ \lim\limits_{n \to \infty} \mathbb{P}^{\hat{S}_n}((-\infty,x]) = \mathbb{P}^N((-\infty,x])
	\]
	für alle Stetigkeitsstellen der Verteilungsfunktion $ \mathbb{P}^N((-\infty,\cdot]) $ von N erfüllt ist.
\end{Satz}
\begin{Definition}(Zufallsfelder) \\
	
\end{Definition}


%%%%%%%%%%%%%%%%%%%%%%%%%%%%%%%%%
\newpage  % neuer Abschnitt auf neue Seite, kann auch entfallen
%%%%%%%%%%%%%%%%%%%%%%%%%%%%%%%%%
\section{Die Monte Carlo Methode}
% !TeX root = bachelorarbeit.tex
%\cite{lapeyre2003introduction} \cite{sullivan2015introduction}
\subsection{Herleitung und Beispiel}
Wie in \cite{lapeyre2003introduction} wollen wir uns, um die Monte Carlo Methode von Grund auf einzuführen, zunächst mit der numerischen Integration beschäftigen. Grundsätzlich handelt es sich bei der Monte Carlo Methode um einen sogenannten Erwartungswertschätzer. Bevor wir also ein Problem mithilfe der Monte Carlo Methode lösen können, müssen wir die Größe, welche wir berechnen wollen, zunächst in der Form eines Erwartungswertes ausdrücken.
Wir suchen dann also einen Erwartungswert $ \mathbb{E}[X] $, wobei $ X $ eine Zufallsvariable, einen Zufallsvektor oder gar ein Zufallsfeld beschreiben kann.
Mithilfe der Monte-Carlo-Methode können wir dann versuchen eben diesen Erwartungswert zu schätzen. Dazu müssen wir $ X $ simulieren können. Damit ist gemeint, dass wir in der Lage sein müssen eine Realisierung $ (x_1,\dots,x_n) $ von $ (X_1,\dots,X_n) $ zu generieren (oft sagt man auch in Anlehnung an das Bernoulli'sche Urnenmodell 'zu ziehen'). Dabei sollen die Zufallsgrößen $ X_1,\dots,X_n $ unabhängig sein und die gleiche Verteilung besitzen wie die Zufallsgröße $ X $. Außerdem sei vorausgesetzt, dass der Erwartungswert $ \mathbb{E}[X] < \infty $ existiert.
Anschließend wird der gesuchte Erwartungswert durch
\[
	\mathbb{E}[X] \approx \frac{1}{n}(x_1 + \dots x_n)
\]
approximiert.

\begin{Beispiel}(Integral über $ [0,1]^d $ - aus \cite{lapeyre2003introduction})\\
	\label{BeispielIntegral}
	Angenommen, wir wollen für $ d \geq 1 $ folgendes Integral berechnen:
	\[
		I = \int_{[0,1]^d} f(u_1,\dots,u_d) \du_1\dots\du_d
	\]
	Wir können das Integral dann wie folgt als Erwartungswert ausdrücken: \\
	Sei $ X = f(U_1,\dots,U_d) $ ein 
	Zufallsvektor, wobei $ U_1,\dots,U_d $ unabhängig und auf $ [0,1] $ gleichverteilt sind, d.h. jedes $ U_i $ besitzt als Dichte $ f_i(x) = \mathds{1}_{[0,1]}(x) $.
	Dann ergibt sich so 
	\[
		I = \int_{[0,1]^d} f(u_1,\dots,u_d) \du_1\dots\du_d = \mathbb{E}[f(U_1,\dots,U_d)] = \mathbb{E}[X]
	\]
	Wir haben also das Integral, welches wir berechnen wollen, als Erwartungswert ausgedrückt. Nun müssen wir die Zufallsvariable $ X = f(U_1,\dots,U_d) $ simulieren.
	Dazu nehmen wir an, gleichverteilte Zufallsvariablen simulieren zu können. Die Simulation solcher Zufallsvariablen spielt in der numerischen Stochastik eine ganz besondere Rolle, denn oft werden andere Verteilungen durch Transformationen auf den Fall einer Gleichverteilung auf $ [0,1] $ reduziert.
	Sei also $ (U_i)_{i \geq 1} $ eine Folge unabhängiger Zufallsvariablen mit Gleichverteilung auf $ [0,1] $. Wir können dann mithilfe der simulierten Realisierungen $ (u_i)_{i \geq 1} $  von $ (U_i)_{i \geq 1} $ die Zufallsvariable $ X $ wie folgt definieren: Wir setzen
	\begin{align*}
		&X_1 = f(U_1,\dots,U_d), & &x_1 = f(u_1,\dots,u_d) \\
		&X_2 = f(U_{d+1},\dots,U_{2d}), & &x_2 = f(u_{d+1},\dots,u_{2d}) \\
		&X_i = f(U_{(i-1)d+1},\dots,U_{id}), & &x_2 = f(u_{(i-1)d+1},\dots,u_{id})
	\end{align*}
	Da $ (U_i)_{i \geq 1} $ eine Folge unabhängiger Zufallsvariablen ist, erhalten wir so unter der einzigen echten Voraussetzung, dass $ f $ messbar ist, nach dem Blockungslemma ebenfalls eine Folge unabhängiger Zufallsvariablen $ (X_i)_{i \geq 1} $.
	Außerdem erhalten wir so für ein großes $ n \in \N $ eine gute Approximation von $ I $ durch:
	\[
		I = \mathbb{E}[X] \approx \frac{1}{n}(x_1+\dots+x_n) = \frac{1}{n} (f(u_1,\dots,u_d)+\dots+f(u_{(n-1)d+1},\dots,u_{nd}))
	\]	
	Inbesondere haben wir keinerlei Regularität an $ f $ vorausgesetzt, es genügt bereits die bloße Messbarkeit von $ f $ .
\end{Beispiel}



 	Oft wollen wir über eine andere Grundmenge als $ [0,1]^d $ integrieren. 
 	Bei endlichen Mengen, etwa einer beschränkten Borelmenge $ B \subset \R^d $ mit $ 0 < \abs{B} \coloneqq \lambda^d(B) $ (hierbei ist $ \lambda^d(\cdot) $ das Borel-Lebesgue-Maß) lässt sich $ I = \int_{B} f(x) \dx $ ähnlich wie in obigem Beispiel berechnen.
	Für einen Zufallsvektor $ U $ mit Gleichverteilung $ U(B) $ auf $ B $ existiert nämlich der Erwartungswert $ f(U) $ und es gilt:
	\[
		\mathbb{E}[f(U)] = \int_{B} f(x) \frac{1}{\abs{B}} \dx = \frac{I}{\abs{B}}
	\]
	Wieder simulieren wir $ (U_i)_{i \geq 1} $ als Folge unabhängiger Zufallsvariablen mit identischer Verteilung zu $ U $. Dann erhalten wir:
	\[
		I = \abs{B} \cdot \mathbb{E}[f(U)] \approx \frac{\abs{B}}{n} \sum_{j=1}^{n}f(u_j)
	\]
	 Wollen wir hingegen ein Integral über $ \R^d $ auswerten, muss es uns in der Form 
	\[
	I = \int_{\R^d} g(x)f(x) \dx = \int_{\R^d} g(x_1,\dots,x_d)f(x_1,\dots,x_d) \dx
	\] 
	vorliegen. Dabei sei $ f(x) $ nichtnegativ und $ \int_{\R^d} f(x) \dx = 1 $.
	Dann lässt sich $ I $ schreiben als $ I = \mathbb{E}[g(X)] $ für eine Zufallsvariable $ X $ mit Werten in $ \R^d $ und Verteilung $ f(x) \dx $.
	Wir können also $ I $ approximieren durch
	\[
		I \approx \frac{1}{n}\sum_{i=1}^{n} g(x_i) \quad ,
	\]
	wobei $ (x_i)_{i \geq 1} $ Realisierungen der Zufallsvariablen $ (X_i)_{i \geq 1} $ sind, welche unabhängig und identisch zu $ X $ verteilt seien.
	
	Betrachten wir nun wieder die Monte Carlo Methode in einem etwas abstrakteren Sinne ganz allgemein.
	An der Stelle, an der wir letztlich die Realisierungen einer Zufallsvariable eingesetzt haben, also einen Erwartungswert durch $
	\mathbb{E}[X] \approx \frac{1}{n}(x_1 + \dots x_n)
	$ approximiert haben, haben wir stets gefordert, dass $ n $ groß ist. 
	Es stellt sich nun die Frage, wann $ n $ groß genug ist.
	Wir wollen uns deshalb noch abschließend damit beschäftigen, wann und wie die Methode konvergiert und was wir über die Genauigkeit der Approximation aussagen können.
	\subsection{Konvergenz und Genauigkeit}
	Damit die Methode überhaupt in irgendeiner Weise als nützlich zu erachten ist, bedarf es Möglichkeiten, den Fehler \[
	 \epsilon_n = \frac{1}{n}\sum_{i=1}^{n}X_i -  \mathbb{E}[X]
	\]
	abzuschätzen. Um diesem Problem beizukommen, bedienen wir uns zweier zentraler Aussagen der Wahrscheinlichkeitstheorie. Zum einen sagt uns das starke Gesetz großer Zahlen \ref{starkesGgZ}, dass unter der Voraussetzung $ \mathbb{E}[\abs{X}]<\infty $ der Fehler $ \epsilon_n $ für $ n \to \infty $ für fast alle $ \omega \in \Omega $ gegen $ 0 $ konvergiert. Wir erhalten also zunächst Konvergenz der Methode in einem sehr grundlegenden Sinn. Aus dem zentralen Grenzwertsatz \ref{ZGWS} lassen sich zum anderen Aussagen über die Genauigkeit der Methode und letztlich somit auch der Art der Konvergenz ableiten. Nach \ref{ZGWS} erhalten wir nämlich für eine u.i.v.-Folge $ (X_i)_{i \in \N} $ mit gleicher Verteilung wie $ X $ und $ \mathbb{E}[X^2] < \infty $, dass
	 \[ 
	 \frac{\sqrt{n}}{\sqrt{\mathbb{V}[X]}} \epsilon_n =  \frac{\frac{1}{\sqrt{n}}\sum_{i=1}^{n}X_i-\sqrt{n}\mathbb{E}[X]}{\sqrt{\mathbb{V}[X]}} = \frac{\sum_{i=1}^{n}X_i-n\mathbb{E}[X]}{\sqrt{n\mathbb{V}[X]}} \eqqcolon \hat{S}_n \stackrel{\mathcal{D}}{\to}  \widetilde{\mathcal{N}} \text{ für } n \to \infty \ ,
	 \]
	 wobei  $\widetilde{\mathcal{N}}$ eine standardnormalverteilte Zufallsvariable ist. Die Wurzel der Varianz wird im Folgenden noch des Öfteren auftauchen, weswegen wir an dieser Stelle die sogenannte Standardabweichung $ \sigma \coloneqq \sqrt{\mathbb{V}[X]} $ einführen.
	 Da also 
	 \[ \lim\limits_{n \to \infty} \mathbb{P}^{\hat{S}_n}((-\infty,x]) = \mathbb{P}^{ \widetilde{\mathcal{N}}}((-\infty,x])
	 \]
	 gilt, ist insbesondere für $ a \leq b $
	 \[
	 	\lim\limits_{n \to \infty} \mathbb{P}(\frac{\sigma}{\sqrt{n}}a \leq \epsilon_n \leq \frac{\sigma}{\sqrt{n}}b) = \lim\limits_{n \to \infty} \mathbb{P}(a \leq \hat{S}_n \leq b) = \int\limits_a^b \frac{1}{\sqrt{2\pi}}e^{-\frac{x^2}{2}} \dx \ .
	 \]
	 An dieser Stellen wollen wir kurz innehalten und uns überlegen, was obiges Resultat für den Fehler der Monte Carlo Methode denn praktisch gesehen bedeutet. 
	 \begin{itemize}
	 	\item Der zentrale Grenzwertsatz liefert uns kein zu der Folgerung aus dem starken Gesetz großer Zahlen vergleichbares Resultat, denn es ist  $ \lim_{n \to \infty} \mathbb{P}(\epsilon_n = 0) = 0 $ nach obiger Überlegung.
	 	\item Der zentrale Grenzwertsatz erlaubt uns ebenso \underline{nicht} eine für andere Verfahren typische Fehlerschranke der Form $ \epsilon_n \leq M_n $ für eine von n und möglicherweise anderen Faktoren, wie z.B. Ausgangsdaten, abhängigen Schranke $ M_n $ aufzustellen. Grund dafür ist, dass der Träger von $\frac{1}{\sqrt{2\pi}}e^{-\frac{x^2}{2}}$ ganz $ \R $ ist.
	 	\item Was der zentrale Grenzwertsatz uns jedoch erlaubt, ist, ein sogenanntes $ 95\% $ Konfidenzintervall für $ \epsilon_n $ zu bestimmen. Das bedeutet, dass das tatsächliche Ergebnis mit einer Wahrscheinlichkeit von mindestens $ 95 \% $ im gegebenen Intervall enthalten ist. Denn, da 
	 	\[
	 		\mathbb{P}(\abs{N} \leq 1.96) \approx 0.95 \quad ,
	 	\]
	 	können wir wegen
	 	\[
	 		\lim\limits_{n\to\infty}\mathbb{P}(-1.96\frac{\sigma}{\sqrt{n}}\leq \epsilon_n \leq 1.96\frac{\sigma}{\sqrt{n}}) \approx 0.95 \quad (\star)
	 	\]
	 	ein Konfidenzintervall für $ \mathbb{E}[X] $ der Form
	 	\[
	 		[\hat{\mu}-1.96\frac{\sigma}{\sqrt{n}},\hat{\mu}+1.96\frac{\sigma}{\sqrt{n}}] \quad \text{ für ein } \hat{\mu} \in \R  
	 	\]
	 	angeben. In der Praxis nehmen wir näherungsweise an, dass $ (\star) $ auch für ein festes $ n \in \N $ erfüllt ist, und entledigen uns so des Grenzwertes. Somit wird dann insbesondere die Wahl $ \hat{\mu} = \frac{1}{n}\sum_{i=1}^{n}(x_1,\dots,x_n) $ gerechtfertigt.
	 \end{itemize}
 	Wir erhalten also (unter den eben erklärten Annahmen) eine Konvergenzrate des (wahrscheinlichen) Fehlers von $ \frac{\sigma}{\sqrt{n}} $. Dieses Resultat mag auf den ersten Blick relativ ernüchternd wirken, allerdings existieren Fälle, in denen solch eine langsame Methode die bestmögliche ist. \cite{lapeyre2003introduction} nennt hierzu zum Beispiel Integrale in mehr als $ 100 $ Dimensionen oder besonders schwere parabolische Differentialgleichungen. Denn anders als andere Verfahren, besonders deutlich wird dies erneut auf der Ebene der Quadratur (vgl Beispiel \ref{mlmcbeispiel}), sind Monte Carlo Methoden nicht vom sogenannten 'Curse of dimensionality' betroffen. Während bei anderen Quadraturformeln die Anzahl der benötigten Quadraturpunkte mit der Dimension im Exponent steigt, gelten obige Resultate unabhängig von der Dimension. Wir werden später Zufallsprobleme mit sehr hohen Dimensionen betrachten, da wir in einer Bodenschicht jede Zelle als einzelne Zufallsvariable betrachten werden. Deswegen ziehen wir die Monte Carlo Methode bzw. später die Multilevel Monte Carlo Methode einem anderen Ansatz zum Lösen stochastischer partieller Differentialgleichungen, wie etwa stochastische Finite Elemente, vor.
 	Außerdem lohnt es sich zu erwähnen, dass wir im Falle der numerischen Integration - bis auf Integrierbarkeit und Messbarkeit - keine Voraussetzungen an die Regularität der Funktion $ f $ gestellt haben.\\
 	Obiges Resultat legt außerdem nahe, dass es entscheidend für eine Aussage über die Konvergenz und Güte der Methode ist, die Standardabweichung $ \sigma $ zu kennen, oder zumindest über einen guten Schätzer für $ \sigma $ zu verfügen.
 	Falls uns $ \sigma $ bzw. $ \mathbb{V} $ nämlich sogar exakt bekannt ist, können wir die sogenannte Chebyshev Ungleichung \ref{ChebCheb} ausnutzen:
 	Da $ (X_i)_{i \in \N} $ eine u.i.v.-Folge mit Verteilung wie $ X $ ist, gilt nämlich mit den üblichen Rechenregeln für die Varianz (zu finden z.B. in \cite{brokate2016grundwissen} auf den Seiten 778 und 779)
 	\[
 		\mathbb{V}[\frac{1}{n}\sum_{i=1}^{n}X_i] =  \frac{1}{n^2} \sum_{i=1}^{n} \mathbb{V}[X] = \frac{\mathbb{V}[X]}{n}
 	\]
 	Dann besagt die Chebychev Ungleichung für alle $ t \geq 0 $:
 	\[
 		\mathbb{P}\left(\abs{\frac{1}{n}\sum_{i=1}^{n}X_i-\mathbb{E}[X]} \geq t \right) \leq \frac{\mathbb{V}[X]}{nt^2}
 	\]
 	Für uns bedeutet das insbesondere, dass für jedes $ \epsilon \in (0,1] $  die berechnete Monte-Carlo Approximation $ \frac{1}{n}\sum_{i=1}^{N} $ mit einer Wahrscheinlichkeit von $ 1-\epsilon $ weniger als $ \left( \frac{\mathbb{V}[X]}{n\epsilon}\right)^{\frac{1}{2}} $ von dem tatsächlichen Erwartungswert $ \mathbb{E}[X] $ entfernt ist.
 	In der Literatur (z.B. in \cite{sullivan2015introduction}) finden sich einige Weiterentwicklungen der Monte Carlo Methode. Abgesehen von der Multilevel Monte Carlo Methode, welche wir in Abschnitt \ref{MLMC} behandeln werden, wollen wir uns hier auf die oben erklärte Standard-Variante beschränken.
 	
 
 	
	
	
	



%%%%%%%%%%%%%%%%%%%%%%%%%%%%%%%%%
\newpage  % neuer Abschnitt auf neue Seite, kann auch entfallen
%%%%%%%%%%%%%%%%%%%%%%%%%%%%%%%%%

\section{Das lineare Transportproblem}
\label{TP}
\subsection{Problemstellung}
 % !TeX root = bachelorarbeit.tex
\subsubsection{deterministisches Problem}
\label{det_prob}
Sei $\mathbb{T} = [0,T]$ ein Zeitintervall für $T>0$ und $\mathbb{D} \subset \R^d , d \in \N$ ein beschränktes, offenes und konvexes Lipschitz-Gebiet mit Rand $ \partial \mathbb{D} = \Gamma_{\text{D}}  \dot{\cup} \Gamma_{\text{N}} $. 
Wie bereits in der Einleitung beschrieben, wollen wir den Transport eines Stoffes in einer porösen Bodenschicht auf Grundlage eines vorhandenen Flusses beschreiben. 
Als modellhaftes Problem soll uns hierfür die Regenwasserversickerung dienen: In einer porösen Bodenschicht befindet sich zum Zeitpunkt $t=0$ ein Stoff (beispielsweise Öl) in einer gegebenen Anfangskonzentration und -verteilung. Nun sickert Regenwasser in diese poröse Bodenschicht ein. Zusätzlich wollen wir weitere Zuflüsse des Fremdstoffes über den Einflussrand $\Gamma_{\text{in}} \subset \partial \mathbb{D}$ zulassen.
Wir sind letztendlich an der Konzentration dieser Substanz an einer Stelle $x \in \overline{\mathbb{D}}$ zu einem Zeitpunkt $t \in \mathbb{T}$ interessiert. \newline
Bevor allerdings die Konzentration als Lösung des Transportproblems bestimmt werden kann, muss zunächst das Flussvektorfeld $q: \overline{\mathbb{D}} \rightarrow \R^d$ berechnet werden. \newline
Sei hierfür $p: D \rightarrow \R$ der hydrostatische Druck, $\kappa: D \rightarrow (\R_{\text{sym}})^{d \times d}$ der Permeabilitätstensor und $G=(0,0,p_0 g_0)^{\top}$. 
Wie bereits in der Einleitung angedeutet, kann der Fluss des Regenwassers durch das Darcy-Gesetz $q=-\kappa(\nabla p + G)$ modelliert werden.
Durch $u(x) := p(x) + p_0 g_0 x_3$ vereinfacht sich das Darcy-Gesetz zu $q=-\kappa \nabla u$.\newline
Nehmen wir die physikalische Annahme hinzu, dass der Fluss $q$ 'quellfrei' sein soll, also an keiner Stelle Masse verschwinden oder erscheinen kann, erhalten wir das Potentialströmungsproblem:
\[ \text{Bestimme } u:\overline{\mathbb{D}} \to \R \text{ und } q: \overline{\mathbb{D}} \to \R^2 \text{ mit } \newline \]
\[\setlength\arraycolsep{1pt}
\text{(PS)}\begin{cases} 
\begin{array}{rlc}
\dive q     &= 0                 &\text{ ,} \text{in } \mathbb{D}\\
q           &= - \kappa \nabla u &\text{ ,} \text{in }\mathbb{D}\\
u           &= u_D               &\text{ ,} \text{auf } \Gamma_D \\
-q \cdot n  &= g_N               &\text{ ,} \text{auf } \Gamma_N 
\end{array}
\end{cases} \\
\]
\begin{Bemerkung}
	Wir wollen aus verschiedenen Gründen direkt die sogenannte gemischte Formulierung des Potentialströmungsproblem nutzen. Näheres dazu findet sich im nächsten Abschnitt.
\end{Bemerkung}
Anschließend suchen wir die Dichteverteilung $\rho: \mathbb{D} \times \mathbb{T} \rightarrow \R_{\geq0} $ einer transportierten Substanz (in unserem Modell das Öl).  \newline
Gegeben sei dazu die Anfangsverteilung $\rho_0: \mathbb{D} \rightarrow \R_{\geq0}$ und der Einfluss der Substanz über die Zeit
$
\rho_{\text{in}} : \Gamma_{\text{in}} \times \mathbb{T} \rightarrow \R_{\geq0} \text{ mit }  \Gamma_{\text{in}} \coloneqq  \{ z \in \partial \mathbb{D}: q(z)\cdot n(z) \leq 0 \} \subset  \partial \mathbb{D}
$.
Dabei ist $n(z)$ der äußere Normalenvektor im (Rand-)Punkt $z$.
Wir bedienen uns wieder der Physik und fordern die Erfüllung der Bilanzgleichung
\begin{align*}
\forall K \subseteq \mathbb{D} , t\in \mathbb{T} : \frac{d}{dt} \int_K \rho(x,t) \, \mathrm{d}x + \int_{\partial K} \rho(x,t)q(x)\cdot n(x) \, \mathrm{d}a = 0.
\end{align*}
Wenden wir für ein zulässiges $K \subseteq \mathbb{D}$ und $\rho,q \in C^1(\mathbb{D})$ den Satz von Gauß an erhalten wir
\begin{align*}
\int_K \partial_t \rho(x,t) + \dive (\rho q)(x,t) \, \mathrm{d}x = 0
\end{align*}
und können so die (lineare) Transportgleichung ableiten:
\begin{align*}
\partial_t \rho + \dive (\rho q) = 0 \text{ in } \mathbb{D} \times (0,T]
\end{align*}
Mit den entsprechenden Rand- und Anfangswerten erhalten wir so:

\[ 
\text{Bestimme } \rho: \overline{\mathbb{D}} \times \mathbb{T} \to \R_{\geq 0} \text{, sodass} \newline \]
\[\setlength\arraycolsep{1pt}
\text{(TP)}\begin{cases} 
\begin{array}{rlll}
\partial_t \rho + \dive(\rho q) &= 0 &\text{ ,in } &\mathbb{D} \times (0,T)\\
\rho(x,t) &= \rho_{\text{in}}(x,t) &\text{ ,auf } &\Gamma_{\text{in}} \times (0,T)\\
\rho(x,0) &= \rho_0(x) &\text{ ,auf } &\mathbb{D} \\
\end{array}
\end{cases} \\
\]

\subsubsection{probabilistisches Problem}
In dem letzten Unterabschnitt sind wir bereits bei der Lösung des Potentialströmungsproblems davon ausgegangen, sämtliche benötigten Randwerte sowie den Permeabilitätstensor $\kappa$ exakt für das gesamte Gebiet $\mathbb{D}$ zu kennen.
Wir wollen uns von dieser durchaus starken Annahme lösen und deshalb zusätzlich die Permeabilität $\kappa$ mit Mitteln der Stochastik modellieren.
Sei dazu $(\Omega, \mathcal{A},\mathbb{P})$ ein Wahrscheinlichkeitsraum und ab nun $d=2$, also $\mathbb{D} \subseteq \R^2$.
\begin{Bemerkung}
	Grundsätzlich funktionieren die vorgestellten Verfahren auch für $d=3$, wir wollen uns aber der Anschaulichkeit halber auf zwei Dimensionen beschränken. Das so betrachtete Gebiet $\mathbb{D}$ lässt sich so z.B. als Querschnitt einer Bodenschicht interpretieren.
\end{Bemerkung} 
Weiter sei nun $\kappa (\cdot,x): \Omega \rightarrow \R_{\geq0}$ die (vom Zufall abhängige) Permeabilität.
Wie schon an anderer Stelle (z.B. in \cite{kumar2018multigrid}) wollen wir die Permeabilität als lognormal-Feld modellieren.
Unser so entstehendes Problem fällt somit in den Bereich der Uncertainty Quantification und lautet: 


\begin{align*}
\setlength\arraycolsep{1pt}
&\text{Für } \omega \in \Omega \text{, bestimme } u(\omega,\cdot):\overline{\mathbb{D}} \to \R \text{ und } q(\omega, \cdot): \overline{\mathbb{D}} \to \R^2 \text{ mit } \\
&\text{(PS)}
	\begin{cases}
		\begin{array}{rlll}
		\dive (q(\omega,x)) &= 0  &\text{, für } x \in \mathbb{ D}\\  
		q(\omega,x) &= - \kappa(\omega) \nabla u(\omega,x)  &\text{, für } x \in \mathbb{D}\\
		-q(\omega,x) \cdot n &= g_N(x)  &\text{, für } x \in \Gamma_N \\
		u(\omega,x) &= u_D(x)  &\text{, für } x \in \Gamma_D \\
		\end{array}
	\end{cases} \\
&\text{Für } \omega \in \Omega \text{ und }q(\omega,\cdot): \overline{\mathbb{D}} \to \R^2 \text{, bestimme }\rho(\omega,\cdot): \overline{\mathbb{D}} \times \mathbb{T} \to \R_{\geq 0} \text{ mit} \\
&\text{(TP)} 
	\begin{cases}
		\begin{array}{rlll}
			\partial_t \rho (\omega, x, t) + \dive(\rho(\omega,x,t)q(\omega,x)) &= 0 &\text{, für } (x,t) \in \mathbb{D} \times (0,T] \\
			\rho(\omega,x,t) &= \rho_{\text{in}}(x,t) &\text{, für } (x,t) \in \Gamma_{\text{in}} \times \mathbb{T} \\
			\rho(\omega,x,0)  &= \rho_0(x) &\text{, für } x \in  \mathbb{D}
		\end{array}
	\end{cases} \\
&\text{ für die Anfangs- und Randwerte: } \\ 
	&\begin{array}{llr}
		g_N&: \Gamma_{\text{N}} \to \R \\
		u_D&: \Gamma_{\text{D}} \to \R \\
	    \rho_{\text{in}}&: \Gamma_{\text{in}} \times \mathbb{T} \to \R_{\geq0} \\
		\rho_0&: \mathbb{D} \to \R_{\geq0} \\
	\end{array} \newline \\
&\text{ wobei } \partial \mathbb{D} = \Gamma_{\text{D}} \dot{\cup} \Gamma_{\text{N}}  \text{ und }  \Gamma_{\text{in}} \coloneqq  \{ z \in \partial \mathbb{D}: q(z)\cdot n(z) \leq 0 \} \subset  \partial \mathbb{D}
\end{align*}
Dabei stellen wir uns die Aufgabe, den Erwartungswert eines gegebenen Zielfunktionals $Q(\rho)$ zu berechnen, etwa dem Ausfluss der transportierten Substanz über den Rand. An dieser Stelle können wir dann, nachdem wir uns in den nächsten zwei Unterabschnitten damit beschäftigt haben, wie wir obige Probleme numerisch lösen, die MLMC Methode nutzen, um diesen Erwartungswert zu berechnen. 

 %%%%%%%%%%%%%%%%%%%%%%%%%%%%%%%%%
 \newpage  % neuer Abschnitt auf neue Seite, kann auch entfallen
 %%%%%%%%%%%%%%%%%%%%%%%%%%%%%%%%%
\subsection{Numerische Lösung des Potentialströmungsproblem}
% !TeX root = bachelorarbeit.tex
\begin{Bemerkung}
	Die beiden folgenden Abschnitte bauen im Wesentlichen auf den beiden Vorlesungen 'Einführung in das Wissenschaftliche Rechnen' (SS 2019) und 'Finite Elemente Methoden' (WS 2019/2020) von Herrn Prof. Dr. Wieners auf. Dem entsprechend sind als Quellen neben \cite{brenner2007mathematical} und 
	\cite{braess2013finite} vor allem die Mitschriebe zu den oben genannten Vorlesungen, sowie die Berichte zum Rechnerpraktikum mit M++ \cite{siteM++} zu nennen.
\end{Bemerkung}
Wie bereits in obigem Abschnitt erwähnt, sollen sich die nächsten beiden Abschnitte damit beschäftigen, wie wir die oben beschriebenen Probleme für ein festes $\omega \in \Omega$ numerisch lösen können. 
Ein Überblick über alle möglichen Verfahren, welche zur Lösung der beiden Probleme geeignet sind, würde den Rahmen dieser Thesis sprengen. Wir wollen deshalb im Folgenden auf eine Möglichkeit eingehen diese Berechnung numerisch durchzuführen. Insbesondere werden dabei jene Verfahren beschrieben, welche wir auch später innerhalb der MLMC Methode in M++ nutzen wollen.
Da wir in diesen beiden Abschnitte $\omega \in \Omega$ ohnehin fest halten, genügt es zudem das deterministische Problem zu betrachten. \newline
Sowohl das hybride Finite Elemente Verfahren, welches wir zur Lösung des Potentialströmungsproblem nutzen wollen, als auch das Discontinuous Galerkin Vefahren, mit dessen Hilfe wir das Transportproblem lösen wollen, bauen auf der Finite Elemente Theorie auf. 
Diese ist im Wesentlichen in der zweiten Hälfte des 20. Jahrhunderts entstanden, ist aber bis heute in praktischer wie auch in theoretischer Sicht aktuell.
Die Grundidee ist hierbei die vorliegenden Rand-Anfangswertaufgaben in einem passenden endlichen Unterraum zu lösen. Dabei löst man sich auf analytischer Seite zunächst oft von einzelnen Regularitäts- und Differenzierbarkeitsbedingungen und führt einen sogenannten schwachen Lösungsbegriff ein (vergleiche Abschnitt 2.1). Statt nun aber solch eine schwache Lösung in einem unendlich dimensionalen Funktionenraum, wie beispielsweise in den Sobolevräumen $W^{1,2}(\mathbb{D})$ oder $W_0^{1,2}(\mathbb{D})$ zu bestimmen, zieht man sich auf endliche Unterräume zurück. \newline
Die folgende Definition entstammt \cite{brenner2007mathematical} und geht ursprünglich (1978) auf Ciarlet zurück.
\begin{Definition}\
	Sei
	\begin{itemize}
		\item $K \subseteq \R^d$ eine beschränkte abgeschlossene Menge mit einem nichtleeren Inneren und stückweise stetig differenzierbarem Rand 
		\item $\mathcal{P}$ ein endlich dimensionaler Funktionenraum auf K
		\item $\mathcal{N} = \{N_1,N_2,\dots,N_k \}$ eine Basis für $\mathcal{P}^{'}$
	\end{itemize}
	Dann heißt $(K,\mathcal{P},\mathcal{N})$ ein finites Element.
\end{Definition}

Wir wollen im Folgenden diese theoretische Definition zwar im Hinterkopf behalten, aber wie in \cite{braess2013finite} meist nur mit den sogenannten Finite-Elemente-Räumen arbeiten. 
Dabei wird eine geeignete Zerlegung $\mathfrak{K} = \{K_1,K_2,\dots, K_M \}$ von $\mathbb{D}$ in endlich viele Teilgebiete gewählt. 
Anschließend betrachten wir einen endlichen Raum von Funktionen, die eingeschränkt auf diese Teilgebiete von einfacher Gestalt sind, beispielsweise bieten sich oft polynomielle Darstellungen niedrigen Grades an. 
Ein solches Teilgebiet $K \in \mathfrak{K}$ nennen wir Finites Element oder auch Zelle und fordern implizit verbunden mit dem betrachteten Funktionenraum die Erfüllung der obigen Definition. \newline
Im Falle $\mathbb{D} \subseteq \R^2$ kommen so z.B. Dreiecke oder Vierecke in Frage, in $\mathbb{D} \subseteq \R^3$ können Tetraeder, Würfel, Quader und andere genutzt werden. \newline
Sei nun $\mathbb{D} \subseteq \R ^2$ zudem ein polygonales Gebiet, um eine einfache Zerlegung in Dreiecke oder Vierecke zu gewährleisten.

\begin{Definition}
	\begin{enumerate}
		\item Eine Zerlegung $\mathfrak{K} = \{ K_1,K_2,\dots,K_M\}$ von $\mathbb{D}$ in Dreiecks- oder Viereckselemente heißt zulässig, wenn folgende Eigenschaften erfüllt sind:
		\begin{itemize}
			\item $\overline{\mathbb{D}} = \bigcup_{i=1}^M K_i$
			\item Für $i \neq j$ ist $K_i\cap K_j$
			\begin{enumerate}
				\item ein gemeinsamer Eckpunkt von $K_i$ und $K_j$
				\item eine gemeinsame Kante von $K_i$ als auch von $K_j$
				\item oder $K_i\cap K_j= \emptyset$
			\end{enumerate}
			
		\end{itemize}
		\item Wir schreiben oft $\mathfrak{K}_h$ anstatt $\mathfrak{K}$, wenn jedes Element einen Durchmesser von höchstens $2h$ besitzt [Vorlesung h = max diam K was passt besser]
	\end{enumerate}
\end{Definition}

\begin{figure}[h]
	\centering
	\captionabove{Zulässige Triangulierung und unzulässige Triangulierung mit hängendem Knoten}
	\includegraphics[width=0.8\textwidth]{triangulierung.png} \\
	Abbildung aus \cite{braess2013finite} Seite 58
\end{figure}

\subsubsection{Schwache Formulierung}
Betrachten wir also das Potentialströmungsproblem (NEW ALIGNMENT? do everywhere):
\[ \text{Bestimme } u:\overline{\mathbb{D}} \to \R \text{ und } q: \overline{\mathbb{D}} \to \R^2 \text{ mit } \newline \]
\[\setlength\arraycolsep{1pt}
\text{(PS)}\begin{cases} 
\begin{array}{rlcl}
\dive q     &= 0                 &\text{ ,} \text{in } \mathbb{D}\\
q           &= - \kappa \nabla u &\text{ ,} \text{in }\mathbb{D}\\
u           &= u_D               &\text{ ,} \text{auf } \Gamma_D \\
-q \cdot n  &= g_N               &\text{ ,} \text{auf } \Gamma_N 
\end{array}
\end{cases} 
\]

%%%%%%%%%%%%%%%%%%%%%%%%%%%%%%%%%
\newpage  % neuer Abschnitt auf neue Seite, kann auch entfallen
%%%%%%%%%%%%%%%%%%%%%%%%%%%%%%%%%
\subsection{Numerische Lösung des Transportproblem}
% !TeX root = bachelorarbeit.tex
 \label{DG}
In diesem Abchnitt soll nun, nachdem wir $q(\omega,\cdot)$ als Finite-Elemente-Lösung des Potentialströmungsproblems erhalten haben, die numerische Lösung des linearen Transportproblems behandelt werden:
\begin{align*}
	&\text{Für } \omega \in \Omega \text{ und }q(\omega,\cdot): \overline{\mathcal{D}} \to \R^2 \text{, bestimme }\rho(\omega,\cdot): \overline{\mathcal{D}} \times \mathbb{T} \to \R_{\geq 0} \text{ mit} \\
	&\text{(pTP)} 
	\begin{cases}
	\begin{array}{rlll}
	\partial_t \rho (\omega, x, t) + \dive(\rho(\omega,x,t)q(\omega,x)) &= 0 &\text{, für } (x,t) \in \mathcal{D} \times (0,T] \\
	\rho(\omega,x,t) &= \rho_{\text{in}}(x,t) &\text{, für } (x,t) \in \Gamma_{\text{in}} \times \mathbb{T} \\
	\rho(\omega,x,0)  &= \rho_0(x) &\text{, für } x \in  \mathcal{D}
	\end{array}
	\end{cases} \\
\end{align*}
Insbesondere wollen wir an dieser Stelle wieder $\omega \in \Omega$ festhalten und betrachten deshalb zunächst nur das deterministische Problem wie in \ref{det_prob}:
\[ 
\text{Bestimme } \rho: \overline{\mathcal{D}} \times \mathbb{T} \to \R_{\geq 0} \text{, sodass} \newline \]
\[\setlength\arraycolsep{1pt}
\text{(dTP)}\begin{cases} 
\begin{array}{rlll}
\partial_t \rho(x,t) + \dive(\rho(x,t) q(x)) &= 0 &\text{ ,in } &\mathcal{D} \times (0,T)\\
\rho(x,t) &= \rho_{\text{in}}(x,t) &\text{ ,auf } &\Gamma_{\text{in}} \times (0,T)\\
\rho(x,0) &= \rho_0(x) &\text{ ,auf } &\mathcal{D} \\
\end{array}
\end{cases} \\
\]
 Wir greifen dabei auf ein sogenanntes discontinuous Galerkin Verfahren zurück, welches für diese Problemklasse bereits an anderen Stellen (z.B. in \cite{cockburn1998runge}) erprobt wurde. Ursprünglich geht das Discontinuous Galerkin Verfahren auf Reed und Hill \cite{reed1973triangular} zurück. Einen guten (wenn auch mittlerweile etwas in die Jahre gekommenen) Überblick über die Anwendung von discontinuous Galerkin Verfahren bietet \cite{cockburn2000development}.
Grundätzlich handelt es sich beim discontinuous Galerkin Verfahren ebenfalls um einen FEM Ansatz, der zwar Ähnlichkeiten zum Finite Elemente Verfahren aufweist, welches wir im letzten Abschnitt gesehen hatten, aber auch einige bedeutende Unterschiede besitzt, auf welche wir im Folgenden besonders eingehen wollen. 
%So werden wir wieder eine schwache Formulierung des analytischen Problems herleiten und uns dann im Rahmen des diskretisierung erneut auf endlich dimensionale Räume zurückziehen.
Anders als zuvor das Potentialströmungsproblem ist die lineare Transportgleichung  nämlich sowohl orts- als auch zeitabhängig. Daher werden wir
%, nachdem wieder zuerst eine schwache Formulierung eingeführt wird,
  die lineare Transportgleichung zunächst im Ort diskretisieren. Wir erhalten so eine Semidiskretisierung, welche wir anschließend mit einem Zeitintegrator, wie beispielsweise der impliziten Mittelpunktsregel, in eine Volldiskretisierung überführen.
 

%\subsubsection{Schwache Formulierung}
%Wie im letzten Abschnitt führen wir nun einen schwachen Lösungsbegriff ein. Sei dazu $ \phi : \mathcal{D} \times \mathbb{T} \to \R $ eine beliebige Testfunktion, etwa aus  $W_0^{1,2}(\mathcal{D} \times \mathbb{T}) $, für die $ \phi(\cdot,T) = 0 $ gelte. 
%Wir beginnen mit der differentialgleichung $ \partial_t \rho(x,t) + \dive(\rho(x,t) q(x)) = 0  $, multiplizieren zunächst mit der Testfunktion $ \phi $ und integrieren anschließend über den Raum-Zeitzylinder $ \mathcal{D} \times \mathbb{T} $:
%\begin{align*}
%	\int_{\mathcal{D} \times \mathbb{T}} \big(\partial_t \rho(x,t) &+ \dive(\rho(x,t) q(x) ) \big) \phi(x,t) \; \mathrm{d} (x,t) =  \\
%	&\underbrace{\int_{\mathcal{D} \times \mathbb{T}}  \partial_t \rho(x,t) \phi(x,t) \; \mathrm{d} (x,t)}_{(1)} +\underbrace{\int_{\mathcal{D} \times \mathbb{T}}  
%    \dive(\rho(x,t) q(x) ) \phi(x,t)\; \mathrm{d} (x,t)}_{(2)}
%\end{align*}
%Betrachten wir nun zunächst Integral (1), so folgt mit partieller Integration:
%\begin{align*}
%	\int_{\mathcal{D} \times \mathbb{T}}  \partial_t \rho(x,t) &\phi(x,t) \; \mathrm{d} (x,t) = \int_{\mathcal{D}} \int_{\mathbb{T}} \partial_t \rho(x,t) \phi(x,t) \dt \dx \\&= \int_{\mathcal{D}} \big( - \int_{\mathbb{T}} \rho(x,t) \partial_t \phi(x,t) \dt + \big[ \rho(x,t) \phi(x,t) \big]_0^T  \big) \dx \\
%	&= - \int_{\mathcal{D} \times \mathbb{T}} \rho(x,t) \partial_t \phi(x,t) \; \mathrm{d} (x,t) + \int_{\mathcal{D}} \underbrace{\rho(x,T) \phi(x,T)}_{ = 0 } - \underbrace{\rho(x,0)}_{ = \rho_0(x) \text{ auf } \mathcal{D}} \phi(x,0)  \dx\\
%	&= - \int_{\mathcal{D} \times \mathbb{T}} \rho(x,t) \partial_t \phi(x,t) \; \mathrm{d} (x,t) - \int_{\mathcal{D}} \rho_0(x) \phi(x,0) \dx 
%\end{align*}
%Außerdem können wir Integral (2) mit  \ref{n_pI} wie folgt ausdrücken:
%\begin{align*}
%	\int_{\mathcal{D} \times \mathbb{T}}  
%	\dive(\rho(x,t) &q(x) ) \phi(x,t)\; \mathrm{d} (x,t) = \int_{\mathbb{T}} \int_{\mathcal{D}} \dive(\rho(x,t) q(x) ) \phi(x,t) \dx \dt \\
%	&= \int_{\mathbb{T}} \big( - \int_{\mathcal{D}} \rho(x,t) q(x) \nabla \phi(x,t) \dx + \int_{\partial \mathcal{D}} \rho(x,t) q(x) \cdot n \ \phi(x,t) \da \big)  \dt \\
%	&= - \int_{\mathcal{D} \times \mathbb{T}} \rho(x,t) q(x) \nabla \phi(x,t)) \; \mathrm{d} (x,t) + \int_{\mathbb{T}} \int_{\partial \mathcal{D}} \rho(x,t) q(x) \cdot n \phi(x,t) \da \dt 
%\end{align*}
%Mit $ \partial \mathcal{D} = \Gamma_{\text{in}} \dot{\cup} \Gamma_{\text{out}} $ und 
%$\rho(x,t) = \rho_{\text{in}}(x,t) \text{ für } (x,t) \in \Gamma_{\text{in}} \times (0,T)$ erhalten wir so die folgende Formulierung:
%\begin{definition} 
%$ \rho \in L_1 (\mathcal{D} \times (0,T)) $ heißt schwache Lösung des linearen Transportproblems, falls es für ein gegebenes $ q : \overline{\mathcal{D}} \to \R^2 $ folgende Bedingungen erfüllt:
%\begin{align*}
%\text{(swTP)}
%\begin{cases}
%\begin{array}{rlll}
%\displaystyle
%\int_{\mathcal{D}} \rho_0 \phi(0) \dx = \mkern-16mu &- \displaystyle \int_{0}^{T} \int_{\mathcal{D}} \rho (\partial_t \phi + q \nabla \phi ) \dx \dt \\
%&+\displaystyle\int_{0}^{T}  \int_{\Gamma_{\text{in}}} \rho_{\text{in}} q \cdot n \phi \da  \dt \\
%&+\displaystyle\int_{0}^{T}  \int_{\Gamma_{\text{out}}} \rho q \cdot n \phi \da  \dt
%\end{array}
%\end{cases}	
%\end{align*}
%für alle Testfunktionen $ \phi: \mathcal{D} \times (0,T) \to \R $ mit $ \phi(\cdot,T) = 0 $ auf $ \mathcal{D}  $ und $ \phi|_{\Gamma_{\text{out}}} = 0 $.
%\end{definition}
%dabei ist $  \Gamma_{\text{out}} \coloneqq  \{ z \in \partial \mathcal{D}: q(z)\cdot n(z) > 0 \}$ 
%und $  \Gamma_{\text{in}} \coloneqq  \{ z \in \partial \mathcal{D}: q(z)\cdot n(z) \leq 0 \} $. \\
%Obige Herleitung zeigt zusammen mit \ref{testfunktionen} insbesondere die Gültigkeit des folgenden Zusammenhangs zwischen klassischen Lösungen der linearen Transportgleichung und schwachen Lösungen von (swTP):
%
%\begin{Lemma}(Zusammenhang der Lösungsbegriffe)
%	\begin{enumerate}
%		\item Ist $ \rho $ eine klassische Lösung, so ist $ \rho $ auch eine schwache Lösung.
%		\item Ist $ \rho \in C^2(\mathcal{D} \times \mathbb{T} , \R )$ und eine schwache Lösung, so ist $ \rho $ auch eine klassische Lösung. 
%	\end{enumerate}
%\end{Lemma}
\subsubsection{Diskretisierung}
\label{Diskretisierung}
Wie bereits weiter oben beschrieben, werden wir im Folgenden zunächst den Raum diskretisieren und anschließend die so entstandene Semidiskretisierung in eine Volldiskretisierung auflösen. Insgesamt wollen wir das discontinuous Galerkin Verfahren mit einem Zeitintegrator, wie der impliziten Mittelpunktsregel oder einem klassischen Runge-Kutta-Verfahren nutzen. Zunächst führen wir die analytische Flussfunktion ein. 
\begin{Definition}(Flussfunkion) \\
	\label{Flussfunktion}
	Zu einem gegebenen Flussvektorfeld $ q : \mathcal{D} \to \R^2 $ ist die Flussfunktion $ \Upsilon$ definiert als:\\
	\begin{align*}
		 \Upsilon : \text{Abb}(\mathcal{D}\times\mathbb{T},\R) &\to \text{Abb}(\mathcal{D}\times\mathbb{T},\R^2) \\
		 \rho &\mapsto \rho q
	\end{align*}
\end{Definition}
$\newline$
Für eine klassische Lösung $ \rho  $ von (dTP) gilt dann insbesondere $ \partial_t \rho = - \dive (\Upsilon(\rho)) $ auf $ \mathcal{D} \times (0,T] $.	\\
Halten wir also zunächst $ t \in \mathbb{T} $ und leiten so die Semidiskretisierung her.\\
Sei nun $ \mathcal{T} $ eine zulässige Triangulierung von $ \mathcal{D} $ aus Dreiecken wie in \ref{num_pot} und  $ (\cdot , \cdot)_A $ das $ L^2(A)-$Skalarprodukt.
Wir wählen als Lösungs-/Testraum $\mathcal{Q}_h = \prod_{K \in \mathcal{T}} \mathbb{P}_p(K,\R) $ für ein festes $p \geq 0 $. Anders als zuvor fordern wir für unsere Lösungs- und Testfunktionen diesmal aber \underline{nicht} die Stetigkeit auf $\mathcal{D}$. Da so $\mathcal{Q}_h$ nicht im betrachteten analytischen Lösungs- und Testraum liegt, etwa  $\mathcal{Q}_h \nsubseteq H^1(\mathcal{D})$, nennt man $\mathcal{Q}_h$ auch einen nicht-konformen Ansatzraum.
Außerdem lässt sich im Allgemeinen auch die später bestimmte Lösung $ \rho_h \in \mathcal{Q}_h $ (definiert auf $\mathcal{D}_h = \bigcup_{K \in \mathcal{T}} K$ ) nicht stetig auf $ \mathcal{D} $ fortsetzen, denn für eine beliebige innere Kante $ F $ kann der Grenzwert von $ \rho_h $ auf den anliegenden Zellen $ K,K' $ ($ \overline{F} = \partial K \cap \partial K' $) unterschiedlich sein. \\
Trotzdem müssen wir auch auf den inneren Kanten $ \mathcal{F}^0 \subset \mathcal{F} $ festlegen, welcher Grenzwert in einem solchen Falle gewählt wird. \\
Dazu führen wir als Pendant zur analytischen Flussfunktion (vgl. \ref{Flussfunktion})
auch eine numerische Flussfunktion ein. Grundsätzlich kommen mehrere solche Flussfunktionen in Frage, welche direkten Einfluss auf Eigenschaften des entstehenden Verfahrens besitzen. Wir entscheiden uns an dieser Stelle für den weit verbreiteten 
sogenannten upwind flux:
\begin{Definition}(upwind flux)\\
	Sei $K \in \mathcal{T}$ eine beliebige Zelle und $ F \in \mathcal{F}_K$ eine Kante von $K$. Dann ist
	\begin{align*}
		\Upsilon^{\star} : \text{Abb}(\mathcal{D}\times\mathbb{T},\R) &\to \text{Abb}(\mathcal{D}\times\mathbb{T},\R^2) \\
		\rho_h &\mapsto 
		\begin{cases}
			\Upsilon(\rho_h|_K) , \ \text{ für } q\cdot n_F^K \geq 0 \\  
			\Upsilon(\rho_h|_{K'}) ,\text{ für } q\cdot n_F^K < 0 \text{ und } \overline{F} = \partial K \cap \partial K'
		\end{cases}
	\end{align*}
\end{Definition}
Sei nun also $ \rho $ klassische Lösung von (dTP) mit $ \partial_t \rho = -\dive(\Upsilon(\rho)) $ auf $ \mathcal{D} $. Dann gilt nach Satz von Gauß:
\begin{align}
		\label{gaus1}
		\int_{\partial \mathcal{D}} \rho q \cdot n \phi \da  =\int_{\partial \mathcal{D}} \Upsilon(\rho) \cdot n \phi \da = \int_{\mathcal{D}} \dive (\Upsilon(\rho)\phi) \dx
\end{align}
Das Integral über den Rand von $ \mathcal{D} $ können wir nach der folgenden kleinen Vorüberlegung auch als Integral über alle Kanten der gewählten Zerlegung $ \mathcal{T} $ ausdrücken: \\
Es gilt nämlich für alle inneren Kanten, also solche Kanten $ F $, für die zwei Zellen $ K $ und $ K' $ existieren, sodass $ \overline{F} = K \cap K' $ ist, dass $ \int_F \Upsilon^{\star}(\rho) \cdot n^K \phi \da = - \int_F \Upsilon^{\star}(\rho) \cdot n^{K'} \phi \da $ stets erhalten ist. \\
Summieren wir also zunächst über alle Zellen, summieren anschließend die Integrale über alle Kanten und ersetzen dabei den analytischen durch den numerischen Fluss, erhalten wir gerade wieder obiges Randintegral. Es gilt also:
\begin{align*}
	\sum_{K\in \mathcal{T}} \sum_{F \in \mathcal{F}_K} \int_F \Upsilon^{\star}(\rho) \cdot n^K \phi \da = \int_{\partial \mathcal{D}} \Upsilon(\rho) \cdot n \phi \da \stackrel{\text{\ref{gaus1}}}{=} \int_{\mathcal{D}} \dive (\Upsilon(\rho)\phi) \dx
\end{align*}
Nach der Produktregel der Divergenz lässt sich das letzte Integral auswerten zu:
\begin{align*}
	 \int_{\mathcal{D}} \dive (\Upsilon(\rho)\phi) \dx = \int_{\mathcal{D}} \phi \dive(\Upsilon(\rho)) + \Upsilon(\rho) \cdot \nabla \phi \dx \stackrel{\text{ Vor. an }\rho }{=} - \int_{\mathcal{D}} \partial_t \rho \phi \dx + \int_{\mathcal{D}} \Upsilon(\rho) \cdot \nabla \phi \dx
\end{align*}
Durch Umstellen und das Zusammenfassen der obigen Resultate erhalten wir so: 
\begin{align*}
	\sum_{K \in \mathcal{T}} \int_K \partial_t \rho  \phi \dx = \sum_{K \in \mathcal{T}} \int_K \Upsilon(\rho ) \cdot \nabla \phi \dx - 	\sum_{K\in \mathcal{T}} \sum_{F \in \mathcal{F}_K} \int_F \Upsilon^{\star}(\rho) \cdot n^K \phi \da
\end{align*}
Dabei wurde zusätzlich ausgenutzt, dass es sich bei den Kanten um Nullmengen handelt und wir so das Integral über $ \mathcal{D} $ als Summe der Integrale über alle Zellen auffassen können. Nutzen wir nun noch aus, dass für den Fluss  $\rho(x,t) = \rho_{\text{in}}(x,t)$ für $ x \in \Gamma_{\text{in}} $ gilt, kommen wir so auf 

\begin{align}
	\label{fastfertigsemi}
	\sum_{K \in \mathcal{T}} \int_K \partial_t \rho  \phi \dx = \sum_{K \in \mathcal{T}} \int_K \Upsilon(\rho ) \cdot \nabla \phi \dx - 	\sum_{K\in \mathcal{T}} \left( \sum_{\substack{F \in \mathcal{F}_K \\ F \not\subseteq \Gamma_{\text{in}}}} \int_F \Upsilon^{\star}(\rho) \cdot n^K \phi \da + \sum_{\substack{F \in \mathcal{F}_K \\ F \subseteq \Gamma_{\text{in}}}} \rho_{\text{in}} q \cdot n^K \phi \da \right)
\end{align}
 Die Semidiskretisierung ist motiviert durch (\ref{fastfertigsemi}) und lautet: Bestimme  $\rho_h \in \mathcal{Q}_h$, sodass für alle $ \phi_h \in \mathcal{Q}_h $ gilt:
\begin{align}
	\label{Semidiskretisierung}
	\sum_{K \in \mathcal{T}} (\partial_t \rho_h, \phi_h)_{K} &= \sum_{K \in \mathcal{T}} \left( (\Upsilon(\rho_h), \nabla \phi_h)_{K} - \sum_{\substack{F \in \mathcal{F}_K \\ F \not \subseteq \Gamma_{\text{in}}}}(\Upsilon^{\star}(\rho_h)\cdot n^K,\phi_h)_{F} - (\rho_{\text{in}}q \cdot n^K,\phi_h)_{\partial K \cap \Gamma_{\text{in}}} \right)
\end{align}
 Sei nun $ G $ die Anzahl der Zellenfreiheitsgrade. Diese hängen direkt von der Wahl des Lösungs-/Testraumes $ \mathcal{Q}_h $, bzw. der Wahl von $ p $ ab. Die Wahl $ p=0 $ liefert gerade ein Finite Volumen Verfahren, $ G $ ist dann gerade $ 1 $ und die Zellenbasis $ \{\mu_K \}_{K \in \mathcal{T}}$ ist durch $ \mu_K =  \mathds{1}_{K} $ gegeben.
 Für $ p \geq 1 $ erhalten wir hingegen ein Discontinuous Galerkin Verfahren. Die Anzahl der Zellfreiheitsgrade einer Zelle $ K $ ist dann gerade durch $ G = p \cdot \abs{\mathcal{V}_K} $ (also $ p $-mal die Anzahl der Ecken von $ K $) gegeben.
 Die Zellenbasis hat in diesem Fall $ G \cdot \abs{\mathcal{T}} $ Elemente und für je $ G $ Basisfunktionen $ \mu $ gilt $ \supp(\mu) \subseteq \overline{K} $. \\
 Sei $ M = \abs{\mathcal{T}} $ die Anzahl der Zellen und $ \mathcal{T} = \{K_1,\dots,K_M \} $ geordnet. Dann lässt sich die Zellenbasis $ \{ \mu_i \}_{i=1}^{GM} \subset \mathcal{Q}_h $ gemäß dem Träger der Basisfunktion zu $ \{ \mu_{K_j,i} \}_{j = 1,\dots,M ; i = 1,\dots,G } $ ordnen.
 Für jede feste Zelle $ K $ ergibt sich durch Einsetzen der zugehörigen Basisfunktionen $ \{ \mu_{K,i} \}_{i = 1,\dots,G }  $ : \\
\begin{align*} 
\left(\partial_t \rho_h, \mu_{K,i}  \right)_{K}  &= \bigg( \left(\Upsilon(\rho_h), \nabla\mu_{K,i}\right)_{K} - \sum_{\substack{F \in \mathcal{F}_{K} \\ F \not\subseteq \Gamma_{\text{in}}}} \left(\Upsilon^*(\rho_h) \cdot n^{K}, \mu_{K,i} \right)_{F} - \left(\rho_{\text{in}} q \cdot n^{K}, \mu_{K,i} \right)_{\partial K \cap \Gamma_{\text{in}}} \bigg) \\
\end{align*}
Wir erhalten so folgende Darstellung:
%\begin{align*}
%\left(\partial_t \rho_h, \mu_i  \right)_{K_i} = \bigg( - \sum_{\substack{F \in \mathcal{F}_{K_i}, F \not\subseteq \Gamma_{\text{in}}\\ F \text{ mit }q \cdot n^K_{|F} > 0} } \left(\underbrace{\Upsilon^*(\rho_h)}_{= \rho_{h|K}\, q} \cdot n^{K_i}, \mu_i \right)_{F} - \sum_{\substack{F \in \mathcal{F}_{K_i}, F \not\subseteq \Gamma_{\text{in}}\\ F \text{ mit }q \cdot n^K_{|F} < 0} } \left(\underbrace{\Upsilon^*(\rho_h)}_{= \rho_{h|K'} \, q} \cdot n^{K_i}, \mu_i \right)_{F}\\
% - \left(\rho_{\text{in}} q \cdot n^{K_i}, \mu_i \right)_{\partial K_i \cap \Gamma_{\text{in}}} \bigg)
%\end{align*} 
%%
\begin{align}
\label{DarstellungvorLGS}
\underbrace{\left(\partial_t \rho_h, \mu_{K,i}  \right)_{K}}_{\eqqcolon M_K(\partial_t \rho_{h}, \mu_{K,i})} = 
\underbrace{ \left(\Upsilon(\rho_h), \nabla\mu_{K,i}\right)_{K} -  \sum_{F \in \mathcal{F}_{K} , F \not \subseteq \Gamma_{\text{in}}}\left(\Upsilon^{\star}(\rho_h)\cdot n^{K},\mu_{K,i}\big)_F \right)}_{\eqqcolon A_K(\rho_{h},\mu_{K,i})} \underbrace{- \left(\rho_{\text{in}}q\cdot n^{K},\mu_{K,i} \right)_{\partial K \cap \Gamma_{\text{in}}}}_{b_K(\mu_{K,i})}
\end{align}
%
Zusammen mit der Basisdarstellung von $ \rho_h $ in $\bigcup_{K \in \mathcal{T}}\{\mu_{K,i}  \}_{i=1}^G$, also 
\[ \rho_h = \underline{\rho} \cdot (\mu_{K_1,1},\dots,\mu_{K_1,G},\mu_{K_2,1}\dots,\mu_{K_N,1},\dots,\mu_{K_N,G})^T\]
mit $ \underline{\rho}\in \R^{MG} $ und 
$ \Upsilon^{\star}(\rho_h) = 
\begin{cases} 
\rho_h|_K \ ,\text{falls } q\cdot n^K|_F \geq 0\\
\rho_{h}|_{K'} \ , \text{falls } q\cdot n^K|_F < 0
\end{cases} 
$
 können wir \ref{DarstellungvorLGS} in eine gewöhnliche Differentialgleichung erster Ordnung umformulieren. 
% s
%%Dazu klammern wir jeweils $ \underline{\rho} $ aus und definieren 
%%$ \text{mithilfe von (\ref{DarstellungvorLGS}.1) die Massenmatrix } \underline{M}\in \R^{N \times N} $
%%\begin{align*}
%% &\underline{M}[K,K'] \coloneqq \begin{dcases}
%%\int_K \abs{\mu_K}^2 \dx & \text{, für } K = K' \\
%%0 &\text{, sonst}
%%\end{dcases} \\
%%&\text{mit (\ref{DarstellungvorLGS}.2) die Flussmatrix } \underline{A}\in \R^{N \times N} \\ & \underline{A}[K,K'] \coloneqq \begin{dcases}
%%\int_K  \nabla\mu_i^2 \dx - \sum_{\substack{F\in \mathcal{F}_K \\ F \text{ mit } q\cdot n^K_{|F} > 0}} \int_F \mu_K^2 q \cdot n^K \da& \text{, für } K = K' \\
%%\int_K \Upsilon(\rho_h), \nabla\mu_i \dx - \int_F \mu_K \mu_{K'} q \cdot n^K \da &\text{, für } q \cdot n^K< 0 \\
%%& \text{ und } \overline{F} = \partial K \cap \partial K'\\
%%\int_K \Upsilon(\rho_h), \nabla\mu_i \dx &\text{, sonst}
%%\end{dcases} \\
%%&\text{und mit (\ref{DarstellungvorLGS}.3) den Lastvektor } \underline{b}\in \R^{N} \\ &\underline{b}[K] \coloneqq \int_{\partial K \cap \Gamma_{\text{in}}} \rho_{\text{in}} q \cdot n \da
%%\end{align*}
So ergibt sich eine Differentialgleichung der Form
\begin{align*}
\begin{cases}
\underline{M} \partial_t \underline{\rho}(t) = \underline{A} \underline{\rho}(t) + \underline{b}(t) \\
\underline{\rho}(0) = \underline{\rho_0}
\end{cases}\\
\end{align*}
mit $ \underline{M},\underline{A} \in \R^{MG\times MG} $ und $ b: \mathbb{T} \to \R^{MG} $. Wir nennen  $ \underline{M} $ Massematrix, $ \underline{A} $ Flussmatrix und $ \underline{b} $ Lastvektor.
Da dies nun eine gewöhnliche Differentialgleichung ist, können wir die Lösung 
\begin{align}
 \label{semidisklsg}
 \underline{\rho}(t) = \exp(t \underline{M}^{-1} \underline{A}) \left( \underline{\rho_0} + \int_{0}^{t} \exp(-s\underline{M}^{-1} \underline{A}) \underline{b}(s) \ds \right)
\end{align}
explizit angeben.
Es handelt sich hierbei aber immer noch um eine semidiskrete Formulierung. 
Wir wollen deshalb zuletzt noch auf die Herleitung der Zeitintegratoren eingehen. Diese nutzen wir, um unter Verwendung der oben hergeleiteten Semidiskretisierung die numerische Lösung $ \underline{\rho} $ sowohl orts- al auch zeitdiskret zu berechnen. Der Ansatz leitet sich hierbei direkt aus dem Resultat (\ref{semidisklsg}) ab und besteht aus der 
Integration der Differentialgleichung $\underline{M} \partial_t \underline{\rho} = \underline{A} \underline{\rho} + \underline{b}$ über die Zeit $t$ im Intervall $[t_i, t_{i+1}]$.
Dabei ist $t_i = i \Delta t$. Hiermit folgt:
\begin{align*}
\underline{M}\underline{\rho}(t_{i+1}) - \underline{M}\underline{\rho}(t_i) = \int_{t_i}^{t_{i+1}}
\underline{M} \partial_t \underline{\rho}(t) dt =\int_{t_i}^{t_{i+1}} \underline{A} \underline{\rho}(t) + \underline{b}(t) dt.
\end{align*}
Mithilfe der Anwendung verschiedener Quadraturformeln lässt sich daraus ein Runge-Kutta Verfahren herleiten. Über die Rechteckformel 
\begin{align*}
\int_{t_i}^{t_{i+1}}\underline{A} \underline{\rho}(t) + \underline{b} dt \approx (t_{i+1}-t_i)( 
\underline{A} \underline{\rho}(t_{i+1}) + \underline{b}(t_{i+1})) = \Delta t (\underline{A} \underline{\rho}(t_{i+1}) + \underline{b}(t_{i+1}))
\end{align*}
ergibt sich z.B. das implizite Euler Verfahren
\begin{align*}
\underline{\rho}(t_{i+1}) = \underline{\rho}(t_{i}) + \Delta t \underline{M}^{-1}(\underline{A} \underline{\rho}(t_{i+1}) + \underline{b}(t_{i+1})).
\end{align*}
%Weitere Verfahren, die wir verwenden werden, sind zum einen das
%klassische Runge-Kutta Verfahren (der Übersicht wegen für $\underline{b} \equiv 0$)
%\begin{align*}
%\underline{\rho}(t_{i+1}) = \underline{\rho}(t_{i}) +
%\delta t \underline{M}^{-1}\underline{A} (
%\underline{\rho}(t_{i}) +
%\frac{\delta t}{2} \underline{M}^{-1}\underline{A} (
%\underline{\rho}(t_{i}) +
%\frac{\delta t}{3} \underline{M}^{-1}\underline{A} (
%\underline{\rho}(t_{i}) +
%\frac{\delta t}{4} \underline{M}^{-1}\underline{A} 
%\underline{\rho}(t_{i}) )))
%\end{align*}
%und zum anderen die implizite Mittelpunktsregel 
Ein weiteres Verfahren dieser Art, welches wir an dieser Stelle verwenden werden, ist die implizite Mittelpunktsregel (der Übersicht wegen für $\underline{b} \equiv 0$):
\begin{align*}
\underline{\rho}(t_{i+1}) = \underline{\rho}(t_{i}) +
\Delta t \underline{M}^{-1} (
\underline{A} 
\frac{1}{2}(\underline{\rho}(t_{i}) +
\underline{\rho}(t_{i+1}) )
+
\frac{1}{2} ( \underline{\rho}(t_{i}) +
\underline{\rho}(t_{i+1}) )).
\end{align*}

Das so entstehenden Gesamtverfahren ist aufgrund der Kombination von Discontinuous Galerkin Verfahren und Runge-Kutta-Zeitintegratoren in der Literatur oft auch unter dem Namen 'Runge–Kutta discontinuous Galerkin Methods' zu finden. Einen schönen Überblick über diese Verfahrensklasse bietet der Artikel \cite{cockburn2001runge}.
Nachdem wir nun das Discontinuous Galerkin Verfahren für die lineare Transportgleichung eingeführt und erklärt haben, sollen nun noch auf einige Eigenschaften des Verfahrens verwiesen werden. Dabei wollen wir uns aber beschränken, einige grundlegende Resultate zu nennen und so eher einen groben Überblick mit Referenzen zur Literatur zu geben. Mehr zur numerischen Analyse des Discontinuous Galerkin Verfahren findet sich zum einen in Standardwerken, wie \cite{ern2004theory}, eine schöne Zusammenstellung bietet aber auch
\cite{Har08b}. \\
Ebenfalls findet sich in \cite{Har08b} eine grundlegende numerische Analyse des Discontinuous Galerkin Verfahrens angewandt auf die stationäre lineare Transportgleichung. Dabei werden unter anderem die Konsistenz, die sogenannte Galerkin-Orthogonalität, sowie die Stabilität und Konvergenz des Verfahrens behandelt.
Mit der numerischen Analyse des Discontinuous Galerkin Verfahrens an sich befassten sich unter anderem LeSaint und Raviart \cite{lesaint1974finite}, Peterson \cite{peterson1991note} und Richter \cite{richter1988optimal}.
Runge-Kutta DG Verfahren für der linearen Transportgleichung ähnliche Problemstellungen betrachteten Cockburn und Shu in einer 5-teiligen Serie von Arbeiten. Besonders zu nennen sind dabei in unserem Kontext \cite{cockburn1989tvb} und \cite{cockburn1990runge}.


\subsection{Eigenschaften des Discontinuous Galerkin Verfahren}
%Nachdem wir nun das Discontinuous Galerkin Verfahren für die lineare Transportgleichung eingeführt und erklärt haben, sollen noch einige Eigenschaften des Verfahrens beleuchtet werden. Dabei wollen wir uns aber darauf beschränken, einige grundlegende Resultate zu nennen, und so eher einen groben Überblick mit Referenzen zur Literatur zu geben. Mehr zur numerischen Analyse des Discontinuous Galerkin Verfahren findet sich zum einen in Standardwerken, wie \cite{ern2004theory}, eine schöne Zusammenstellung bietet aber auch \cite{Har08b}.
%Ebenfalls findet sich in \cite{Har08b} eine grundlegende numerische Analyse des Discontinuous Galerkin Verfahrens auf die stationäre lineare Transportgleichung.
\subsubsection{Lösungsbegriffe}
	Wie zuvor bereits beim Potentialströmungsproblem können wir auch für das Transportproblem eine sogenannte schwache Formulierung bestimmen. Diese hängt im Fall des Transportproblems eng mit der Semidiskretisierung zusammen und lautet mit $ \partial \mathcal{D} = \Gamma_{\text{in}} \dot{\cup} \Gamma_{\text{out}} $ und 
$\rho(x,t) = \rho_{\text{in}}(x,t) \text{ für } (x,t) \in \Gamma_{\text{in}} \times (0,T)$ :
\begin{Definition} 
	$ \rho \in L_1 (\mathcal{D} \times (0,T)) $ heißt schwache Lösung des linearen Transportproblems, falls es für ein gegebenes $ q : \overline{\mathcal{D}} \to \R^2 $ folgende Bedingungen erfüllt:
	\begin{align*}
	\begin{array}{rllll}
	&\text{(swTP)} \ &B(\rho,\phi) &= \langle b,\phi\rangle \quad \forall \phi \in H^1(\mathcal{D}\times\mathbb{T}) \text{ mit } \phi(\cdot,T) = 0 \text{ und } \phi|_{\Gamma_{\text{out}}} = 0 \\
	&\text{Dabei sind :} & & &\\
	& &B(\rho,\phi) &\coloneqq  \int_{0}^{T} \int_{\mathcal{D}} \rho (\partial_t \phi + q \nabla \phi ) \dx \dt - \int_{\Gamma_{\text{out}}} \rho q \cdot n \phi \da  \dt \\
	& &\langle b,\phi \rangle &\coloneqq \int_{\Gamma_{\text{in}}} \rho_{\text{in}} q \cdot n \phi \da  \dt - \int_{\mathcal{D}} \rho_0 \phi(0) \dx \\
	& & \Gamma_{\text{out}} &\coloneqq  \{ z \in \partial \mathcal{D}: q(z)\cdot n(z) > 0 \} & \\
	& & \Gamma_{\text{in}} &\coloneqq  \{ z \in \partial \mathcal{D}: q(z)\cdot n(z) \leq 0 \} &
	\end{array}\\
	%		\text{(swTP)}
	%		\begin{cases}
	%		\begin{array}{rlll}
	%		\displaystyle
	%		\int_{\mathcal{D}} \rho_0 \phi(0) \dx = \mkern-16mu &- \displaystyle \int_{0}^{T} \int_{\mathcal{D}} \rho (\partial_t \phi + q \nabla \phi ) \dx \dt \\
	%		&+\displaystyle\int_{0}^{T}  \int_{\Gamma_{\text{in}}} \rho_{\text{in}} q \cdot n \phi \da  \dt \\
	%		&+\displaystyle\int_{0}^{T}  \int_{\Gamma_{\text{out}}} \rho q \cdot n \phi \da  \dt
	%		\end{array}
	%		\end{cases}	
	\end{align*}
	%		für alle Testfunktionen $ \phi: \mathcal{D} \times (0,T) \to \R $ mit $ \phi(\cdot,T) = 0 $ auf $ \mathcal{D}  $ und $ \phi|_{\Gamma_{\text{out}}} = 0 $.
\end{Definition}
%	dabei ist $  \Gamma_{\text{out}} \coloneqq  \{ z \in \partial \mathcal{D}: q(z)\cdot n(z) > 0 \}$ 
%	und $  \Gamma_{\text{in}} \coloneqq  \{ z \in \partial \mathcal{D}: q(z)\cdot n(z) \leq 0 \} $. \\
%	

Es gilt an dieser Stelle außerdem:

\begin{Lemma}(Zusammenhang der Lösungsbegriffe)
	\begin{enumerate}
		\item Ist $ \rho $ eine klassische Lösung, so ist $ \rho $ auch eine schwache Lösung.
		\item Ist $ \rho \in C^2(\mathcal{D} \times \mathbb{T} , \R )$ und eine schwache Lösung, so ist $ \rho $ auch eine klassische Lösung. 
	\end{enumerate}
\end{Lemma}

\begin{proof}
	Sei $ \phi : \mathcal{D} \times \mathbb{T} \to \R $ eine beliebige Testfunktion aus  $H^1(\mathcal{D} \times \mathbb{T}) $, für die $ \phi(\cdot,T) = 0 $ und $ \phi|_{\Gamma_{\text{out}}} = 0 $ gelte. Wir halten zunächst fest, dass der Raum $ H_0^1(\mathcal{D} \times \mathbb{T}) $ vollständig in dem so betrachteten Testraum enthalten ist.
	Wir beginnen nun mit der Differentialgleichung $ \partial_t \rho(x,t) + \dive(\rho(x,t) q(x)) = 0  $, multiplizieren zunächst mit einer Testfunktion $ \phi $ aus dem Testraum und integrieren anschließend über den Raum-Zeitzylinder $ \mathcal{D} \times \mathbb{T} $:
	\begin{align*}
	\int_{\mathcal{D} \times \mathbb{T}} \big(\partial_t \rho &+ \dive(\rho q ) \big) \phi \; \mathrm{d} (x,t) =  \\
	&\underbrace{\int_{\mathcal{D} \times \mathbb{T}}  \partial_t \rho \phi \; \mathrm{d} (x,t)}_{(1)} +\underbrace{\int_{\mathcal{D} \times \mathbb{T}}  
		\dive(\rho q ) \phi \; \mathrm{d} (x,t)}_{(2)}
	\end{align*}
	Betrachten wir nun zunächst Integral (1), so folgt mit partieller Integration:
	\begin{align*}
	\int_{\mathcal{D} \times \mathbb{T}}  \partial_t \rho &\phi \; \mathrm{d} (x,t) = \int_{\mathcal{D}} \int_{\mathbb{T}} \partial_t \rho \phi \dt \dx \\&= \int_{\mathcal{D}} \big( - \int_{\mathbb{T}} \rho \partial_t \phi \dt + \big[ \rho \phi \big]_0^T  \big) \dx \\
	&= - \int_{\mathcal{D} \times \mathbb{T}} \rho \partial_t \phi \; \mathrm{d} (x,t) + \int_{\mathcal{D}} \underbrace{\rho(x,T) \phi(x,T)}_{ = 0 } - \underbrace{\rho(x,0)}_{ = \rho_0(x) \text{ auf } \mathcal{D}} \phi(x,0)  \dx\\
	&= - \int_{\mathcal{D} \times \mathbb{T}} \rho \partial_t \phi \; \mathrm{d} (x,t) - \int_{\mathcal{D}} \rho_0 \phi(x,0) \dx 
	\end{align*}
	Außerdem können wir Integral (2) mit  \ref{n_pI} wie folgt ausdrücken:
	\begin{align*}
	\int_{\mathcal{D} \times \mathbb{T}}  
	\dive(\rho &q ) \phi\; \mathrm{d} (x,t) = \int_{\mathbb{T}} \int_{\mathcal{D}} \dive(\rho q ) \phi \dx \dt \\
	&= \int_{\mathbb{T}} \big( - \int_{\mathcal{D}} \rho q \nabla \phi \dx + \int_{\partial \mathcal{D}} \rho q \cdot n \ \phi \da \big)  \dt \\
	&= - \int_{\mathcal{D} \times \mathbb{T}} \rho q \nabla \phi) \; \mathrm{d} (x,t) + \int_{\mathbb{T}} \int_{\partial \mathcal{D}} \rho q \cdot n \phi \da \dt 
	\end{align*}
	Wenn $ \rho  $ eine klassische Lösung ist, dann existieren insbesondere $ \partial_t \rho  $ und $ \dive(\rho q) $ und obige Umformungen sind zulässig, d.h. $ \rho $ erfüllt auch die schwache Formulierung. \\
	Ist $ \rho $ hingegen eine schwache Lösung die zusätzlich in $ C^2(\mathcal{D}\times\mathbb{T},\R) $ liegt, so lassen sich alle oben durchgeführten Umformungen auch in die andere Richtung durchführen und wir erhalten: 
	\[
	\int_{\mathcal{D}\times\mathbb{T}} (\partial_t\rho + \dive(\rho q))\phi \; \mathrm{d} (x,t) = 0 \quad \forall \phi \in H_0^1(\mathcal{D}\times\mathbb{T}) 
	\] 
	Dann folgt mit \eqref{testfunktionen}, dass $ \rho $ auch die ursprüngliche Differentialgleichung\\
	$ \partial_t \rho(x,t) + \dive(\rho(x,t) q(x)) = 0  $ erfüllt. Somit ist $ \rho $ also auch klassische Lösung.\\
	
\end{proof}

Auch die Semidiskretisierung können wir in ähnlicher Form, wie eben noch die schwache Formulierung, ausdrücken. Sei dazu
\begin{align*}
B_h(\rho_h,\phi) &\coloneqq \sum_{i=1}^{N} (\partial_t\rho_{h},\phi_h)_{K_i}  - \left( \sum_{i=1}^{N} (\Upsilon(\rho_h),\nabla \phi_h)_{K_i} - \sum_{\substack{F \in \mathcal{F}_{K_i} \\ F \not\subseteq \Gamma_{\text{in}}}} (\Upsilon^{\star}(\rho_h)\cdot n^K,\phi_h)_F\right) \\
\langle b_h , \phi_h \rangle &\coloneqq (\rho_{\text{in}}q \cdot n^K,\phi_h)_{\partial K_i \cap \Gamma_{\text{in}}}
\end{align*}
Dann löst $ \rho_h \in Q_h $ die Semidiskretisierung \eqref{Semidiskretisierung}, wenn $ B_h(\rho_{h},\phi) = \langle l_h,\phi_h \rangle  $ für alle $ \phi_h \in Q_h$ erhalten ist.



\subsubsection{Konsistenz}
	 Abschnitt \ref{Diskretisierung} hat sich damit beschäftigt, aus dem ursprünglich unendlich dimensionalen Problem letztendlich eine volldiskretisierte Verfahrensvorschrift in einem endlichen Ansatzraum herzuleiten. Fragen wir nun nach der Konsistenz des Verfahrens, stellen wir damit zugleich die Frage, ob wir immer noch die richtige Gleichung lösen. Genauer heißt das Verfahren genau dann konsistent, wenn eine analytische Lösung $ \rho $ des ursprünglichen Problems (dTP) auch die hergeleitete Verfahrensvorschrift erfüllt. 
	 Wir betrachten zunächst etwas abstrakter den formalen Prozess der Diskretisierung einer Gleichung \[ B \rho=0 \  \].
	 Dabei werde das abstrakte Problem $ B \rho=0 $ mit einem Operator $ B: E \to F $ mithilfe der Abbildungen $ L_1,L_2,\Phi_h $ diskretisiert.
	 Zu einer Lösung der abstrakten Gleichung $ z \in E $ existiere genau ein $ z_h \in E_h$ und der lokale Diskretisierungsfehler sei gegeben durch $
	 l_h \coloneqq B_h L_1(z) = \phi_h(B) L_1(z) \in F_h
	 $. Diese Situation kann durch folgendes Diagramm verdeutlicht werden:
	 \begin{figure}[H]
	 	\centering
	 	\captionabove{Diskretisierungsprozess}
	 	\includegraphics[width=0.45\textwidth]{abstraktkonsistenz2.png} \\
	 	Abbildung leicht abgeändert aus \cite{brokate2016grundwissen} Seite 390
	 \end{figure}
	 Das diskretisierte Problem $ B_h \rho_h = 0  $ heißt genau dann konsistent, wenn für eine analytische Lösung $ \rho^{\star} \in E $ gilt, dass $ \lim\limits_{h \to 0}\lVert \Phi_h(B)L_1\rho^{\star} - L_2B\rho^{\star} \rVert_{F_h} = 0 $.
	 Betrachten wir zunächst in einem Zwischenschritt die Semidiskretisierung (5.3).
	 Die gewählte Herleitung dieser Formulierung soll dabei nahelegen, dass für eine exakte (und glatte) Lösung $ \rho $ die Konsistenz für die Semidiskretisierung erfüllt ist. In oben eingeführter Notation gilt also gerade $ B_h(\rho^{\star},\phi) $ für alle Testfunktionen $ \phi $. Auch wenn es sich bei $ \mathcal{Q}_h $ um einen nicht konformen Ansatzraum handelt, ist der Herleitung der Semidiskretisierung dennoch zu entnehmen, dass auch $ B_h(\rho^{\star},\phi_h) $ für alle $ \phi_h \in \mathcal{Q}_h $ gilt $(\star)$.  Wichtig ist dabei unter anderem auch die Wahl der numerischen Flussfunktion, der upwind flux erhält aber gerade gewünschten Eigenschaften. 
	 Mehr dazu findet sich in \cite{Har08b}.
	 Für die Zeitdiskretisierung in Form der impliziten Mittelpunktsregel gilt als einstufige Gauß-Quadratur $ \lVert \Phi_h(B)L_1\rho^{\star} - L_2B\rho^{\star} \rVert_{F_h} = O(h^2)$, womit sie insbesondere konsistent ist. \\
	 Daher gilt so für das kombinierte Verfahren, dass eine klassische Lösung $ \rho^{\star} $ somit auch die Volldiskretisierung in obigem Sinne erfüllt. 
\subsubsection{Galerkin Orthogonalität}
Eine direkte Folgerung aus der Konsistenz und $ (\star) $ stellt die Galerkin Orthogonalität dar. Für eine Lösung der Semidiskretisierung $ \rho_h $ und eine analytische Lösung im klassischen Sinne $ \rho^{\star} $ gilt dann nämlich:
\[
 B_h(\rho^{\star}-\rho_h,\phi_h = 0) \quad \forall \phi_h \in \mathcal{Q}_h
\]
\subsubsection{Stabilität und Konvergenz}
Die Stabilität ist bei der numerischen Lösung der Transportgleichung einer der wesentlichen Gründe, weswegen wir das Discontinuous Galerkin Verfahren einem Standard-Finite-Elemente-Ansatz vorziehen. Es zeigt sich nämlich, dass ein normales Finite Elemente Verfahren, wie wir es zuvor bei der Lösung des Potentialströmungsproblems genutzt haben, beim Transportproblem instabil ist. Grund dafür ist, dass $ \lVert q \cdot \nabla \rho_h \rVert $ beliebig groß werden kann. Für die DG-Diskretisierung mit upwind flux kann hingegen gezeigt werden, dass die Lösung des Transportproblems stabil ist. Auf eine theoretische Stabilitätsanalyse möchten wir aber an dieser Stelle verzichten und verweisen z.B. auf \cite{Har08b} oder \cite{ern2004theory}. Wie bereits an früherer Stelle erwähnt findet sich in \cite{Har08b} auch eine grundlegende numerische Analyse des Discontinuous Galerkin Verfahren angewandt auf die stationäre lineare Transportgleichung.
Ebenso wollen wir bezüglich der Konvergenz des Verfahrens auf entsprechende Literatur verweisen. Wir werden später Konvergenzannahmen stellen und diese in entsprechenden Experimenten verifizieren. In der Literatur finden sich aber auch theoretische Konvergenzbeweise, meist unter Verwendung vielfältiger funktionalanalytischer Grundlagen und unter gewissen Regularitätsvoraussetzungen an die bestimmte Lösung. Die theoretische Betrachtung von Discontinuous Galerkinverfahren ist bereits Gegenstand vielfältiger wissenschaftlicher Arbeiten, aber zugleich ein breites Feld, welches noch lange nicht vollständig erforscht ist. \\
%Im nächsten Abschnitt sollen nun schließlich die Überlegungen der letzten Abschnitte gebündelt werden und wir wollen die Multilevel Monte Carlo Methode bei der speziellen Anwendung auf das Transportproblem betrachten.






%%%%%%%%%%%%%%%%%%%%%%%%%%%%%%%%%
\newpage  % neuer Abschnitt auf neue Seite, kann auch entfallen
%%%%%%%%%%%%%%%%%%%%%%%%%%%%%%%%%
\section{Die Multilevel Monte Carlo Methode}
\subsection{Motivation und Beispiel}
\label{MLMC}
% !TeX root = bachelorarbeit.tex

Nachdem wir im dritten Abschnitt die Monte Carlo Methode betrachtet haben, wollen wir uns nun einer Weiterentwicklung der Monte Carlo Methode, der sogenannten Multi Level Monte Carlo Methode zuwenden. Grundsätzlich liegt dieselbe Situation vor wie bei der Monte Carlo Methode:
Wir wollen wieder eine Größe bestimmen, welche sich nach geeigneter Modellierung in der Form eines Erwartungswertes $ \mathbb{E}[X] $ einer Zufallsvariablen $ X $ schreiben lässt. Besonders wenn diese Größe mit der Lösung von gewöhnlichen oder partiellen Differentialgleichungen, wie wir sie später betrachten wollen, hat man nun jedoch die Wahl, wie genau die numerische Lösung des zugrunde liegenden Problems, z.B. der Differentialgleichung, erfolgen soll. Beispielsweise können wir im Falle der numerischen Lösung von Differentialgleichung Zeitschrittweiten und/oder Gitterweiten der Ortsdiskretisierung festlegen. Wir werden dann in diesem Zusammenhang auch von verschiedenen (Genauigkeits-)Leveln sprechen. An dieser Stelle tritt stets ein typischer Zwiespalt auf:
\begin{itemize}
	\item Zu Einen wollen wir möglichst genau rechnen. Dies legt die Wahl von besonders kleinen Zeitschrittweiten bzw. feinen Gittern zur Ortsdiskretisierung nahe.
	\item Zum Anderen wollen wir die Anzahl der Rechenschritte bzw. die Rechenzeit möglichst gering halten. Dies spricht hingegen für große Zeitschritte bzw. grobe Gitter.
\end{itemize}
Zusätzlich zu der oft bereits alleine anspruchsvollen Aufgabe solche Probleme numerisch zu lösen, müssen wir also stets einen für unsere Bedürfnisse passenden Kompromiss aus möglichst genauer numerischer Approximation und geringem (oder zumindest machbarem) Rechenaufwand eingehen. Obwohl dies zunächst wie eine zusätzliche Hürde erscheint und Mehraufwand vermuten lässt, stellt sich heraus, dass solche eine Wahl der Genauigkeit im Kontext von Monte Carlo Methode sich durchaus als Nützlich erweisen kann. 
Die Multi Level Monte Carlo Methode, wir werden im folgenden auch oft vom sogenannten Multi Level Monte Carlo Schätzer sprechen, ist der Prototyp einer Familie sogenannter Varianz-reduzierender Methoden, welche das Ziel haben die naive Monte Carlo Methode in Sachen Konvergenzrate und Effizienz zu schlagen. Bevor wir erklären wie genau die Multilevel Monte Carlo Methode im Allgemeinen dabei vorgeht, möchten wir die Funktionsweise wieder anhand eines Beispiels erklären, welches in \cite{heinrich2001multilevel} ausführlich erklärt wird.

\begin{Beispiel}(Wieder ein Integral über $[0,1]^d$)\\
	Wie bereits im letzten Abschnitt setzen wir uns die Aufgabe das Integral einer Funktion $ f $ zunächst über $ [0,1]^d $ zu bestimmen. Damit wir aber überhaupt in oben erklärte Situation kommen und von verschiedenen 'Leveln' sprechen können, sei $ f $ nun zusätzlich abhängig von einem Parameter $ \lambda \in \Lambda \subseteq \R^{d_2}$, also $f : \Lambda \times [0,1]^d \to \R $. Um bei den folgenden Überlegungen die Notation so schlank wie möglich zu halten, betrachten wir an dieser Stelle nur einen konkreten Spezialfall: \\
	Sei $ d = d_2 = 1 $ und $ f \in C([0,1]^2,\R) $, d.h. wir wollen das Integral 
	\[
		I(\lambda) = \int_{0}^{1} f(\lambda,u) \du
	\]
	für alle $ \lambda \in \Lambda = [0,1] $ bestimmen, wir suchen also nach einer Funktion in Abhängigkeit von $ \lambda $.\\
	
	\textbf{Monte Carlo Schätzer für $I(\lambda)$}\\
	Wollen wir an dieser Stelle einen normalen Monte Carlo Schätzer nutzen, stellt sich die Frage, wie wir mit dem zusätzlichen Parameter umsetzen sollen. Die wohl naheliegendste und einfachste Idee ist, zunächst für ein festes $ h \in \N $ ein Gitter $ \{ \lambda_i = \frac{i}{h}, i=0,\dots,h\} $ festzulegen und für jedes $ \lambda_i $ wie im letzten Abschnitt vorzugehen und für ein $ n \in \N $
	\[
		I(\lambda_i) \approx \hat{I}(\lambda_i) \coloneqq \frac{1}{n} \sum_{k=1}^{n} f(\lambda_i,x_k)
	\]
	zu schätzen. Dabei seien wieder $ (x_k)_{k=1,\dots,n} $ Realisierungen von unabhängigen auf $ [0,1] $ gleichverteilten Zufallsvariablen $ (X_k)_{k=1,\dots,n} $.
	Anschließend lässt sich aus den so ermittelten Werten durch Interpolation einen Schätzer für die gesamte Funktion $ I(\lambda) $ bestimmen. Grundsätzlich sind verschiedene Interpolationsansätze möglich. Für dieses grundlegende Beispiel wählen wir stückweise lineare Interpolation. Wir erhalten so für alle $ \lambda \in \Lambda $:
	\[
		I(\lambda) \approx (PI)(\lambda) = \sum_{i=0}^{h} \hat{I}(\lambda_i) \varphi_i(\lambda)
	\]
	mit $ \varphi_i \coloneqq \mathds{1}_{ \{\abs{\lambda - \lambda_i} \leq h \} }(1-h\abs{\lambda-\lambda_i})$. Ein solcher Interpolationsansatz lässt sich insbesondere auf mehrdimensionale Gitter übertragen.
%	Alternativ $ \alpha_i \coloneqq \frac{I(\lambda_{i+1}) - I(\lambda_i)}{\lambda_{i+1}-\lambda_i} $ und $ \varphi_i(\lambda) \coloneqq \mathds{1}_{\lambda \in [\lambda_i,\lambda_{i+1}]} \left(  \alpha_i\lambda + (I(\lambda_i) - \alpha_i \lambda_i) \right)$.
	Somit erhalten wir für $ I(\lambda) $:
	\[
		I(\lambda) \approx \mathcal{I}_{MC}(\lambda) \coloneqq \sum_{i=0}^{h} \left( \frac{1}{n}\sum_{k=1}^{n} f(\lambda_i, x_k)\right) \varphi_i (\lambda) = \frac{1}{n} \sum_{k=1}^{n} (Pf(\cdot,x_k))(\lambda)
	\]
	Als Fehler dieser Methode können wir den sogenannten 'root mean square error' verbunden mit einer beliebigen Norm betrachten, wir wählen hierbei die $ L^2 $-Norm.
	Wir erhalten so 
	\[ 
	\epsilon(\mathcal{I}_{MC})  = \left( \mathbb{E} [\lVert I -  \mathcal{I}_{MC} \rVert_{L^2([0,1])}^2] \right)^{\frac{1}{2}} = \left( \mathbb{E} \left[ \int\limits_{0}^{1} \abs{I(\lambda) - \mathcal{I}_{MC}(\lambda)}^2 \dlam \right] \right)^{\frac{1}{2}}
	\]
	Ist $ f $ zusätzlich stetig differenzierbar im Parameter $ \lambda $, kann gezeigt werden, dass 
	\[
		\epsilon(\mathcal{I}_{MC}) = \mathcal{O}(n^{-\frac{1}{2}}+h^{-1}) \ .
	\].
	Gleichzeitig ist die Anzahl der arithmetischen Operationen, Funktionsaufrufe und generierter Zufallszahlen in $ \mathcal{O}(hn) $.
	Wir sehen also, dass wir an dieser Stelle genau diesen Zwiespalt antreffen, welchen wir zuvor abstrakt beschrieben haben. Aus diesem Grund wollen wir nun einen Multilevel Monte Carlo Schätzer für $ I(\lambda) $ einführen.\\	
	
	\textbf{Multilevel Monte Carlo Schätzer für $ I(\lambda) $}\\
	Wir betrachten nun eine Familie von Gittern $ \{ \lambda_{li} = \frac{i}{h_l} : h_l = 2^l \ , i=0,1,\dots h_l \} $ für $ l = 0,\dots,m $.
	Analog zu oben führen wir zugehörige Interpolationsoperatoren 
	\[
	 (P_l I)(\lambda) = \sum_{i=0}^{h_l} \hat{I}(\lambda_{li}) \varphi_{li} \quad (l = 0,\dots,m)
	 \]
	 ein. Wir können nun also insbesondere $ P \coloneqq P_m $ also Teleskopsumme darstellen. Es gilt nämlich:
	 \[
	 	P = P_m = P_0 + \sum_{l=1}^{m} (P_l-P_{l-1}) \ .
	 \]
	 Der Monte Carlo Schätzer von oben lässt sich (mit $ P_{-1} \coloneqq 0 $)dann durch 
	 \[	
	 \mathcal{I}_{MC} = \sum\limits_{l=0}^{m} \frac{1}{n} \sum\limits_{k=1}^{n} (P_l-P_{l-1})f(\cdot,x_k)	
	 \]
	  umschreiben. Um nun tatsächlich einen Nutzen aus der Aufteilung in verschiedene Level zu ziehen und einen guten Kompromiss zwischen Kosten und Fehler herzustellen erlauben wir nun zusätzlich die Anzahl der Zufallsauswertungen $ n $ von Level zu Level zu variieren. 
	  Wir wählen also $ (n_l)_{l=0,\dots,m} \in \N^{m+1}  $.  Außerdem seien $ \{ X_{lj} , l=0,\dots,m \ , j= 0,\dots,n_l\} $ unabhängige auf $ [0,1] $ gleichverteilte Zufallsvariablen und $ (x_{lj})_{l=0,\dots,m \ ,j=0,\dots,n_l} $ zugehörige Realisierungen.
	  Dann erhalten wir den Multilevel Monte Carlo Schätzer 
	  \[
	   I(\lambda) \approx \mathcal{I}_{MLMC}(\lambda) = \sum_{l=0}^{m} \frac{1}{n_l} \sum_{k=1}^{n_l} ((P_l - P_{l-1}) f(\cdot,x_{lj}))(\lambda) \ .
	  \]
	  Der bedeutendste Schritt ist an dieser Stelle eine passende Wahl der $ n_l $. Bei diesem Beispiel wollen wir uns darauf beschränken eine passende Wahl anzugeben und den Nutzen hervorzuheben, welchen wir durch diese Wahl erlangen. So zeigt sich, dass eine passende Wahl beispielsweise durch $ n_l = \Theta(2^{-\frac{3l}{2}n})$ für ein $ n \in \N $ groß genug gegeben ist. 
	  Dann kann für den analog wie für den MC-Schätzer definierten (RMSE-)Fehler gezeigt werden, dass 
	  \[
	  	\epsilon(\mathcal{I}_{MLMC}) = \mathcal{O}(n^{-\frac{1}{2}} + n^{-\frac{1}{2}}) = \mathcal{O}(n^{-\frac{1}{2}})
	  \].
	  Zugleich ist die Anzahl der benötigten Rechenoperationen inklusive Funktions- und Zufallszahlauswertungen diesmal in $ \mathcal{O}(n) $.
	  Verglichen mit der Standard (Ein-Level) Monte Carlo Methode können wir nun also eine Approximation für die gesamte Familie von Integralen $ I(\lambda) $ mit einem Fehler von $ \mathcal{O}(n^{-\frac{1}{2}}) $, aber den Kosten von $ \mathcal{O}(n) $  berechnen. Das ist durchaus erstaunlich, denn bereits die Kosten der Auswertung eines einzigen Integrals $ I(\lambda) $ für ein festes $ \lambda \in \Lambda $ liegen in $ \mathcal{O}(n) $. 
\end{Beispiel}

Wir sehen also, dass die Multilevel Monte Carlo Methode in Situationen, in denen wir bei der Wahl von Zeitschrittweiten und/oder feinen Gittern zur Ortsdiskretisierung zwischen Anzahl an Rechenoperationen und Genauigkeit einen Kompromiss finden müssen, einen Ein-Level Ansatz, wie die Standard Monte Carlo Methode, durchaus übertreffen kann.
Der Kern dieser Methode bildet dabei eine geschickte Wahl der Anzahl $ n_l $ der Zufallssamples, welche wir auf je einem Level auswerten. Wie wir in unserem Fall diese Wahl durchführen soll an anderer Stelle in Abschnitt \ref{MLMCTP} ausführlich erläutert werden, in welchem wir die bisher zunächst beispielhaft anhand der Integration eingeführte Multilevel Monte Carlo Methode auf das probabilistische Transportproblem, welches wir in Abschnitt \ref{TP} bereits näher beleuchtet haben, übertragen werden.
Mehr zu Monte Carlo und Multilevel Monte Carlo Methoden für Parameterintegrale findet sich neben \cite{heinrich2001multilevel} auch in \cite{heinrich1992random}.

\subsection{Konvergenz und Genauigkeit}
Da wir in Abschnitt \ref{MLMCTP} noch einmal ausführlich auf die Eigenschaften des Verfahrens für unsere konkrete Anwendung eingehen werden, soll dieser Unterabschnitt eher noch einmal etwas allgemeiner auf die stochastischen Hintergünde eingehen.
%%%%%%%%%%%%%%%%%%%%%%%%%%%%%%%%%
\newpage  % neuer Abschnitt auf neue Seite, kann auch entfallen
%%%%%%%%%%%%%%%%%%%%%%%%%%%%%%%%%
\subsection{Anwendung der Multilevel Monte Carlo Methode auf das Transportproblem}
\label{MLMCTP}
% !TeX root = bachelorarbeit.tex
Nachdem wir in Abschnitt 2 an einige Grundlagen erinnert, in Abschnitt 3 und 4 sowohl die Monte Carlo Methode, als auch die Multilevel Monte Carlo Methode eingeführt und in Abschnitt 5 das Transportproblem, sowie das Potentialströmungsproblem inklusive numerischer Verfahren, welche zur Lösung dergleichen genutzt werden können, erklärt haben, soll nun dieser Abschnitt dazu dienen, die bisherigen Ergebnisse zu bündeln und die Anwendung der Multilevel Monte Carlo Methode auf partielle Differentialgleichungen am Beispiel des Transportproblem nahe zu legen. Dabei nimmt dieser Abschnitt einen zentralen Platz in dieser Thesis ein, weswegen wir noch einmal darlegen wollen, was genau unser Ziel ist und wie wir die uns an dieser Stelle zu Verfügung stehenden Mittel einsetzen, um das Gewünschte zu erreichen.
\begin{align*}
&\text{Für ein stochastisches Flussvektorfeld } q: \Omega \times \overline{\mathbb{D}} \to \R^2 \text{, bestimme }\rho: \Omega \times \overline{\mathbb{D}} \times \mathbb{T} \to \R_{\geq 0} \text{ mit} \\
&\text{(pTP)} 
\begin{cases}
\begin{array}{rlll}
\partial_t \rho (\omega, x, t) + \dive(\rho(\omega,x,t)q(\omega,x)) &= 0 &\text{, für } (x,t) \in \mathbb{D} \times (0,T] \\
\rho(\omega,x,t) &= \rho_{\text{in}}(x,t) &\text{, für } (x,t) \in \Gamma_{\text{in}} \times \mathbb{T} \\
\rho(\omega,x,0)  &= \rho_0(x) &\text{, für } x \in  \mathbb{D}
\end{array}
\end{cases} \\
&\text{ für die Anfangs- und Randwerte: } \\ 
&\begin{array}{llr}
g_N&: \Gamma_{\text{N}} \to \R \\
u_D&: \Gamma_{\text{D}} \to \R \\
\rho_{\text{in}}&: \Gamma_{\text{in}} \times \mathbb{T} \to \R_{\geq0} \\
\rho_0&: \mathbb{D} \to \R_{\geq0} \\
\end{array} \newline \\
&\text{ wobei } \partial \mathbb{D} = \Gamma_{\text{D}} \dot{\cup} \Gamma_{\text{N}}  \text{ und }  \Gamma_{\text{in}} \coloneqq  \{ z \in \partial \mathbb{D}: q(z)\cdot n(z) \leq 0 \} \subset  \partial \mathbb{D}
\end{align*}
Genauer sei $ Q(\omega) = J(\rho(\omega)) $ ein gegebenes Zielfunktional, dann ist es unser Ziel den Erwartungswert $ \mathbb{E}[Q(\omega)] $ möglichst genau zu bestimmen. Beispielsweise kann eine beliebige Norm von $ \rho(\omega) $ als Zielfunktional betrachtet werden.
Unser Modellproblem lautet also:

\begin{align}
\text{(MP)}
\begin{cases}
\label{Modellproblem}
&\text{Für ein stochastisches Flussvektorfeld } q: \Omega \times \overline{\mathbb{D}} \to \R^2 \\
&\text{und ein Zielfunktional } J \text{ , bestimme }  \mathbb{E}[J(\rho)]  \\
&\text{mit } \rho \text{ als Lösung von (pTP) inklusive Anfangs- und Randwerten}
\end{cases}
\end{align} 
Dabei erhalten wir das stochastische Flussvektorfeld $ q $, wie bereits in \ref{Probabilistisches Problem} erklärt selbst als Lösung des Potentialströmungsproblem.
Da wir an der numerischen Lösung partieller Differentialgleichungen interessiert sind und wir die im Allgemeinen unendlich dimensionale Lösung $ \rho $ durch eine endlich dimensionale Lösung $ \rho_{h,\Delta t} $ approximieren, betrachten wir $ Q_{h,\Delta t}(\omega) \coloneqq J(\rho_{h,\Delta t}(\omega )) $.
Wir nutzen dabei das in Abschnitt \ref{DG} behandelte Discontinuous Galerkin Verfahren zur Lösung des linearen Transportproblems. Außerdem werden wir im Folgenden eine uniforme Familie $ \{ \mathcal{T}_h \} $ von Zerlegungen von $ \mathcal{D} $ (vgl. Abschnitt \ref{num_pot} Definition \ref{FEMDISC}) als Diskretisierungsgitter für die Ortsdiskretisierung, sowie $ \mathbb{T}_{\Delta t} $ als Zerlegung von $ \mathbb{T} $ betrachten.
Insbesondere wählen wir später $ \Delta t $ in Abhängigkeit von $ h $, etwa $ \Delta t = c  h $ für ein $ c>0 $ und betrachten dann
$ Q_h(\omega) \coloneqq Q_{h,ch}(\omega) = J(\rho_h(\omega )) \coloneqq J(\rho_{h,ch}(\omega )) $.
An dieser Stelle sei betont, dass es sich in diesem Abschnitt bei $ \rho_h $ um eine volldiskrtisierte Approximation an $ \rho $ handelt und nicht die Semidiskretisierung aus Abschnitt \ref{DG} gemeint ist.
\begin{Annahme}(Konvergenz im Erwartungswert des Zielfunktionals)\\
	In obiger Situation gelte 
	\[ 
	\mathbb{E}[Q_h] \to \mathbb{E}[Q] \text{ für } h \to 0   
	\]
\end{Annahme}


\subsection{Die Monte Carlo Methode}
Wie bereits im Kontext der numerischen Integration wollen wir die Monte Carlo Methode als Ausgangspunkt nutzen und anschließend bei der Betrachtung der Multilevel Monte Carlo Methode auch auf entscheidende Unterschiede zu und Vorteile gegenüber der Monte Carlo Methode eingehen.
Sowohl bei der Monte Carlo Methode, als auch bei der Multilevel Variante approximieren wir den Erwartungswert $ \mathbb{E}[Q_h] $ durch einen Schätzwert $ \widehat{Q}_h $. Um die Genauigkeit und die Kosten zu bemessen betrachten wir zum einen den sogenannten 'root mean square error' (RMSE) 
\begin{align}
	\label{RMSE}
	e( \widehat{Q}_h) \coloneqq \left(  \mathbb{E} \left[ (\widehat{Q}_h - \mathbb{E}[Q] )^2 \right] \right)^{\frac{1}{2}}
\end{align}
zum anderen die Anzahl an floating-point-Rechenoperationen $ C_{\epsilon}(\widehat{Q}_h) $, die benötigt werden um einen RMSE mit $ e( \widehat{Q}_h) \leq \epsilon $ zu erhalten.
Zu beachten ist hierbei, dass in dem RMSE einige Fehlerquellen gemeinsam betrachtet werden. So gehen sowohl der Finite Elemente Fehler des Potentialströmungsproblems (Approximation von $ q $ durch $ q_h $), der Discontinuous Galerkin Fehler des Transportproblems (Approximation von $ \rho $ durch $ \rho_h $), der Approximationsfehler der Approximation von $ Q $ durch $ Q_h $, als auch der statistische Fehler des Schätzers $\widehat{Q}_h$ in den RMSE mit ein. Insbesondere bedeutet also $ e( \widehat{Q}_h) \leq \epsilon $, dass alle oben genannten Fehlerquellen kleiner als $ \epsilon $ ausfallen. Betrachten wir also, wie bei der Monte Carlo Methode, welche wir gleich noch einmal kurz behandeln wollen, nur eine einzige Zerlegung $ \mathcal{T} $ der Familie $ \mathcal{T}_h $, so kann es sein, dass es gar nicht möglich ist den RMSE durch ein bestimmtes $ \epsilon $ zu beschränken, da einer der Approximationsfehler für das gewählte feste Level bereits alleine größer als $ \epsilon $ ist. Wir wollen diesen Sachverhalt im Hinterkopf behalten, denn auch bei der Multilevel Monte Carlo  Methode werden wir später initiale Level wählen, in der konkreten Anwendung geben wir aus praktischen Gründen sogar ein maximales Level an, welches wiederum unter Umständen einen RMSE $ e( \widehat{Q}_h) \leq \epsilon $ verhindern kann.\\
Bei der Standard Monte Carlo Methode schätzen wir $ \mathbb{E}[Q] $ durch den Mittelwert  $ n $ unabhängiger gleichverteilter Zufallssamples und erhalten so 
\begin{align}
	\label{MC-Schätzer}
	\widehat{Q}_{h,n}^{\text{MC}} \coloneqq \frac{1}{n} \sum_{i=1}^{n} Q_h(\omega_i) = \sum_{i=1}^{n} J(\rho_h(\omega_i))
\end{align}
Dabei modellieren wir das zufällige Flussvektorfeld $ q(\omega_i)  : \mathbb{D} \to \R^2 $, indem wir zunächst für $ \kappa(\omega_i) : \mathbb{D} \rightarrow (\R_{\text{sym}})^{d \times d} $ ein lognormal-verteiltes unabhängiges Zufallsfeld erzeugen und anschließend das Potentialströmungsproblem
\begin{align*}
\setlength\arraycolsep{1pt}
&\text{Für } \kappa(\omega_i) : \mathbb{D} \rightarrow (\R_{\text{sym}})^{d \times d} \text{, bestimme } u(\omega_i,\cdot):\overline{\mathbb{D}} \to \R \text{ und } q(\omega_i, \cdot): \overline{\mathbb{D}} \to \R^2 \text{ mit } \\
&\text{(PS)}
\begin{cases}
\begin{array}{rlll}
\dive (q(\omega_i,x)) &= 0  &\text{, für } x \in \mathbb{ D}\\  
q(\omega_i,x) &= - \kappa(\omega_i) \nabla u(\omega_i,x)  &\text{, für } x \in \mathbb{D}\\
-q(\omega_i,x) \cdot n &= g_N(x)  &\text{, für } x \in \Gamma_N \\
u(\omega_i,x) &= u_D(x)  &\text{, für } x \in \Gamma_D \\
\end{array}
\end{cases} \\
\end{align*}
lösen. Zur Erzeugung des lognormal-verteilten Zufallsfeldes können wir auf entsprechende Algorithmen zurückgreifen, in unserem Fall etwa dem sogenannten Circulant Embedding.
Der Algorithmus wurde 1997 erstmals in \cite{dietrich1997fast} vorgestellt und 
erzeugt Gauß'sche Zufallsfelder auf regulären Gittern und basiert auf der Fast Fourier Transformation, welche in der Literatur auch oft unter der Abkürzung FFT zu finden ist.
Dabei werden spezielle Strukturen der Kovarianzmatrix ausgenutzt. Anschließend kann das Gauß'sche Zufallsfeld über eine einfache Transformation in ein lognormal Feld überführt werden. Mehr zu Circulant Embedding findet sich z.B. in \cite{schmidt2014stochastic} Abschnitt 12.
Auch die grundsätzliche Idee Circulant Embedding für die Modellierung der Ausgangsdaten in stochastischen partiellen Differentialgleichungen zu nutzen ist keineswegs neu und findet sich z.B. in \cite{charrier2012strong} oder \cite{cliffe2011multilevel}.
Wir haben nun alle Mittel in der Hand um die Monte Carlo Methode angewandt auf das Transportproblem als Algorithmus zu formulieren. Dabei fassen wir die eben erklärte Erzeugung des zufälligen Vektorfeldes in der Funktion 'RndVecField' zusammen:

\begin{algorithm}[H]
	\DontPrintSemicolon
	\SetAlgoLined
	%\KwResult{}
	\SetKwInOut{Input}{Input}\SetKwInOut{Output}{Output}
	\Input{$h,n$}
	\Output{$\widehat{Q}_{h,n}^{\text{MC}}$}
	\BlankLine
	Initialisiere: $ \Sigma =0, i=0 $\;
	\While{$i<n$}{
		Erzeuge ein Zufallssample: $ q(\omega_i,x)  \leftarrow $ RndVecField\;
		Löse das Transportproblem: $ \rho_{h}(\omega_i,x)  \leftarrow$ Löse mit Discontinuous Galerkin\;
		Berechne das zugehörige Zielfuinktional: $ Q_h(\omega_i) \leftarrow$ Berechne Zielfunktional\;
		Setze: $ \Sigma = \Sigma + Q_h(\omega_i) $, i = i+1\;
	}
	\BlankLine
	\KwResult{$\widehat{Q}_{h,n}^{\text{MC}} = \Sigma /n$}
	\caption{Monte Carlo Methode angewandt auf das Transportproblem}
\end{algorithm}

\bigskip % add 12pt space in-between


\begin{itemize}
	\item Theorie Konvergenzannahmen
	\item Resultate Giles + parrallelFEM Paper
	\item Algorithmus
\end{itemize}

\begin{algorithm}[H]
	\DontPrintSemicolon
	\SetAlgoLined
	%\KwResult{}
	\SetKwInOut{Input}{Input}\SetKwInOut{Output}{Output}
	\Input{h,n}
	\Output{$\widehat{Q}_{h,n}^{\text{MC}}$}
	\BlankLine
	
	\While{While condition}{
		instructions\;
		\eIf{condition}{
			instructions1\;
			instructions2\;  
		}{
			instructions3\;
		}
	}
	\BlankLine
	\KwResult{$\widehat{Q}_{h,n}^{\text{MC}} = \sum/n$}
	\caption{While loop with If/Else condition}
\end{algorithm}

\bigskip % add 12pt space in-between


%%%%%%%%%%%%%%%%%%%%%%%%%%%%%%%%%
\newpage  % neuer Abschnitt auf neue Seite, kann auch entfallen
%%%%%%%%%%%%%%%%%%%%%%%%%%%%%%%%%
\section{Beispiel/Experiment}
\subsection{Konkretes Problem}

\subsection{Ergebnisse}
%%%%%%%%%%%%%%%%%%%%%%%%%%%%%%%%%
\newpage  % neuer Abschnitt auf neue Seite, kann auch entfallen
%%%%%%%%%%%%%%%%%%%%%%%%%%%%%%%%%
\section{Ausblick und Fazit}

\section{Appendix}
% !TeX root = bachelorarbeit.tex
\begin{Satz}
	Sei $ (\Omega,\mathcal{A},\mathbb{P}) $ ein Wahrscheinlichkeitsraum und $ X = (X_1,\dots,X_n) $ ein Zufallsvektor mit (nicht entarteter) multivariater Normalverteilung mit Parametern $ \mu = (\mu_1,\dots,\mu_n) \in \R^n $ und $ C = (\sigma_{ij})_{1 \leq i,j \leq n} \in \R^{n \times n} $.\\
	Dann ist $ \mathbb{E}[X] = \mu $ und für alle $ i,j \in {1,\dots,n}  $ gelten:
	\[
		 X_j \sim N(\mu_j,\sigma_{jj}) \text{ und } \sigma_{ij} = \text{Cov}(X_i,X_j)
	\]
\end{Satz}
\begin{proof}(fasst mehrere Resultate aus \cite{brokate2016grundwissen} zusammen)
	
	Da $ C $ symmetrisch positiv definit ist, existiert ein invertierbares $ A \in \R^{n \times n} $ mit $ C = AA^{\top} $ (Cholesky-Zerlegung).
	Weiter sei $ Y = (Y_1,\dots,Y_n)^{\top} $ ein Zufallsvektor, wobei die einzelnen $ Y_1,\dots,Y_n $ unabhängige und je $ N(0,1) $-verteilte Zufallsvariablen sind. 
	Durch $ T(x) \coloneqq Ax + \mu $ erhalten wir somit für $ x \in \R^k $ eine stetig differenzierbare Abbildung die den $ \R^k $ auf sich selbst abbildet und die Funktionaldeterminante $ \det A$ besitzt.
	Ist $ Y $ nun ein $ n $-dimensionaler Zufallsvektor mit Dichte $ f $, so besitzt der Zufallsvektor $ Z \coloneqq  AY + \mu$ nach dem Transformationssatz die Dichte
	\[
		g(y) = \frac{f(A^{-1}(y-\mu))}{\abs{\det A}}, \quad y \in \R^k .
	\]
	Wir erhalten also mit 
	\begin{align*}
		f(x) = \prod_{j=1}^n \left( \frac{1}{\sqrt{2\pi}}\exp\left(-\frac{x_j^2}{2}\right) \right) = \frac{1}{(2\pi)^{\frac{n}{2}}} \exp \left( - \frac{x^{\top}x}{2}\right), \text{ für } x \in \R^n \\
		g(y) = \frac{1}{(2\pi)^{\frac{n}{2}}\abs{\det A}} \exp \left( - \frac{1}{2} \left( A^{-1}(y-\mu)\right)^{\top}\left( A^{-1}(y-\mu)\right)\right), \text{ für }y \in \R^n .
	\end{align*}
	Wegen $ C = A A^{\top} $, $ (A^{-1})^{\top} = (A^{\top})^{-1} $ und $ \abs{\det A} = \sqrt{\det C} $ ist somit 
	\[
		g(y) = \frac{1}{(2\pi)^{\frac{n}{2}}\sqrt{\det C}} \exp \left( - \frac{1}{2} (y-\mu)^{\top}C^{-1} (y-\mu)\right), \text{ für }y \in \R^n.
	\]. 
	Insbesondere ist also $ Z \sim X \sim N_n(\mu,C) $.\\
	Seien nun $ A = (a_{ij})_{1 \leq i,j \leq n} $, dann folgt
	\[
		X_j \sim \sum_{l=1}^n a_{jl} Y_l + \mu_j \ .
	\]
	Wegen $ K_l \coloneqq a_{ij} Y_l  \sim N(0,a_{jl}^2) $ und der Unabhängigkeit der $ Y_l $ (mit dem sogennanten Blockungslemma folgt somit Unabhängigkeit der $ K_l $) gilt nach dem Additionsgesetz für die Normalverteilung 
	\[
		X_j\sim N \left( \mu_j,\sum_{l=1}^{n}a_{jl}^2 \right).
	\]
	Aus $ C = A A^{\top}  $ folgt schließlich $ \sigma_{jj} = \sum_{l=1}^{n} a_{jl}^2 $.
	Es bleibt nun also noch zu zeigen, dass $ \mathbb{E}[X] = \mu $ und $ \sigma_{ij} = \text{Cov}(X_i,X_j) $.
	Wir bezeichnen mit $ \text{Cov}(X) \coloneqq (\text{Cov}(X_i,X_j))_{1 \leq i,j \leq n} $
	die Kovarianzmatrix.
	Es ist $ \mathbb{E}[Y] = 0 $ und  $\text{Cov}(Y) = I_n $.
	Es gilt also:
	\begin{align*}
		\mathbb{E}[X] = \mathbb{E}[AY + \mu] = (\mathbb{E}[\sum_{l=1}^{n}K_l+\mu_j]) =(\mathbb{E}[\sum_{l=1}^{n}a_{jl}Y_l+\mu_j])= A \mathbb{E}[Y] + \mu = \mu \\
		\text{Cov}(X) = \text{Cov}(AY + \mu) =  \text{Cov}(AY) = A \text{Cov}(Y) A^{\top} = AA^{\top} = C
	\end{align*}
	
	
	
	
	
	
\end{proof}
% !TeX root = bachelorarbeit.tex
\subsection{Referenzzelle und Hybridisierung}
\label{Referenzzelle&Hyb}
\subsubsection{Referenzzelle}
Bevor wie uns an dieser Stelle der Hybridisierung widmen können, wollen wir noch kurz auf einen wichtigen Aspekt der Implementierung finiter Elemente eingehen.
An dieser Stelle hat es sich nämlich bereits oft als nützlich erwiesen, eine sogenannte Referenzzelle einzuführen. Statt sich die Daten jeder Zelle statisch zu speichern und dann darauf zuzugreifen, gehen wir dabei stets von der Referenzzelle aus und können über je eine linear affine Abbildung in den tatsächlichen Zellen operieren. Da wir ausschließlich Vierecke verwenden, wollen wir uns dementsprechend an dieser Stelle auf Vierecke beschränken.

\begin{Definition}
	
	Das Referenzviereck $ \square $ ist definiert als
	\[ \hat{K} \coloneqq \conv \{ \hat{\mathcal{V}} \} \text{, wobei } \hat{\mathcal{V}} \coloneqq \left\{ 
	\begin{pmatrix}
	0\\
	0
	\end{pmatrix},
	\begin{pmatrix}
	1\\
	0
	\end{pmatrix},
	\begin{pmatrix}
	0\\
	1
	\end{pmatrix},
		\begin{pmatrix}
	1\\
	1
	\end{pmatrix} \right\} \]
	Die Seiten von $ \square $ sind
	\begin{align*}
	\hat{F}_0 &\coloneqq \conv \left\{ 
	\begin{pmatrix}
	0\\
	0
	\end{pmatrix},
	\begin{pmatrix}
	1\\
	0
	\end{pmatrix} \right\}\\
	\hat{F}_1 &\coloneqq \conv \left\{ 
	\begin{pmatrix}
	1\\
	0
	\end{pmatrix},
	\begin{pmatrix}
	1\\
	1
	\end{pmatrix} \right\}\\
	\hat{F}_2 &\coloneqq \conv \left\{ 
	\begin{pmatrix}
	1\\
	1
	\end{pmatrix},
	\begin{pmatrix}
	0\\
	1
	\end{pmatrix}
	 \right\}\\ 
	 \hat{F}_3 &\coloneqq \conv \left\{ 
	 \begin{pmatrix}
	 0\\
	 1
	 \end{pmatrix},
	 \begin{pmatrix}
	 0\\
	 0	
	 \end{pmatrix}
	 \right\}\\ 	
	\end{align*}
	Weiter sei $ \{ \hat{\psi}_i \}_{i=0}^3 $ die Seitenbasis aus Hütchenfunktionen
%	\begin{align*}
%	\hat{\psi}_0:\hat{K} \to \R^2, \hat{\psi}_0(\xi) &\coloneqq 
%	\begin{pmatrix}
%	\xi_1\\
%	\xi_2-1
%	\end{pmatrix}\\
%	\hat{\psi}_1:\hat{K} \to \R^2, \hat{\psi}_1(\xi) &\coloneqq  
%	\begin{pmatrix}
%	\xi_1\\
%	\xi_2
%	\end{pmatrix}\\
%	\hat{\psi}_2:\hat{K} \to \R^2, \hat{\psi}_2(\xi) &\coloneqq 
%	\begin{pmatrix}
%	\xi_1-1\\
%	\xi_2
%	\end{pmatrix}\\
%	\hat{\psi}_3:\hat{K} \to \R^2, \hat{\psi}_3(\xi) &\coloneqq 
%	\begin{pmatrix}
%	\xi_1-1\\
%	\xi_2
%	\end{pmatrix}.
%	\end{align*}
	und $ \hat{\nu} $ sei der äußere Normalenvektor von $ \hat{K} $.
\end{Definition}

\begin{figure}[H]
	\centering
	\captionabove{Referenzzelle}
	\includegraphics[width=0.33\textwidth]{referenzzelle2.png} \\
	Abbildung modifiziert aus \cite{knabner2013numerik} Seite 51
\end{figure}

\begin{Bemerkung}
	$\forall i,j\in\{0,1,2,3\} \colon \int_{F_j} \hat{\psi}_i \cdot \hat{\nu} \da = \delta_{i,j} \text{ und } \hat{\psi}_i \in \mathbb{P}_1(\hat{K}, \R^2).$
\end{Bemerkung}

Weiter setzen wir noch 
\begin{align*}
\text{ die Menge der Seiten }&& \hat{\mathcal{F}} &\coloneqq \left\{ \hat{F}_0, \hat{F}_1, \hat{F}_2,\hat{F}_3 \right\} \\
\text{und den Seitenansatzraum }&&  \hat{W} &\coloneqq \spann \left\{ \psi_0,\psi_1, \psi_2, \psi_3 \right\}.
\end{align*}
\textbf{Transformation von $ \hat{K} $ zu $ K $:} Für ein beliebiges $ K \in \mathcal{K} $ wollen wir jetzt eine Seitenbasis $ \{ \psi_0^K, \psi_1^K, \psi_2^K, \psi_3^K \} $ berechnen (Wie bisher gegeben durch $ \forall i\in\{0,1,2,3\}\colon \psi_i^K\in\mathbb{P}_1(K,\R^2) \text{ und } \int_{F_j^K} \psi_i^K \cdot \nu^K \da = \delta_{i,j} $, wobei $ \nu^K $ äußere Normale von  $ K $ und $ F_j^K  $ beliebige Seite von $ K $). Dazu betrachten wir die affine Transformationsabbildung $ \varphi_K $ von $ \hat{K} $ zu $ K $:
\begin{align*}
\varphi_K\colon \hat{K} \to K, \varphi_K(\xi) = z_{0,K} + B_K \xi \text{ mit passenden } B_K\in\R^{2 \times 2} \text{ und } \\
J_K \coloneqq \det(B_K) > 0.
\end{align*}

\begin{figure}[H]
	\centering
	\captionabove{Affine lineare Transformation von $\hat{K}$ nach $ K $}
	\includegraphics[width=0.55\textwidth]{affinlinearetransf2.png} \\
	Abbildung modifiziert aus \cite{knabner2013numerik} Seite 53
\end{figure} 
\begin{Lemma}
	Es gilt: $\nu^K = \frac{1}{\abs{B_K^{-T} \hat{\nu}}} B_K^{-T} \hat{\nu} $ ist Normale zu $ \partial K $.
\end{Lemma} $\newline$
Die Seitenbasis auf $ K $ ist dann gegeben durch 
\[ \psi_i^K = J_K^{-1}B_K \hat{\psi}_i \circ \varphi_K \ (i\in \{0,1,2,3\}) \]
Die globale Seitenbasis $\{\psi_j\}_{j = 1}^{\abs{\mathcal{F}}}$ auf $ \mathcal{D}  $ erhalten wir dann mithilfe einer weiteren Abbildung $l$, die zwischen der Seitennummerierung in einer Zelle $ K $ und der globalen Seitennummerierung vermittelt. Es ist dabei
\[ l\colon \mathcal{K}\times \{0,1,2,3\} \to \{1, \dots , \abs{\mathcal{F}} \}, (K,i) \mapsto l(K,i). \]
Wir setzen nun also $ \psi_j (j \in \{1,\dots , \abs{\mathcal{F}}\})$ durch
\[ \psi_j(x) = 
\begin{cases}
\psi_i^K(x) , &\text{falls } j = l(K,i)\\
0,			  &\text{sonst}.
\end{cases} \]
\begin{Bemerkung}
	Für alle Zellen $ K \in \mathcal{K} $ von denen $ F_j $ eine anliegende Seite ($ \overline{K} \cap F_j \ne \emptyset $) und $ F_j $ lokal mit $ i \in \{0,1,2,3\} $ nummeriert ist, gilt:
	\[ \psi_j|_K = \psi_i^K. \]
\end{Bemerkung}

\subsubsection{Hybridisierung}

Wir betrachten die Räume
\[W_K \coloneqq \left\{ \psi_K\colon K \to \R^2 \colon \psi_K = J_K^{-1}  B_K \hat{\psi} \circ \varphi_K^{-1}, \hat{\psi} \in \hat{W} \right\} \]
\[ W_\mathcal{K} \coloneqq \prod_{K \in \mathcal{K}} W_K, \qquad M_h \coloneqq \prod_{F \in \mathcal{F}} \mathbb{P}_0(F) \]
\[ M_h(u_D) \coloneqq \left\{ \mu_h \in M_h \colon \forall  F \subset \Gamma_D \int_F \mu_h \da = \int_F u_D \da  \right\}\]

An dieser Stelle nutzen wir die folgende Äquivalenz:\\
	\[ \psi_h \in W_h \iff \big[ \psi_h \in W_{\mathcal{K}} \text{ und }  (\psi_{K_1} - \psi_{K_2}) \cdot \nu^F = 0 \ (F= \partial K_1 \cap \partial K_2 \in \mathcal{F}^{\circ})\big] \]



Und untersuchen folgendes Problem:
\begin{align*}
\text{Bestimmme } (q_h,u_h, \lambda_h) \in W_\mathcal{K} \times \mathcal{Q}_h \times M_h(u_D) \text{ mit}\\
\begin{dcases}
(1) \int_K \kappa^{-1} q_h \psi_K \dx - \int_K u_h \dive(\psi_K) \dx = - \int_{\partial K} \lambda_h \psi_K \cdot \nu^K \da \\
(2) \int_K \dive(q_h) \phi_K \dx = 0\\
(3) \sum_{K\in \mathcal{K}} \int_{\partial K} q_h \cdot \nu \mu_h \da = - \int_{\Gamma_N} g_N \mu_h \da
\end{dcases}\\
\text{für alle } K \in \mathcal{K}, \psi_K \in W_K, \phi_K \in \mathcal{Q}_h  \text{ und } \mu_h \in M_h(0)
\end{align*}

Dieses Problem ist äquivalent zu dem diskreten gemischten FE-Problem, welches wir zuvor betrachtet haben:
\begin{align*}
\text{Bestimme } (q_h,u_h) \in W_h(-g_N) \times \mathcal{Q}_h \text{ mit}\\
\begin{cases}
\int_{\Omega} \kappa ^{-1} q_h \cdot \psi_h \dx \mkern-15mu &- \int_{\Omega} u_h \dive(\psi_h) \dx = - \int_{\Gamma_D} u_D \psi_h \cdot \nu \da\\
&- \int_{\Omega} \dive(q_h) \phi_h \dx = 0
\end{cases} \\
\text{für alle } (\psi_h, \phi_h) \in W_h(0) \times \mathcal{Q}_h
\end{align*}

Für ein festes $ K \in \mathcal{K} $ ergibt sich mit der Wahl einer Basis von $ W_K $, $ \mathcal{Q}_h $ und $ M_h $ 
%\begin{align*}
%	&W_K = \spann\{\psi_F \mid F \in \mathcal{F}_K \}  &&(\psi_F \cdot n^F)_{|F'} = \begin{cases}
%		1 &F = F'\\
%		0 &\text{sonst}
%	\end{cases}\\
%	&Q_h = \spann\{\eta_K \mid K \in \mathcal{K}  \}  &&{\eta_K}_{|K'} = \begin{cases}
%	1 &, K = K'\\
%	0 &,\text{sonst}
%	\end{cases}\\
%	&M_h = \spann \{ \nu_F \mid F \in \mathcal{F}_K \}  &&{\nu_F}_{|F'} = \begin{cases}
%		1 &, F = F'\\
%		0 &, \text{sonst}
%	\end{cases}  
%\end{align*}

eine Formulierung als LGS mit Nebenbedingung, wobei $ \ubar{q}_K \coloneqq \ubar{R}_K \ubar{q}, \ \ubar{u}_K \coloneqq \ubar{R}_K \ubar{u} $ .
\begin{align*}
\text{Bestimme } \ubar{q}, \ubar{u} \text{ und } \ubar{\lambda} \text{ mit}\\
\begin{dcases}
(1) \begin{pmatrix}
\ubar{A}_K & \ubar{B}_K \\
\ubar{B}_K^T & 0
\end{pmatrix}
\begin{pmatrix}
\ubar{q}_K\\
\ubar{u}_K
\end{pmatrix}
\mkern-16mu&= \begin{pmatrix}
- \ubar{C}_K \, \ubar{R}_K \ubar{\lambda} \\
0
\end{pmatrix}\\
(2) \sum_{K \in \mathcal{K}}  (\ubar{R}_K \ubar{\mu})^T \ubar{C}_K \ubar{q}_K \mkern-16mu&= \ubar{\mu}^T \ubar{b}		
\end{dcases}\\
\text{für alle } \ubar{\mu} \text{ mit } \ubar{\mu}[F] = 0 \text{ für } F \in \Gamma_D \cap \mathcal{F}
\end{align*}

\begin{align*}
\left(
\begin{array}{c}
1\\ 1 \\ \vdots \\ 1 \\ 1\\ \ubar{\mu}
\end{array}
\right)^T
\underbrace{\left(
	\begin{array}{ccccc|c}
	\ubar{A}_{K_1} & \ubar{B}_{K_1}&&&& \ubar{C}_{K_1} \ubar{R}_{K_1}\\
	\ubar{B}_{K_1} & 0 &&&& 0\\
	&& \ubar{A}_{K_2} & \ubar{B}_{K_2} && \ubar{C}_{K_2} \ubar{R}_{K_2}\\
	&  & \ubar{B}_{K_2} & 0 && 0\\
	&  &  &  &\ddots &\\
	\hline
	\ubar{R}_{K_1}^T \ubar{C}_{K_1}^T& 0 & \ubar{R}_{K_2}^T \ubar{C}_{K_2}^T& 0& &0 \\
	\end{array}
	\right)}_{\eqqcolon \left( \begin{array}{c|c}
	\ubar{D} & \ubar{E}\\
	\hline
	\ubar{E}^T & 0
	\end{array} \right)}
\underbrace{\left(
	\begin{array}{c}
	\ubar{q}_{K_1}\\ \ubar{u}_{K_1} \\ \ubar{q}_{K_2}\\ \ubar{u}_{K_2} \\ \vdots \\ \ubar{\lambda}
	\end{array}
	\right)}_{\eqqcolon \left( \begin{array}{c}
	\left(
	\begin{array}{c}
	\ubar{q}_{K_i}\\ \ubar{u}_{K_i}
	\end{array}
	\right)_{K_i \in \mathcal{K}} \\ \ubar{\lambda} 
	\end{array}\right)} = 
\left(
\begin{array}{c}
0\\ 0 \\ \vdots \\ 0 \\ 0\\ \ubar{\mu}^T \ubar{b}
\end{array}
\right)
\end{align*}
Mit dem Schurkomplement $ \ubar{S} \coloneqq \ubar{E}^T \ubar{D}^{-1} \ubar{E} $ folgt
\[ \ubar{\mu}^T \ubar{S} \, \ubar{\lambda} = \ubar{\mu}^T \ubar{b} 	\text{ für alle } \ubar{\mu} \text{ mit } \ubar{\mu}[F] = 0 \text{ für } F \in \Gamma_D \cap \mathcal{F} \]

Sobald wir $ \ubar{\lambda}_k \coloneqq \ubar{R}_K \ubar{\lambda}$ bestimmt haben, können wir auch das obere LGS (1) lösen, um $ \ubar{q}_K $ und $ \ubar{u}_K $ zu erhalten.
% Um $ \ubar{\lambda}_k $ zu erhalten lösen wir das globales System:
%\begin{align*}
%	\underline{S} \, \underline{\lambda} = \underline{b}
%\end{align*}
%und bekommen somit $ \ubar{\lambda}_k $ über die Restriktionen
%\[ \ubar{\lambda} = \sum_{K \in \mathcal{K}} \ubar{R}_K \ubar{\lambda}_K .\]


  % Literaturverzeichnis (beginnt auf einer ungeraden Seite)
  \newpage

%\begin{thebibliography}{Lam00}
 %Bibiographie
  \bibliographystyle{abbrv}
  \bibliography{References}
%\end{thebibliography}
 
      
  % ggf. hier Tabelle mit Symbolen 
  % (kann auch auf das Inhaltsverzeichnis folgen)

\newpage
  
 \thispagestyle{empty}


\vspace*{8cm}


\section*{Erkl\"arung}

Ich  versichere  wahrheitsgem\"a\ss,  die  Arbeit selbstst\"andig verfasst,  alle  benutzten  Hilfsmittel  vollst\"andig  und  genau  angegeben  und  alles kenntlich  gemacht  zu  haben,  was  aus  Arbeiten  anderer  unver\"andert  oder  mit  Ab\"anderungen entnommen  wurde,  sowie die Satzung  des  KIT  zur  Sicherung guter wissenschaftlicher Praxis in der jeweils g\"ultigen Fassung beachtet zu haben.
\\[2ex] 

\noindent
Ort, den Datum\\[5ex]

% Unterschrift (handgeschrieben)



\end{document}


% Vorlage für eine Bachelorarbeit
% Siehe auch LaTeX-Kurs von Mathematik-Online
% www.mathematik-online.org/kurse
% Anpassungen für die Fakultät für Mathematik
% am KIT durch Klaus Spitzmüller und Roland Schnaubelt Dezember 2011

\documentclass[12pt,a4paper]{scrartcl}
% scrartcl ist eine abgeleitete Artikel-Klasse im Koma-Skript
% zur Kontrolle des Umbruchs Klassenoption draft verwenden


% die folgenden Packete erlauben den Gebrauch von Umlauten und ß
% in der Latex Datei
\usepackage[utf8]{inputenc}
% \usepackage[latin1]{inputenc} %  Alternativ unter Windows
\usepackage[T1]{fontenc}
\usepackage[ngerman]{babel}


\usepackage[pdftex]{graphicx}
\usepackage{latexsym}
\usepackage{amsmath,amssymb,amsthm}
\usepackage{setspace}
\usepackage{url}
\usepackage{mathtools}


% Abstand obere Blattkante zur Kopfzeile ist 2.54cm - 15mm
\setlength{\topmargin}{-15mm}


% Umgebungen für Definitionen, Sätze, usw.
% Es werden Sätze, Definitionen etc innerhalb einer Section mit
% 1.1, 1.2 etc durchnummeriert, ebenso die Gleichungen mit (1.1), (1.2) ..
\newtheorem{Satz}{Satz}[section]
\newtheorem{Definition}[Satz]{Definition} 
\newtheorem{Lemma}[Satz]{Lemma}		   
                  
\numberwithin{equation}{section} 

\newtheorem*{Bemerkung}{Bemerkung}

% einige Abkuerzungen
\newcommand{\C}{\mathbb{C}} % komplexe
\newcommand{\K}{\mathbb{K}} % komplexe
\newcommand{\R}{\mathbb{R}} % reelle
\newcommand{\Q}{\mathbb{Q}} % rationale
\newcommand{\Z}{\mathbb{Z}} % ganze
\newcommand{\N}{\mathbb{N}} % natuerliche

% eigene Definitionen
\newcommand{\dx}{\, \mathrm{d} x}
\DeclareMathOperator{\dive}{div}



\begin{document}
  % Keine Seitenzahlen im Vorspann
  \pagestyle{empty}
  
  
  % Titelblatt der Arbeit
  \begin{titlepage}

    \includegraphics[scale=0.45]{kit-logo.jpg} 
    \vspace*{2cm} 

 \begin{center} \large 
    
    Bachelorarbeit
    \vspace*{2cm}

    {\huge Die Multilevel Monte Carlo Methode und deren Anwendung am Beispiel der linearen Transportgleichung}
    \vspace*{2.5cm}

    Tim Buchholz
    \vspace*{1.5cm}

    ??.??.??
    \vspace*{4.5cm}


    Betreuung: Prof.Dr. Christian Wieners und M.Sc. Niklas Baumgarten \\[1cm]
    Fakultät für Mathematik \\[1cm]
		Karlsruher Institut für Technologie
  \end{center}
\end{titlepage}



  % Inhaltsverzeichnis
  \tableofcontents

\newpage
 


  % Ab sofort Seitenzahlen in der Kopfzeile anzeigen
  \pagestyle{headings}

\section{Einleitung}
% !TeX root = bachelorarbeit.tex

TODO(Einleitung wird zu einem späterem Zeitpunkt noch ausgebaut und nachgebessert mehr cites mehr forschung mehr inhalt)
Monte Carlo Methoden sind weit verbreitet und finden in verschiedenen Bereichen der Mathematik ihre Anwendung.
Sie dienen dabei als statistische Schätzer für Erwartungswerte. 
Eine der bekanntesten Anwendungen ist wohl die Monte Carlo Quadratur, welche zur numerischen Integration genutzt werden kann.
 
Nachdem Giles (cite ...) ... gewöhnliche DGL ... kam ... für SPDE's zu nutzen ...cite .

So entstehende Problemstellungen fallen in das Gebiet der Uncertainty Quantification, einem 'Zusammentreffen der Wahrscheinlichkeitstheorie, Numerik, Statistik und der echten Welt' \cite{sullivan2015introduction}.
Allerdings besitzt die Monte Carlo Methode einen entscheidenden Nachteil, will man sie im Zusammenhang unsicherer Ausgangsdaten für die Lösung von partiellen Differentialgleichungen nutzen, sie konvergiert im Normalfall relativ langsam und das numerische Lösen von PDE's ist oft sehr aufwendig.
Es werden also unter Umständen sehr viele, sehr teure Zufallssamples benötigt, um ein vernünftiges Ergebnis zu erhalten. \newline
Diese Thesis soll sich daher mit der Multilevel Monte Carlo Methode (im Folgenden MLMC Methode genannt) beschäftigen, welche an die Monte Carlo Methode angelehnt ist, aber durch die geschickte Auswertung der (Zufalls-Samples) deutliche Effizienzvorteile gegenüber der Standard Monte Carlo Methode besitzt.
Die MLMC Methode soll nach einer ausführlichen theoretischen Analyse auch praktisch auf das Transportproblem angewandt werden.
Genauer soll für
\begin{itemize}
	\item ein beschränktes Gebiet $\mathbb{D} \subseteq \R^d$
	\item  ein Zeitintervall $\mathbb{T} = [0,T]$
	\item  ein Wahrscheinlichkeitsraum $(\Omega,\mathcal{A},\mathbb{P})$
	\item  ein zufälliges Flussvektorfeld $q: \Omega \times \overline{\mathbb{D}} \rightarrow \R^d$
	\item  eine Anfangskonzentration eines (zu transportierenden) Stoffes $\rho_0: \overline{\mathbb{D}} \rightarrow \R^d$
	\item einen Einfluss $\rho_{\text{in}} : \Gamma_{\text{in}} \times \mathbb{T} \rightarrow \R$ über den Einflussrand $\Gamma_{\text{in}} \coloneqq  \{ z \in \partial \mathbb{D}: q(z)\cdot n(z) \leq 0 \} \subset  \partial \mathbb{D}$ mit $n(z)$ als äußeren Normalenvektor im (Rand-)Punkt $z$
\end{itemize}
der Erwartungswert eines Funktionals der  Konzentration des Stoffes $\rho: \overline{\mathbb{D}} \times \mathbb{T}  \rightarrow \R_{\geq0}$ bestimmt werden. Dabei erhält man $\rho$ als Lösung der folgenden partiellen Differentialgleichung:
\begin{gather*}
\text{Bestimme } \rho: \overline{\mathbb{D}} \times \mathbb{T} \to \R_{\geq 0} \text{, sodass}\\
(\text{TP})
\begin{cases}
\partial_t \rho + \dive(\rho q) = 0 &\text{ in } \mathbb{D} \times (0,T)\\
\rho(x,t) = \rho_{\text{in}}(x,t) &\text{ auf } \Gamma_{\text{in}} \times (0,T)\\
\rho(x,0) = \rho_0(x) &\text{ auf } \mathbb{D}.
\end{cases}
\end{gather*}
Außerdem muss zunächst ein zwar zufälliges, aber dennoch sinnvolles Vektorfeld $q$ erzeugt werden. Wir nutzen hierbei das Darcy-Gesetz, welches als Modellierung von Fluiden in porösen Bodenschichten bereits oft genutzt wurde (vgl. z.B. \cite{de1986quantitative}).
Dabei soll später, bevor wir das eigentliche Transportproblem lösen, stets zunächst für einen zufälligen Permeabilitätstensor, welcher die unbekannte Bodenbeschaffenheit modellieren soll, ein entsprechendes Flussvektorfeld $q$ über das sogenannte Potentialströmungsproblem, welches sich aus dem Darcy-Gesetz ableitet, berechnet werden. 
Die genauere Modellierung des so entstehenden Gesamtproblems soll aber an späterer Stelle erfolgen. \newline
Die Thesis ist dazu folgendermaßen unterteilt:\newline 
Abschnitt 2 sammelt verschiedene Grundlagen aus den Bereichen der Stochastik, der Analysis und Numerik partieller Differentialgleichungen. Besonders werden wir hierbei auf einige zentrale Aussagen der Wahrscheinlichkeitstheorie eingehen, welche für die Konvergenzanalyse von Monte Carlo Methoden im Allgemeinen eine wichtige Rolle spielen. 
In Abschnitt 3 betrachten wir einige Aspekte der (standard) Monte Carlo Methode, welche auch der MLMC Methode als theoretischer Unterbau dienen sollen. Dabei erklären wir die Monte Carlo Methoden zunächst anhand des Beispiels der numerischen Integration, gehen dann aber auch abstrakter auf Konvergenz und Genauigkeit der Methode ein.  \newline
Anschließend werden wir in Abschnitt 4 die Multilevel Monte Carlo Methode an sich erklären.
Dazu greifen wir das Beispiel der numerischen Integration aus Abschnitt 3 in einer etwas abgewandelten Form wieder auf. Auch hier wollen wir dann aber auch etwas abstrakter Eigenschaften der Methode betrachten, welche uns auch später bei der Anwendung auf das Transportproblem wieder beschäftigen werden. \newline
In Abschnitt 5 werden dann das Transportproblem und das Potentialströmungsproblem beschrieben, welches wir lösen müssen, um an die entsprechenden Ausgangsdaten zu kommen. Anschließend wird die numerische Lösung der beiden Probleme mit Finite Elemente Methoden behandelt, bevor schließlich in Abschnitt 6 auf die Anwendung der Multilevel Monte Carlo Methode auf das Transportproblem mit unsicheren Ausgangsdaten am Beispiel der Permeabilität $\kappa$ eingegangen wird. \newline
Der siebte und letzte Abschnitt befasst sich mitder  konkreten Durchführung und Implementierung des zuvor theoretisch beleuchteten Problem innerhalb der parallelen Finite Elemente Softwarebibliothek "M++" \cite{siteM++},
welche am Institut für Angewandte und Numerische Mathematik 3 (KIT) von Herrn Prof. Dr. C. Wieners entwickelt wurde. \newline
Am Schluss der Thesis steht eine kleine Zusammenfassung der bis dahin erarbeiteten Resultate und der Ausblick auf Möglichkeiten verschiedener Art an, diese weiter zu entwickeln.

%%%%%%%%%%%%%%%%%%%%%%%%%%%%%%%%%
 \newpage  % neuer Abschnitt auf neue Seite, kann auch entfallen
%%%%%%%%%%%%%%%%%%%%%%%%%%%%%%%%%
 
\section{Grundlagen}
\subsection{analytische/numerische Grundlagen}
% !TeX root = bachelorarbeit.tex
Sei $\mathbb{D} \subseteq \R^d$ für $d\in \N$
\begin{Definition}(Einige Operatoren)
	\begin{enumerate}
		\item Für $F: \R^d \to \R^d$ ist die \underline{Divergenz} von F definiert durch
			\begin{align*}
				\dive F = \nabla \cdot F \coloneqq \sum_{i=1}^{d} \frac{\partial F_i}{ \partial x_i}
			\end{align*}
		\item Für $f: \R^d \to \R$ und $\alpha = (\alpha_1,\dots,\alpha_d) \in \N^d$ ist die partielle Ableitung von $f$ nach dem Multiindex $\alpha$ definiert durch
			\begin{align*}
				D^{\alpha}f \coloneqq 
				\frac{\partial^{|\alpha|} f}{\partial x_1 ^{\alpha_1} \cdots  \partial x_d^{\alpha_d} } 
				=\frac{\partial^{\alpha_1+\dots +\alpha_d} f}{\partial x_1 ^{\alpha_1} \cdots  \partial x_d^{\alpha_d} } 
			\end{align*}
	\end{enumerate}
\end{Definition}
\begin{Definition}(Einige Funktionenräume)
	todo C,C1,Ccinf,Lp, L1loc
\end{Definition}
\begin{Definition}(schwache Ableitung)\\
	Sei $u \in L_{\text{loc}}^1$. Wir sagen v besitzt eine schwache Ableitung zum Multiindex $\alpha$, falls eine Funktion $v \in L_{\text{loc}}^1$ existiert, mit 
	\begin{align*}
		\int_{\mathbb{D}} u D^{\alpha} \Phi \dx = (-1)^{|\alpha|} \int_{\mathbb{D}} v \Phi \dx \qquad \forall \Phi \in C_0^{\infty}(\mathbb{D})
	\end{align*}
	In diesem Zusammenhang nennen wir $\Phi$ auch Testfunktion und wir definieren $D^{\alpha} u \coloneqq v$ als die schwache Ableitung von $u$ zum Multiindex $\alpha$.
\end{Definition}
\begin{Definition}(Sobolevräume)
	todo RWP 
\end{Definition}
\begin{Satz} (Satz von Gauß)
	todo RWP/FEM
\end{Satz}
\begin{Satz} (Rechnen mit Differentialoperatoren)
	todo FEM
\end{Satz}
\subsection{stochastische Grundlagen}
\subsection{Monte Carlo Methoden}
%%%%%%%%%%%%%%%%%%%%%%%%%%%%%%%%%
\newpage  % neuer Abschnitt auf neue Seite, kann auch entfallen
%%%%%%%%%%%%%%%%%%%%%%%%%%%%%%%%%
\section{Multilevel Monte Carlo Methode (MLMC)}
%%%%%%%%%%%%%%%%%%%%%%%%%%%%%%%%%
\newpage  % neuer Abschnitt auf neue Seite, kann auch entfallen
%%%%%%%%%%%%%%%%%%%%%%%%%%%%%%%%%
\section{MLMC angewandt auf das Transportproblem}
\subsection{Problemstellung}
 % !TeX root = bachelorarbeit.tex
\subsubsection{deterministisches Problem}
\label{det_prob}
Sei $\mathbb{T} = [0,T]$ ein Zeitintervall für $T>0$ und $\mathbb{D} \subset \R^d , d \in \N$ ein beschränktes, offenes und konvexes Lipschitz-Gebiet mit Rand $ \partial \mathbb{D} = \Gamma_{\text{D}}  \dot{\cup} \Gamma_{\text{N}} $. 
Wie bereits in der Einleitung beschrieben, wollen wir den Transport eines Stoffes in einer porösen Bodenschicht auf Grundlage eines vorhandenen Flusses beschreiben. 
Als modellhaftes Problem soll uns hierfür die Regenwasserversickerung dienen: In einer porösen Bodenschicht befindet sich zum Zeitpunkt $t=0$ ein Stoff (beispielsweise Öl) in einer gegebenen Anfangskonzentration und -verteilung. Nun sickert Regenwasser in diese poröse Bodenschicht ein. Zusätzlich wollen wir weitere Zuflüsse des Fremdstoffes über den Einflussrand $\Gamma_{\text{in}} \subset \partial \mathbb{D}$ zulassen.
Wir sind letztendlich an der Konzentration dieser Substanz an einer Stelle $x \in \overline{\mathbb{D}}$ zu einem Zeitpunkt $t \in \mathbb{T}$ interessiert. \newline
Bevor allerdings die Konzentration als Lösung des Transportproblems bestimmt werden kann, muss zunächst das Flussvektorfeld $q: \overline{\mathbb{D}} \rightarrow \R^d$ berechnet werden. \newline
Sei hierfür $p: D \rightarrow \R$ der hydrostatische Druck, $\kappa: D \rightarrow (\R_{\text{sym}})^{d \times d}$ der Permeabilitätstensor und $G=(0,0,p_0 g_0)^{\top}$. 
Wie bereits in der Einleitung angedeutet, kann der Fluss des Regenwassers durch das Darcy-Gesetz $q=-\kappa(\nabla p + G)$ modelliert werden.
Durch $u(x) := p(x) + p_0 g_0 x_3$ vereinfacht sich das Darcy-Gesetz zu $q=-\kappa \nabla u$.\newline
Nehmen wir die physikalische Annahme hinzu, dass der Fluss $q$ 'quellfrei' sein soll, also an keiner Stelle Masse verschwinden oder erscheinen kann, erhalten wir das Potentialströmungsproblem:
\[ \text{Bestimme } u:\overline{\mathbb{D}} \to \R \text{ und } q: \overline{\mathbb{D}} \to \R^2 \text{ mit } \newline \]
\[\setlength\arraycolsep{1pt}
\text{(PS)}\begin{cases} 
\begin{array}{rlc}
\dive q     &= 0                 &\text{ ,} \text{in } \mathbb{D}\\
q           &= - \kappa \nabla u &\text{ ,} \text{in }\mathbb{D}\\
u           &= u_D               &\text{ ,} \text{auf } \Gamma_D \\
-q \cdot n  &= g_N               &\text{ ,} \text{auf } \Gamma_N 
\end{array}
\end{cases} \\
\]
\begin{Bemerkung}
	Wir wollen aus verschiedenen Gründen direkt die sogenannte gemischte Formulierung des Potentialströmungsproblem nutzen. Näheres dazu findet sich im nächsten Abschnitt.
\end{Bemerkung}
Anschließend suchen wir die Dichteverteilung $\rho: \mathbb{D} \times \mathbb{T} \rightarrow \R_{\geq0} $ einer transportierten Substanz (in unserem Modell das Öl).  \newline
Gegeben sei dazu die Anfangsverteilung $\rho_0: \mathbb{D} \rightarrow \R_{\geq0}$ und der Einfluss der Substanz über die Zeit
$
\rho_{\text{in}} : \Gamma_{\text{in}} \times \mathbb{T} \rightarrow \R_{\geq0} \text{ mit }  \Gamma_{\text{in}} \coloneqq  \{ z \in \partial \mathbb{D}: q(z)\cdot n(z) \leq 0 \} \subset  \partial \mathbb{D}
$.
Dabei ist $n(z)$ der äußere Normalenvektor im (Rand-)Punkt $z$.
Wir bedienen uns wieder der Physik und fordern die Erfüllung der Bilanzgleichung
\begin{align*}
\forall K \subseteq \mathbb{D} , t\in \mathbb{T} : \frac{d}{dt} \int_K \rho(x,t) \, \mathrm{d}x + \int_{\partial K} \rho(x,t)q(x)\cdot n(x) \, \mathrm{d}a = 0.
\end{align*}
Wenden wir für ein zulässiges $K \subseteq \mathbb{D}$ und $\rho,q \in C^1(\mathbb{D})$ den Satz von Gauß an erhalten wir
\begin{align*}
\int_K \partial_t \rho(x,t) + \dive (\rho q)(x,t) \, \mathrm{d}x = 0
\end{align*}
und können so die (lineare) Transportgleichung ableiten:
\begin{align*}
\partial_t \rho + \dive (\rho q) = 0 \text{ in } \mathbb{D} \times (0,T]
\end{align*}
Mit den entsprechenden Rand- und Anfangswerten erhalten wir so:

\[ 
\text{Bestimme } \rho: \overline{\mathbb{D}} \times \mathbb{T} \to \R_{\geq 0} \text{, sodass} \newline \]
\[\setlength\arraycolsep{1pt}
\text{(TP)}\begin{cases} 
\begin{array}{rlll}
\partial_t \rho + \dive(\rho q) &= 0 &\text{ ,in } &\mathbb{D} \times (0,T)\\
\rho(x,t) &= \rho_{\text{in}}(x,t) &\text{ ,auf } &\Gamma_{\text{in}} \times (0,T)\\
\rho(x,0) &= \rho_0(x) &\text{ ,auf } &\mathbb{D} \\
\end{array}
\end{cases} \\
\]

\subsubsection{probabilistisches Problem}
In dem letzten Unterabschnitt sind wir bereits bei der Lösung des Potentialströmungsproblems davon ausgegangen, sämtliche benötigten Randwerte sowie den Permeabilitätstensor $\kappa$ exakt für das gesamte Gebiet $\mathbb{D}$ zu kennen.
Wir wollen uns von dieser durchaus starken Annahme lösen und deshalb zusätzlich die Permeabilität $\kappa$ mit Mitteln der Stochastik modellieren.
Sei dazu $(\Omega, \mathcal{A},\mathbb{P})$ ein Wahrscheinlichkeitsraum und ab nun $d=2$, also $\mathbb{D} \subseteq \R^2$.
\begin{Bemerkung}
	Grundsätzlich funktionieren die vorgestellten Verfahren auch für $d=3$, wir wollen uns aber der Anschaulichkeit halber auf zwei Dimensionen beschränken. Das so betrachtete Gebiet $\mathbb{D}$ lässt sich so z.B. als Querschnitt einer Bodenschicht interpretieren.
\end{Bemerkung} 
Weiter sei nun $\kappa (\cdot,x): \Omega \rightarrow \R_{\geq0}$ die (vom Zufall abhängige) Permeabilität.
Wie schon an anderer Stelle (z.B. in \cite{kumar2018multigrid}) wollen wir die Permeabilität als lognormal-Feld modellieren.
Unser so entstehendes Problem fällt somit in den Bereich der Uncertainty Quantification und lautet: 


\begin{align*}
\setlength\arraycolsep{1pt}
&\text{Für } \omega \in \Omega \text{, bestimme } u(\omega,\cdot):\overline{\mathbb{D}} \to \R \text{ und } q(\omega, \cdot): \overline{\mathbb{D}} \to \R^2 \text{ mit } \\
&\text{(PS)}
	\begin{cases}
		\begin{array}{rlll}
		\dive (q(\omega,x)) &= 0  &\text{, für } x \in \mathbb{ D}\\  
		q(\omega,x) &= - \kappa(\omega) \nabla u(\omega,x)  &\text{, für } x \in \mathbb{D}\\
		-q(\omega,x) \cdot n &= g_N(x)  &\text{, für } x \in \Gamma_N \\
		u(\omega,x) &= u_D(x)  &\text{, für } x \in \Gamma_D \\
		\end{array}
	\end{cases} \\
&\text{Für } \omega \in \Omega \text{ und }q(\omega,\cdot): \overline{\mathbb{D}} \to \R^2 \text{, bestimme }\rho(\omega,\cdot): \overline{\mathbb{D}} \times \mathbb{T} \to \R_{\geq 0} \text{ mit} \\
&\text{(TP)} 
	\begin{cases}
		\begin{array}{rlll}
			\partial_t \rho (\omega, x, t) + \dive(\rho(\omega,x,t)q(\omega,x)) &= 0 &\text{, für } (x,t) \in \mathbb{D} \times (0,T] \\
			\rho(\omega,x,t) &= \rho_{\text{in}}(x,t) &\text{, für } (x,t) \in \Gamma_{\text{in}} \times \mathbb{T} \\
			\rho(\omega,x,0)  &= \rho_0(x) &\text{, für } x \in  \mathbb{D}
		\end{array}
	\end{cases} \\
&\text{ für die Anfangs- und Randwerte: } \\ 
	&\begin{array}{llr}
		g_N&: \Gamma_{\text{N}} \to \R \\
		u_D&: \Gamma_{\text{D}} \to \R \\
	    \rho_{\text{in}}&: \Gamma_{\text{in}} \times \mathbb{T} \to \R_{\geq0} \\
		\rho_0&: \mathbb{D} \to \R_{\geq0} \\
	\end{array} \newline \\
&\text{ wobei } \partial \mathbb{D} = \Gamma_{\text{D}} \dot{\cup} \Gamma_{\text{N}}  \text{ und }  \Gamma_{\text{in}} \coloneqq  \{ z \in \partial \mathbb{D}: q(z)\cdot n(z) \leq 0 \} \subset  \partial \mathbb{D}
\end{align*}
Dabei stellen wir uns die Aufgabe, den Erwartungswert eines gegebenen Zielfunktionals $Q(\rho)$ zu berechnen, etwa dem Ausfluss der transportierten Substanz über den Rand. An dieser Stelle können wir dann, nachdem wir uns in den nächsten zwei Unterabschnitten damit beschäftigt haben, wie wir obige Probleme numerisch lösen, die MLMC Methode nutzen, um diesen Erwartungswert zu berechnen. 

\subsection{Numerische Lösung des Potentialströmungsproblem}
% !TeX root = bachelorarbeit.tex
\begin{Bemerkung}
	Die beiden folgenden Abschnitte bauen im Wesentlichen auf den beiden Vorlesungen 'Einführung in das Wissenschaftliche Rechnen' (SS 2019) und 'Finite Elemente Methoden' (WS 2019/2020) von Herrn Prof. Dr. Wieners auf. Dem entsprechend sind als Quellen neben \cite{brenner2007mathematical} und 
	\cite{braess2013finite} vor allem die Mitschriebe zu den oben genannten Vorlesungen, sowie die Berichte zum Rechnerpraktikum mit M++ \cite{siteM++} zu nennen.
\end{Bemerkung}
Wie bereits in obigem Abschnitt erwähnt, sollen sich die nächsten beiden Abschnitte damit beschäftigen, wie wir die oben beschriebenen Probleme für ein festes $\omega \in \Omega$ numerisch lösen können. 
Ein Überblick über alle möglichen Verfahren, welche zur Lösung der beiden Probleme geeignet sind, würde den Rahmen dieser Thesis sprengen. Wir wollen deshalb im Folgenden auf eine Möglichkeit eingehen diese Berechnung numerisch durchzuführen. Insbesondere werden dabei jene Verfahren beschrieben, welche wir auch später innerhalb der MLMC Methode in M++ nutzen wollen.
Da wir in diesen beiden Abschnitte $\omega \in \Omega$ ohnehin fest halten, genügt es zudem das deterministische Problem zu betrachten. \newline
Sowohl das hybride Finite Elemente Verfahren, welches wir zur Lösung des Potentialströmungsproblem nutzen wollen, als auch das Discontinuous Galerkin Vefahren, mit dessen Hilfe wir das Transportproblem lösen wollen, bauen auf der Finite Elemente Theorie auf. 
Diese ist im Wesentlichen in der zweiten Hälfte des 20. Jahrhunderts entstanden, ist aber bis heute in praktischer wie auch in theoretischer Sicht aktuell.
Die Grundidee ist hierbei die vorliegenden Rand-Anfangswertaufgaben in einem passenden endlichen Unterraum zu lösen. Dabei löst man sich auf analytischer Seite zunächst oft von einzelnen Regularitäts- und Differenzierbarkeitsbedingungen und führt einen sogenannten schwachen Lösungsbegriff ein (vergleiche Abschnitt 2.1). Statt nun aber solch eine schwache Lösung in einem unendlich dimensionalen Funktionenraum, wie beispielsweise in den Sobolevräumen $W^{1,2}(\mathbb{D})$ oder $W_0^{1,2}(\mathbb{D})$ zu bestimmen, zieht man sich auf endliche Unterräume zurück. \newline
Die folgende Definition entstammt \cite{brenner2007mathematical} und geht ursprünglich (1978) auf Ciarlet zurück.
\begin{Definition}\
	Sei
	\begin{itemize}
		\item $K \subseteq \R^d$ eine beschränkte abgeschlossene Menge mit einem nichtleeren Inneren und stückweise stetig differenzierbarem Rand 
		\item $\mathcal{P}$ ein endlich dimensionaler Funktionenraum auf K
		\item $\mathcal{N} = \{N_1,N_2,\dots,N_k \}$ eine Basis für $\mathcal{P}^{'}$
	\end{itemize}
	Dann heißt $(K,\mathcal{P},\mathcal{N})$ ein finites Element.
\end{Definition}

Wir wollen im Folgenden diese theoretische Definition zwar im Hinterkopf behalten, aber wie in \cite{braess2013finite} meist nur mit den sogenannten Finite-Elemente-Räumen arbeiten. 
Dabei wird eine geeignete Zerlegung $\mathfrak{K} = \{K_1,K_2,\dots, K_M \}$ von $\mathbb{D}$ in endlich viele Teilgebiete gewählt. 
Anschließend betrachten wir einen endlichen Raum von Funktionen, die eingeschränkt auf diese Teilgebiete von einfacher Gestalt sind, beispielsweise bieten sich oft polynomielle Darstellungen niedrigen Grades an. 
Ein solches Teilgebiet $K \in \mathfrak{K}$ nennen wir Finites Element oder auch Zelle und fordern implizit verbunden mit dem betrachteten Funktionenraum die Erfüllung der obigen Definition. \newline
Im Falle $\mathbb{D} \subseteq \R^2$ kommen so z.B. Dreiecke oder Vierecke in Frage, in $\mathbb{D} \subseteq \R^3$ können Tetraeder, Würfel, Quader und andere genutzt werden. \newline
Sei nun $\mathbb{D} \subseteq \R ^2$ zudem ein polygonales Gebiet, um eine einfache Zerlegung in Dreiecke oder Vierecke zu gewährleisten.

\begin{Definition}
	\begin{enumerate}
		\item Eine Zerlegung $\mathfrak{K} = \{ K_1,K_2,\dots,K_M\}$ von $\mathbb{D}$ in Dreiecks- oder Viereckselemente heißt zulässig, wenn folgende Eigenschaften erfüllt sind:
		\begin{itemize}
			\item $\overline{\mathbb{D}} = \bigcup_{i=1}^M K_i$
			\item Für $i \neq j$ ist $K_i\cap K_j$
			\begin{enumerate}
				\item ein gemeinsamer Eckpunkt von $K_i$ und $K_j$
				\item eine gemeinsame Kante von $K_i$ als auch von $K_j$
				\item oder $K_i\cap K_j= \emptyset$
			\end{enumerate}
			
		\end{itemize}
		\item Wir schreiben oft $\mathfrak{K}_h$ anstatt $\mathfrak{K}$, wenn jedes Element einen Durchmesser von höchstens $2h$ besitzt [Vorlesung h = max diam K was passt besser]
	\end{enumerate}
\end{Definition}

\begin{figure}[h]
	\centering
	\captionabove{Zulässige Triangulierung und unzulässige Triangulierung mit hängendem Knoten}
	\includegraphics[width=0.8\textwidth]{triangulierung.png} \\
	Abbildung aus \cite{braess2013finite} Seite 58
\end{figure}

\subsubsection{Schwache Formulierung}
Betrachten wir also das Potentialströmungsproblem (NEW ALIGNMENT? do everywhere):
\[ \text{Bestimme } u:\overline{\mathbb{D}} \to \R \text{ und } q: \overline{\mathbb{D}} \to \R^2 \text{ mit } \newline \]
\[\setlength\arraycolsep{1pt}
\text{(PS)}\begin{cases} 
\begin{array}{rlcl}
\dive q     &= 0                 &\text{ ,} \text{in } \mathbb{D}\\
q           &= - \kappa \nabla u &\text{ ,} \text{in }\mathbb{D}\\
u           &= u_D               &\text{ ,} \text{auf } \Gamma_D \\
-q \cdot n  &= g_N               &\text{ ,} \text{auf } \Gamma_N 
\end{array}
\end{cases} 
\]

\subsection{Numerische Lösung des Transportproblem}

\subsection{Anwendung der MLMC Methode auf das Transportproblem für unsichere Permeabilität $\kappa$ }
%%%%%%%%%%%%%%%%%%%%%%%%%%%%%%%%%
\newpage  % neuer Abschnitt auf neue Seite, kann auch entfallen
%%%%%%%%%%%%%%%%%%%%%%%%%%%%%%%%%
\section{Beispiel/Experiment}
\subsection{Konkretes Problem}

\subsection{Ergebnisse}
%%%%%%%%%%%%%%%%%%%%%%%%%%%%%%%%%
\newpage  % neuer Abschnitt auf neue Seite, kann auch entfallen
%%%%%%%%%%%%%%%%%%%%%%%%%%%%%%%%%
\section{Ausblick und Fazit}

  % Literaturverzeichnis (beginnt auf einer ungeraden Seite)
  \newpage

%\begin{thebibliography}{Lam00}
 %Bibiographie
  \bibliographystyle{abbrv}
  \bibliography{References}
%\end{thebibliography}
 
      
  % ggf. hier Tabelle mit Symbolen 
  % (kann auch auf das Inhaltsverzeichnis folgen)

\newpage
  
 \thispagestyle{empty}


\vspace*{8cm}


\section*{Erkl\"arung}

Ich  versichere  wahrheitsgem\"a\ss,  die  Arbeit selbstst\"andig verfasst,  alle  benutzten  Hilfsmittel  vollst\"andig  und  genau  angegeben  und  alles kenntlich  gemacht  zu  haben,  was  aus  Arbeiten  anderer  unver\"andert  oder  mit  Ab\"anderungen entnommen  wurde,  sowie die Satzung  des  KIT  zur  Sicherung guter wissenschaftlicher Praxis in der jeweils g\"ultigen Fassung beachtet zu haben.
\\[2ex] 

\noindent
Ort, den Datum\\[5ex]

% Unterschrift (handgeschrieben)



\end{document}


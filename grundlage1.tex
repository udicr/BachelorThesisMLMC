% !TeX root = bachelorarbeit.tex
Sei $\mathcal{D} \subseteq \R^d$ offen für $d\in \N$ und $\lVert \cdot \rVert$ eine Norm auf $\R^d$.
Die folgenden Definitionen und Sätze sollen als Grundlagen für die weiteren Betrachtungen dieser Thesis dienen. Insbesondere wollen wir hierbei meist auf konkrete Beweise verzichten und verweisen dahingehend auf die Literatur. 
Die analytischen Grundlagen bauen zum Teil auf der Vorlesung Rand- und Eigenwertprobleme aus dem Sommersemester 2019 von Herrn Prof. Dr. Reichel auf, sind aber auch  z.B. in \cite{dobrowolski2010angewandte} oder \cite{evans10} zu finden.
\begin{Definition}(Einige Operatoren)
	\begin{enumerate}[label=(\alph*)]
		\item Für $F: \R^d \to \R^d$ ist die Divergenz von F definiert durch
			\begin{align*}
				\dive  : \begin{cases}
				\R^d  \to \R \\
				F \mapsto \dive F= \nabla \cdot  F \coloneqq \sum_{i=1}^{d} \frac{\partial F_i}{ \partial x_i}
				\end{cases}
			\end{align*}
		\item Für $f: \R^d \to \R$ und $\alpha = (\alpha_1,\dots,\alpha_d) \in \N^d$ ist die partielle Ableitung von $f$ nach dem sogenannten Multiindex $\alpha$ definiert durch
			\begin{align*}
				\partial^{\alpha}f \coloneqq 
				\frac{\partial^{|\alpha|} f}{\partial x_1 ^{\alpha_1} \cdots  \partial x_d^{\alpha_d} } 
				=\frac{\partial^{\alpha_1+\dots +\alpha_d} f}{\partial x_1 ^{\alpha_1} \cdots  \partial x_d^{\alpha_d} } 
			\end{align*}
	\end{enumerate}
\end{Definition}

%\begin{Satz}
%	Sei $ \mathcal{D} \subset \R^d $ offen und beschränkt und $ 1 \leq p < \infty $, dann gilt: \\
%	$ C_c^{\infty} (\mathcal{D})$ liegt dicht in $ L^p(\mathcal{D}) $, d.h. $ C_c^{\infty}(\mathcal{D}) \subseteq L^p(\mathcal{D})$ und $ \overline{ C_c^{\infty}(\mathcal{D}) }^{\lVert \cdot \rVert_p}  = L^p(\mathcal{D})$
%\end{Satz}


%\begin{Definition}(Lipschitz-Gebiet TODO cite rwp) \newline 
%	\begin{enumerate}[label=(\alph*)]
%		\item Eine offene zusammenhängenden Menge $\mathcal{D} \subset \R^d$ heißt Lipschitz-Gebiet, falls für jedes $x_0 \in \partial \mathcal{D}$ ein Radius $r>0$ und eine Lipschitz-stetige Funktion $\phi : \R^{d-1} \to \R$ existiert, so dass (nach einer geeigneten Bewegung des Koordinatensystems) gilt: \\
%		Für $ x' = (x_1,\dots,x_{d-1})  \text{ und } B_r(x_0) \coloneqq \{ x\in \R^d : \lVert x-x_0 \rVert < r\}  \text{ ist }$
%			\begin{align*}
%				\mathcal{D} \cap B_r(x_0) = \{x = (x',x_d) \in B_r(x_0): x_d > \phi(x')\} 
%			\end{align*}
%		Es gilt dann notwendigerweise:
%			\begin{align*}
%				\partial \mathcal{D} \cap B_r(x_0) = \{x = (x',x_d) \in B_r(x_0): x_d = \phi(x')\}
%			\end{align*}
%		\item Ist $\mathcal{D}$ ein Lipschitz-Gebiet so existiert $x = (x',\phi(x')) \in \partial \mathcal{D} \cap B_r(x_0)$ der Vektor
%			\begin{align*}
%				n(x) = \frac{1}{\sqrt{1+|\nabla \phi(x')|^2}} \icol{\nabla \phi(x')\\-1}
%			\end{align*}
%		fast überall bezüglich des Oberflächenmaßes auf $\partial \mathcal{D}$ und heißt der äußere 
%		Einheitsnormalenvektor an $\partial \mathcal{D}$ im Punkt $x$. 
%	\end{enumerate}
%	
%\end{Definition}


\begin{Satz} (Gaußscher Integralsatz für Lipschitz-Gebiete) \newline
 	Sei $\mathcal{D}  \subset \R^d$ ein beschränktes Lipschitz-Gebiet und sei $n$ der äußere Einheitsnormalenvektor an $\partial \mathcal{D}$. Dann gilt:
 		\begin{align*}
	 		\int_{\mathcal{D}} \frac{\partial f}{\partial x_i} \dx  = \int_{\partial \mathcal{D}} f n_i \da
 		\end{align*}
 		für jede Funktion $f \in C^1(\overline{\mathcal{D}})$. \\ 
 		Oft erscheint der Gaußsche Integralsatz auch in folgender Form:
 		\begin{align*}
 		\int_{\mathcal{D}} \dive F \dx =  \int_{\partial \mathcal{D}} F \cdot n \da
 		\end{align*}
 		wobei $F:\mathcal{D} \to \R^d$ ein Vektorfeld ist. Die Komponentenfunktionen von $F = (f_1,\dots,f_d)$ sollen dann $f_i \in C^1(\overline{\mathcal{D}})$ für $i=1,\dots,n$ erfüllen.
\end{Satz}

\begin{Folgerung}(mehrdimensionale partielle Integration) \\
	\label{n_pI}
	Sei $ f \in C^1(\R^d, \R)$ und $ F : \mathcal{D} \to \R^d $ ein stetig partiell differenzierbares Vektorfeld.
	Dann gilt:
	
	\begin{align*}
		\int_{\mathcal{D}} f \dive ( F) \dx = \int_{\partial \mathcal{D}} f F \cdot n \da
		- \int_{\mathcal{D}}  F \cdot \nabla f \dx
	\end{align*}
\end{Folgerung}


%\begin{Definition}($ L^1_{\text{loc}} $)\\
%	Todo
%\end{Definition}



\begin{Definition}(schwache Ableitung)\\
	Sei $f \in L_{\text{loc}}^1(\mathcal{D})$. Dabei bezeichne $  L_{\text{loc}}^1(\mathcal{D}) $ den Raum der lokal integrierbaren Funktionen auf $ \mathcal{D} $. Wir sagen v besitzt eine schwache Ableitung zum Multiindex $\alpha$, falls eine Funktion $v \in L_{\text{loc}}^1$ existiert, mit 
		\begin{align*}
			\int_{\mathcal{D}} f \partial^{\alpha} \phi \dx = (-1)^{|\alpha|} \int_{\mathcal{D}} v \phi \dx \qquad \forall \phi \in C_c^{\infty}(\mathcal{D})
		\end{align*}
	In diesem Zusammenhang nennen wir $\Phi$ auch Testfunktion und wir definieren $D^{\alpha} f \coloneqq v$ als die schwache Ableitung von $f$ zum Multiindex $\alpha$. 
\end{Definition}
\begin{Bemerkung}
	Per Konvention ist für $ \alpha = (0,\dots, 0) \quad \partial^{\alpha}f = f $
\end{Bemerkung}
\begin{Definition}(Sobolevräume)\\
	Sei $\mathcal{D} \subseteq \R^d $ offen, $ k \in \N $ und $ 1 \leq p \leq \infty $.  Weiter sei $ L : C^1(\mathcal{D},\R^m) \to L^{\infty}(\mathcal{D},\R^k) $ ein linearer Differentialoperator erster Ordnung und $ L^{\star} : C^1(\mathcal{D},\R^k) \to L^{\infty}(\mathcal{D},\R^m) $ der zugehörige adjungierte Operator. Es gelte also
	\[
	\int_{\mathcal{D}} Lf \cdot \phi \dx = \int_{\mathcal{D}} f \cdot L^{\star} \phi \dx \text{ für } f \in C^1_c(\mathcal{D},\R^m) \text{ , } \phi \in C^1_c (\mathcal{D},\R^k)
	\]
	Dann sind:
	\begin{enumerate}[label=(\alph*)]
		\item $ W^{k,p} (\mathcal{D}) \coloneqq \{ f \in L^p(\mathcal{D})$ und die schwachen Ableitungen $ \partial^{\alpha}f $ existieren, mit $ \partial^{\alpha}f \in L^p(\mathcal{D}) $ für alle $ \alpha \in \N_0^d , |\alpha| \leq k \} $	
		\item $ \lVert f \rVert_{k,p} =  \lVert f \rVert_{W^{k,p}(\mathcal{D})} \coloneqq 
				\begin{cases}
					\begin{array}{lll}
						(\sum\limits_{|\alpha|\leq k} \int_{\mathcal{D}} |\partial^{\alpha} f |^p \dx)^{\frac{1}{p}} &, 1 \leq p < \infty \\
						\sum\limits_{|\alpha|\leq k}   \lVert \partial^{\alpha} f \rVert_{\infty}        &, p = \infty
					\end{array}
				\end{cases}  $
		\item $ W_0^{k,p}(\mathcal{D}) \coloneqq \overline{ C_c^{\infty}(\mathcal{D}) }^{\lVert \cdot \rVert_{k,p}} $. Über den sogenannten Spursatz erhält man eine äquivalente Charakterisierung durch: 
		$ W_0^{k,p}(\mathcal{D}) = \{ f \in W^{k,p}(\mathcal{D}) : f|_{\partial \mathcal{D}} = 0 \}$
		\item Im Falle $ p = 2 $ schreibt man aufgrund der Tatsache, dass es sich dann bei $ W^{k,p}(\mathcal{D}) $ um einen Hilbertraum handelt, oft auch $ H^k(\mathcal{D}) \coloneqq W^{k,p}(\mathcal{D}) $
		\item $ H(L,\mathcal{D}) \coloneqq \{ f \in L^2(\mathcal{D},\R^m) : \exists v \in L^2(\mathcal{D},R^k) \text{ mit } \int_{\mathcal{D}} v \cdot \phi \dx = \int_{\mathcal{D}} f \cdot L^{\star}\phi \dx \ \forall \phi \in C^1_c(\mathcal{D},\R^k) \}$
	\end{enumerate}
\end{Definition}

\begin{Satz}(Multiplikation mit Testfunktionen und Integration) \\ 
	\label{testfunktionen}
	Sei $\mathcal{D} \subset \R^d$ offen und $f \in L_{\text{loc}}^1(\mathcal{D})$ und $\int_{\mathcal{D}} f \phi \dx = 0 \; \forall \phi \in C_c^{\infty}(\mathcal{D})$ ,\\
	dann gilt $ f \equiv 0 $.
\end{Satz}
%\begin{Satz} (Rechnen mit Differentialoperatoren)
%	todo FEM
%\end{Satz}
%\begin{Definition}
%	Hdiv? todo
%\end{Definition}
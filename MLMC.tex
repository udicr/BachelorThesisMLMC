% !TeX root = bachelorarbeit.tex
Nachdem wir im letzten Abschnitt die Monte Carlo Methode betrachtet haben, wollen wir uns nun einer Weiterentwicklung der Monte Carlo Methode, der sogenannten Multi Level Monte Carlo Methode zuwenden. Grundsätzlich liegt dieselbe Situation vor wie bei der Monte Carlo Methode:
Wir wollen wieder eine Größe bestimmen, welche sich nach geeigneter Modellierung in der Form eines Erwartungswertes $ \mathbb{E}[X] $ einer Zufallsvariablen $ X $ schreiben lässt. Besonders wenn diese Größe mit der Lösung von gewöhnlichen oder partiellen Differentialgleichungen, wie wir sie später betrachten wollen, hat man nun jedoch die Wahl, wie genau die numerische Lösung des zugrunde liegenden Problems, z.B. der Differentialgleichung, erfolgen soll. Beispielsweise können wir im Falle der numerischen Lösung von Differentialgleichung Zeitschrittweiten und/oder Gitterweiten der Ortsdiskretisierung festlegen. An dieser Stelle tritt stets ein typischer Zwiespalt auf:
\begin{itemize}
	\item Zu Einen wollen wir möglichst genau rechnen. Dies legt die Wahl von besonders kleinen Zeitschrittweiten bzw. feinen Gittern zur Ortsdiskretisierung nahe.
	\item Zum Anderen wollen wir die Anzahl der Rechenschritte bzw. die Rechenzeit möglichst gering halten. Dies spricht hingegen für große Zeitschritte bzw. grobe Gitter.
\end{itemize}
Zusätzlich zu der oft bereits alleine anspruchsvollen Aufgabe solche Probleme numerisch zu lösen, müssen wir also stets einen für unsere Bedürfnisse passenden Kompromiss aus möglichst genauer numerischer Approximation und geringem (oder zumindest machbarem) Rechenaufwand eingehen. Obwohl dies zunächst wie eine zusätzliche Hürde erscheint und Mehraufwand vermuten lässt, stellt sich heraus, dass solche eine Wahl der Genauigkeit im Kontext von Monte Carlo Methode sich durchaus als Nützlich erweisen kann. 
Die Multi Level Monte Carlo Methode, wir werden im folgenden auch oft vom sogenannten Multi Level Monte Carlo Schätzer sprechen, ist der Prototyp einer Familie sogenannter Varianz-reduzierender Methoden, welche das Ziel haben die naive Monte Carlo Methode in Sachen Konvergenzrate und Effizienz zu schlagen.